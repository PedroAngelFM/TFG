\newpage
\section{Análisis Factorial}
\noindent El análisis factorial es un conjunto de técnicas que en primera instancia fue desarrollado en el campo de la psicología ya que estos buscaban encontrar las causas que provocaban la variabilidad de los datos que recogían \cite{Vincent 1953}. 

\noindent En 1888 \emph{Galton, F} escuchó la exposición de un articulo que relacionaba las sociedades que habían abandonado un modelo matriarcal con una estructura más compleja sociológicamente hablando. Ante esto, \emph{Galton} no estuvo de acuerdo, ya que detectó un fallo en el razonamiento de su compañero y es por ello, que escribió el artículo \cite{Galton 1889} dando pie a los trabajos sobre el estudio de las correlaciones.   

\noindent Es en 1904 cuando \emph{Spearman, C.} \cite{Spearman 1904} viendo el trabajo de \emph{Galton, F.} aplicó esas ideas al estudio de la inteligencia, buscando una medida objetiva de la misma. En este desarrollo, dio las primeras nociones de qué era un factor y calculó una aproximación para un modelo factorial con 6 factores comunes. A pesar de que no especifica el modelo usado, se ha comprobado a posteriori  que eran bastante precisos. A partir de este momento, fue cuando el análisis factorial empezó a ser importante y varios psicólogos comenzaron a usarlo de forma habitual. 
Desde hace años el Análisis factorial, al igual que otras técnicas multivariantes, ha experimentado un nuevo impulso en el desarrollo teórico y práctico debido al auge de la computación. 

\noindent El Análisis factorial nació con un caracter confirmatorio, en el que se tenían unas ciertas ideas preconcebidas y se buscaba aceptar o rechazar dichas ideas. Actualmente, la mayoría de los investigadores lo utilizan de manera exploratoria para extraer información de los datos.\cite{Hair 1995}

\noindent El objetivo específico del Análisis factorial es, dado un conjunto de variables aleatorias, construir un modelo en el cual con un conjunto de nuevas variables llamadas \emph{factores comunes}, que expliquen la mayor variabilidad posible de las variables iniciales. De esta manera, se descompondrá la varianza de cada una de las variables en una parte común llamada \emph{comunalidad}  y otra específica denominada \emph{especificidad}. 

\noindent La mayor diferencia con el análisis de componentes principales en el que se busca nuevas direcciones que maximicen la varianza explicada, es que en este caso, son de interés tanto las comunalidades como las especifidades ya que nos permiten analizar si una variable da información de manera o común o tiene algo diferencial. 

\noindent Para empezar, empezaremos formalizando el modelo factorial, en particular el modelo multifactorial ortogonal \cite{Johnson 2007} sin pérdida de generalidad. Tras la modelización, se desarrolla el método de estimación de las matrices de carga mediante el método del factor principal \cite{Peña 2002} y se mencionarán los fundamentos de otros métodos como el de máxima verosimilitud. Seguidamente, se darán algunos detalles sobre el problema de la no unicidad del modelo y soluciones ante ello. Y por último, se dará una introducción a la interpretación de los datos que arroja el modelo factorial. 

\newpage
\subsection{Formalización}
\noindent Supóngase un vector aleatorio $\mathbf{x}$ de longitud $p$ con una distribución $N_p(0,\Sigma)$, centrada sin pérdida de generalidad, donde la matriz de covarianzas $\mathbf{\Sigma}$ es simétrica y definido positiva. El modelo factorial afirma lo siguiente \cite{Chatfield 1989}:  
\begin{equation}\label{eq Fact}
X_j= \lambda_{1j}F_1+\ldots+\lambda_{mj}F_m+\psi_j U_j\quad j=1\ldots p 
\end{equation}
\noindent Donde:
\begin{itemize}
\item $\lambda_{jk}$ son las \emph{cargas del factor común} $k$-ésimo de la $j$-ésima variable.
\item $F_k$ es el $k$-ésimo factor común
\item $\psi_j$ es la \emph{carga específica} de la variable $X_j$
\item $U_j$ es el \emph{factor específico} para la variable $X_j$
\end{itemize}

\noindent En este caso haremos las siguientes suposiciones\cite{Cuadras 2014}:
\begin{itemize}
\item Los factores comunes $F_k$ son variables aleatorias que siguen una distribución marginal $N(0,1)$. Además se supondrá que $Cov(F_k,F_{k'})=0, k\neq k'$, de manera que el vector aleatorio $\mathbf{f}$ de longitud $m$ con distribución $N_m(0,\mathbf{I})$. Y son completamente independientes de los factores específicos. 

\item Los factores específicos $U_j$ son variables aleatorias con una distribución normal $N(0,1)$ no correladas, entre ellas  que forman el vector aleatorio $\mathbf{u}\sim N_p(0,\mathbf{I})$ de longitud $p$. Son completamente independientes de los factores comunes. 
\end{itemize}

\noindent De esta manera se pueden definir las siguientes matrices: 

\begin{defi}
Llamaremos \emph{matriz de cargas de los factores o matriz de cargas}, $\mathbf{\Lambda}$ de tamaño $p \times m$ a la matriz :
\begin{equation}
\mathbf{\Lambda}=\begin{pmatrix}
\lambda_{11} & \cdots & \lambda_{1 m}\\
\vdots & \ddots & \vdots\\
\lambda_{p1} & \cdots & \lambda_{pm}
\end{pmatrix}
\end{equation}
\end{defi}

\begin{defi}
Llamaremos matriz específica a la matriz diagonal $\mathbf{\Psi}$ de tamaño $p\times p$ a aquella que tiene los términos $\psi_j$ donde $j=1,\ldots , p$:
\begin{equation}
\mathbf{\Psi}=\begin{pmatrix}
    \psi_1 & 0 & \dots & 0 \\
    0 & \psi_2 & \dots & 0 \\
    \vdots & \vdots & \ddots & \vdots \\
    0 & 0 & \dots & \psi_p
\end{pmatrix}
\end{equation}
\end{defi}

\noindent \emph{Observación:} En distintas fuentes no aparece la matriz $\mathbf{\Psi}$ ya que los factores específicos $U_j, j=1\ldots p$ en esos casos no se consideran variables estandarizadas no correladas, como sería el caso descrito. Por ejemplo,  \emph{Mardia, K.V.} \cite{Mardia 1979} denotan como $\mathbf{\Psi}$ a la matriz de covarianzas del vector $\mathbf{u}$.

\noindent De esta manera, se puede dar una versión matricial de la expresión \ref{eq Fact}:
\begin{equation}
\mathbf{x}^T=\Lambda \mathbf{f}^T+ \mathbf{\Psi}\mathbf{u}^T
\end{equation}
Ya que se consideran los vectores aleatorios $\mathbf{x}, \mathbf{f}, \mathbf{u}$ vectores ordenados como filas. 

\begin{propo}
La varianza de la variable aleatoria $X_j$, $\sigma_j^2$ se puede descomponer de la siguiente manera:
\begin{equation}
\sigma_j^2 = \sum_{k=1}^{m}\lambda_{kj}^2+\psi_j^2
\end{equation}
\begin{proof}
\begin{align*}
\mathbb{E}(X_j^2)&=\sigma_j^2 = \sum_{k=1}^{m}\lambda_{kj}^2\mathbb{E}(F_k
^2)+\sum_{k=1}^{m}\sum_{n=1,n\neq k}^{m} \lambda_{kj}\lambda_{nj}\mathbb{E}(F_k\cdot F_n)\\
&+\sum_{k=1}^{m}\lambda_{kj}\psi_j \mathbb{E}(F_k\cdot U_j)+\psi_j^2\mathbb{E}(U_j^2)
\intertext{Utilizando las hipótesis del modelo:}
&\ \sigma_j^2=\sum_{k=1}^{m}\lambda_{kj}^2+\psi_j^2
\end{align*}
\end{proof}
\end{propo}
\noindent Teniendo en cuenta esta descomposición se forma el siguiente concepto.
\begin{defi}
Se llama \emph{comunalidad} de la variable $X_j$, $h_j^2=\sum_{k=1}^m \lambda_{kj}^2 $ \cite{Peña 2002}
\end{defi} 
\noindent Esto permite interpretar la varianza de la variable como la varianza explicada por los factores comunes por un lado y la variabilidad específica de la propia variable. 
\begin{propo}
La covarianza de dos variables $Cov(X_j, X_j')$, $\sigma_{jj'}$ cumple que \cite{Morrison 1976, Chatfield 1989}: 
\begin{equation}
\sigma_{jj'}=\sigma_{j'j}=\sum_{k=1}^{m}\lambda_{kj}\lambda_{kj'}
\end{equation}
\begin{proof}
\begin{align}
\sigma_{jj'} &= \sum_{k=1}^{m}\lambda_{kj}\lambda_{kj'}\mathbb{E}(F_k
^2)+\sum_{k=1}^{m}\sum_{n=1,n\neq k}^{m} \lambda_{kj}\lambda_{nj'}\mathbb{E}(F_k\cdot F_n)\\
&+ \sum_{k=1}^{m}\lambda_{kj}\psi_{j'} \mathbb{E}(F_k\cdot U_{j'})+\sum_{k=1}^{m}\lambda_{kj'}\psi_j\mathbb{E}(F_k\cdot U_j)+\psi_j\psi_{j'}\mathbb{E}(U_j\cdot U_{j'})
\intertext{De nuevo, teniendo en cuenta las hipótesis}
\sigma_{jj'}=\sigma_{j'j}&=\sum_{k=1}^{m}\lambda_{kj}\lambda_{kj'}
\end{align}
\end{proof}
\end{propo}

\noindent Y tomando estas dos proposiciones se puede tener que la matriz $\mathbf{\Sigma}$, cumple lo siguiente:
\begin{coro}\label{Descomposición Varianza}
La matriz $\mathbf{\Sigma}=\mathbf{\Lambda}\mathbf{\Lambda}^T+\mathbf{\Psi}$
\end{coro}

\noindent Bajo esta idea en la que la varianza se puede descomponer en una parte específica y una común a todas las variables. 

\subsection{Estimación de la matriz de cargas}
\noindent Supóngase el vector aleatorio $x$ como antes del cual se recogen $N$ observaciones para obtener la matriz de datos $\mathbf{X}$ de tamaño $N\times p $que se supondrá centrada sin pérdida de generalidad. Esto provoca que la matriz $\mathbf{S}=\mathbf{\hat{\Lambda}}\mathbf{\hat{\Lambda}}^T+\mathbf{\hat{\Psi}}$ \cite{Peña 2002}. 

\noindent Entonces el objetivo es estimar la matriz de cargas $\mathbf{\Lambda}$, para ello se puede tomar la matriz $\mathbf{S}-\mathbf{\hat{\Psi}}$, si de esta matriz se obtiene la descomposición espectral:
\begin{equation}
\mathbf{S}-\mathbf{\hat{\Psi}}=\mathbf{U}\mathbf{D}\mathbf{U}^T
\end{equation}

\noindent Donde la matriz $\mathbf{D}$ es una matriz diagonal de tamaño $r\times r$ con los $r=rg(\mathbf{X})$ valores propios no nulos, y la matriz $\mathbf{U}$ de tamaño $p\times r$ de manera las columnas son los vectores propios asociados a los valores propios no nulos. 

\noindent De esta manera, \emph{Cuadras C.M;} \cite{Cuadras 2014} toma como estimador de $\mathbf{\hat{\Lambda}}=\mathbf{UD}^{\frac{1}{2}}$, ya que $\mathbf{\hat{\Lambda}}\mathbf{\hat{\Lambda}}^T=\mathbf{U}\mathbf{D}\mathbf{U}^T= \mathbf{S-\hat{\Psi}}$. El problema que surge aquí es que la estimación $\mathbf{\hat{\Psi}}$ no es conocido, pero se desarrollará en lo siguiente. 

\noindent De esta manera, supóngase que es conocido el número de factores, $m$. Por tanto, como el objetivo es que los factores comunes es maximizar su varianza común explicada, entonces siguiendo el mismo razonamiento que en las componentes principales, se puede tomar como estimación de las matrices factoriales la matriz siguiente:

\begin{equation}
\mathbf{\hat{\Lambda}}=(\sqrt{\mu_1}e_1,\ldots,\sqrt{\mu_m}e_m)=\mathbf{U}_m\mathbf{D}_m^{\frac{1}{2}}
\end{equation}

\noindent Donde la pareja $(\mu_i, e_i)$ es el vector propio $e_i$ en formato columna y $\mu_i$ su valor propio asociado. Además la matriz $\mathbf{U}_m$ es la matriz que tiene como columnas los $m$ vectores propios de la matriz $\mathbf{S}-\mathbf{\Psi}$ que tienen asociados los $m$ valores propios más grandes. 

\noindent El problema de que no se conoce $\mathbf{\hat{\Psi}}$, se puede resolver planteando el siguiente método iterativo, para ello se introduce la notación:
\begin{itemize}
\item $\mathbf{S}_i^*=\mathbf{S}-\hat{\mathbf{\Psi}}_i$ es la matriz reducida en la $i$-ésima iteración del método
\item $\mathbf{\hat{\Lambda}}_i$ es la matriz de cargas obtenida en la $i$-ésima iteración del método. 
\end{itemize}
\noindent El método se inicializa con $\mathbf{\hat{\Psi}}_0=0$ de tal manera que $\mathbf{S}_0^*=\mathbf{S}$ y consiste en cada iteración hacer las siguientes operaciones \cite{Johnson 2007, Peña 2002, Cuadras 2014}: 
\begin{enumerate}
\item Se calcula $\mathbf{S}_i^*=\mathbf{S-\hat{\Psi}}_i$
\item Se calcula la descomposición $\mathbf{\hat{\Lambda}}_i$ como se ha detallado antes obteniendo $\mathbf{U}_i$ y $\mathbf{D}_i$ de tamaños $p\times m $ y $m\times m$. 
\item Se obtiene la nueva estimación de la $\mathbf{\hat{\Psi}}_{i+1}=\mathbf{S}-\mathbf{\hat{\Lambda}}_i\mathbf{\hat{\Lambda}}_i^T$
\end{enumerate}

\noindent El proceso se para cuando no haya cambios en las matrices $\mathbf{\hat{\Psi}}_i,\mathbf{\hat{\Psi}}_{i+1}$ o cuando $\mathbf{\hat{\Lambda}}\mathbf{\hat{\Lambda}}^T+\mathbf{\Psi}$ sea lo más parecido posible a la matriz de covarianzas $\mathbf{S}$.

\noindent \emph{Peña D} \cite{Peña 2002} lo generaliza utilizando la función $F=tr(\mathbf{S}-\mathbf{\Lambda}\mathbf{\Lambda}^T-\Psi)^2$. Que en esencia es minimizar la distancia de Frobenius, implicando un razonamiento desde la optimización, ya que se busca minimizar  una función de pérdida, la distancia entre ambas matrices en este caso. 

\noindent Una vez estimadas las cargas de los factores, se pueden estimar las \emph{puntuaciones de los factores}, que son los valores que toman los factores pero no se detallarán en esta memoria.  \emph{(Para más detalles de esta parte véase  la sección 9.5 de \cite{Johnson 2007} o la 12.7 de \cite{Peña 2002})}

\subsection{Unicidad del modelo}

\noindent Un detalle en el que no se ha profundizado, es que el modelo factorial, no es único, es decir, teniendo en cuenta las suposiciones hechas anteriormente, todas las rotaciones de los factores $\mathbf{Tf}^T$, donde $\mathbf{T}$ es una matriz ortogonal de tamaño $m\times m$, entonces al transformar la matriz $\mathbf{\Lambda}$ a la matriz $\mathbf{\Lambda T}^T$ se cumple que \cite{Mardia 1979}:
\begin{equation}
\mathbf{x}^T=(\mathbf{\Lambda T}^T)(\mathbf{Tf}^T)+\Psi\mathbf{u}^T
\end{equation}
\noindent Es equivalente al modelo \ref{eq Fact}, ya que se cumplen las mismas suposiciones. 

\noindent Para evitar estas indeterminaciones se suelen tomar dos restricciones \cite{Mardia 1979}:
\begin{equation}
\mathbf{\Lambda}^T \mathbf{\Psi}^{-1}\mathbf{\Lambda} \text{ ó } \mathbf{\Lambda}^T \mathbf{\Lambda} \text{ son diagonales }
\end{equation}
\noindent En particular, la segunda restricción es la que se toma en el método del factor principal, ya que $\mathbf{\hat{\Lambda}}=\mathbf{UD}^{\frac{1}{2}}$, donde $\mathbf{U}$, es una matriz ortogonal, por tanto, $\mathbf{\hat{\Lambda}^T \hat{\Lambda}}=\mathbf{D}^{\frac{1}{2}}\mathbf{U}^T\mathbf{U}\mathbf{D}^{\frac{1}{2}}=\mathbf{D}$. 

\begin{propo}
La segunda restricción hace que el modelo sea único \cite{Peña 2002}. 
\begin{proof}
Sea la descomposición de la varianza \emph{(Corolario \ref{Descomposición Varianza})}: \begin{align*}
\mathbf{\Sigma}&=\mathbf{\Lambda\Lambda}^T+\mathbf{\Psi}\\
\mathbf{\Sigma}-\mathbf{\Psi}&=\mathbf{\Lambda\Lambda}^T\\
\intertext{Si postmultiplicamos por $\mathbf{\Lambda}$ a ambos lados:}
(\mathbf{\Sigma}-\mathbf{\Psi})\mathbf{\Lambda}&=\mathbf{\Lambda\Lambda}^T \mathbf{\Lambda}\\
\intertext{ Como se supone que $\mathbf{\Lambda}^T \mathbf{\Lambda}=\mathbf{D}$ se tiene que:}
(\mathbf{\Sigma}-\mathbf{\Psi})\mathbf{\Lambda}&=\mathbf{\Lambda D}
\end{align*}
Por tanto, la matriz $\mathbf{\Lambda}$ tiene por columnas los $m$ vectores propios de la matriz $\mathbf{\Sigma-\Psi}$, como se ha señalado anteriormente, esta propiedad ha sido usada anteriormente en el método del factor principal, es decir, la estimación por este método es única. 
\end{proof}
\end{propo}
\subsection{Interpretación del modelo factorial}

\noindent Para interpretar el modelo factorial hay que tener en cuenta los siguientes aspectos:
\begin{itemize}
\item \emph{Proporción de varianza común explicada} Es decir, analizar que parte de la varianza se puede explicar por los factores comunes. A mayor sea, más relación entre las variables explican los factores, 
\item \emph{Especificidad de cada una de las variables}: De esta manera, se puede analizar si una variable está alejada de las demás y no tiene tanto que ver con el resto. 
\item \emph{Cargas de cada uno de los factores en cada variable} Esto también nos puede dar datos, por ejemplo si un grupo de variables tiene cargas mayores en uno de los factores $F_i$ y en otro grupo es más prevalente otro factor, eso también es significativo. 
\end{itemize}

\noindent \emph{Johnson y Wichern } analizan en el capitulo 9 de \cite{Johnson 2007} un ejemplo con un conjunto de datos de varias medidas de aves. 

\noindent Como medidas observables tienen las medidas de 6 huesos y aplican el modelo factorial con 3 factores. Si nos centramos en los datos obtenidos aplicando el método del factor principal \cite{Peña 2002, Johnson 2007}, lo primero a destacar es que 3 factores consiguen acumular un 95\% de la varianza, por lo tanto, es posible afirmar que hay 3 variables latentes, a las que luego daremos interpretación. Por otro lado, la varianza específica es poca o nula en todas las variables, por lo tanto, nos reafirma en nuestra sospecha de que hay 3 variables latentes. 

\noindent Por último, podemos ver que las cargas que tiene en todas el primer factor son cercanas a 1 por tanto tiene gran importancia en todas. Sin embargo el 3er factor en las ultimas 4 variables no tiene esa misma importancia. Por tanto el 3er factor casi no influye en las últimas medidas. 

\noindent \emph{Johnson, R.A. y Wichern, D.W.}\cite{Johnson 2007} además utilizan el método de máxima verosimilitud y utilizan rotaciones para mejorar la interpretación. 

