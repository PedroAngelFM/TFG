\chapter{Resumen PCA}
\section{Introducción}
Reducir la dimensionalidad de un conjunto de datos estadísticos siempre es una buena opción, sobre todo en el caso del aprendizaje automático. En ese campo una reducción de la dimensionalidad puede incurrir en paliar problemas como el overfitting \textit{(fallos del modelo al generalizar a nuevos datos.)}

En esencia esta es una técnica relacionada de manera directa con las variables, sobre todo aquellas que están igualdad de condiciones, no teniendo variables explicatorias o dependientes unas de otras.

El objetivo de este analisis es ver si unas pocas variables pueden representar la mayoría de la variación de los datos. Si esto se da podemos reducir la dimensionalidad del problema y será menor que $p$.
Podemos decir, que si varias variables están altamente correladas, expresan la misma información