\section{Análisis de Componentes Principales}
\subsection{Introducción}

%
\textit{Aquí va la intro de la seccion de las componentes principales.}
\noindent Dado un conjunto de datos, recogidos en una matriz \textbf{X} que llamaremos matriz de datos. Estos datos provienen de tomar observaciones de $p$ variables de manera simultánea. 

\noindent Cada vector fila es una observación $\textbf{x}_i$ que se puede pensar como un elemento del espacio $\mathbb{R}^p$. Cuando el numero de variables a estudiar excede 



%\noindent Sea una población de la que se han tomado $n$ observaciones de las cuales hemos medido $p$ variables. Del conjunto de datos que resulta, podemos dar la matriz de datos $\mathbf{X}$ de tamaño $n\times p$. 
%
%\noindent Una vez tenemos la matriz de datos, tenemos que cada 
%
%El análisis de componentes principales, aunque normalmente se utiliza con el objetivo de representar los datos de manera sencilla, también podría llegar a ser útil en la reducción de la dimensionalidad para evitar el overfitting en el aprendizaje automático. 
%
%Este método es una técnica matemática que dado un vector aleatorio $\textbf{X}^T=[X_1,\ldots X_p]$, con vector de media $\mu$ y matriz de covarianzas $\Sigma$, utiliza transformaciones ortogonales para conseguir las componentes principales. Aunque también es realizable con la media poblacional $\overline{x}$ y la matriz de varianzas \textbf{S} de manera análoga.  
%
%Estas componentes principales son combinaciones lineales de las variables que forman el vector, de manera que estando correladas las iniciales, las componentes no lo están y se busca calcularlas con la máxima varianza posible. 
%
%Esta transformación se busca ya que si varias variables están altamente correladas, entonces están aportando la misma información, siempre que no sean dependientes. Podemos encontrar por esta razón variables que no estén correladas, lo que implica que no dan la misma información y que en consecuencia pueden aportar mayor información acerca de la variación de los datos, sin tener que ser compartida o repetida por varias variables. 
%
%Al finalizar este Análisis de Componentes Principales, obtendremos un conjunto de variables nuevas no correladas entre sí, que son combinación lineal de las iniciales que maximizan la varianza en cada paso.
%
%Añadir que si las variables no están correladas o están cerca de no estarlo, el Análisis de Componentes Principales no tiene sentido ya que el conjunto de componentes principales será parecido a las variables iniciales, con la única diferencia que estarán ordenadas por orden creciente de varianza.  



\subsection{Definición y cálculo de las Componentes}

\noindent Sea un vector aleatorio $\textbf{X}^T=[X_1,\ldots X_p]$ con vector de medias $\mu$ y matriz de covarianzas $\Sigma$.
\begin{defi}
Las componentes principales son combinaciones lineales de las variables $X_1 \ldots X_p$
\begin{equation}
\textbf{Y}_j=a_{1j}X_1+\ldots a_{pj}X_p=\textbf{a}_j^T\textbf{X}\quad 
\end{equation}

\noindent Donde $\textbf{a}_j$ es un vector de constantes y la variable $\textbf{Y}_j$ cumple lo siguiente:
\begin{itemize}
\item Si $j=1$ $Var(\textbf{Y}_1)$ es máxima restringido a $\textbf{a}_1^T \textbf{a}_1=1$
\item Si $j>1$ debe cumplir:
\begin{itemize}
\item $Cov(\textbf{Y}_j,\textbf{Y}_i)=0\quad \forall i<j $
\item $\textbf{a}_j^T \textbf{a}_j=1$
\item $Var(\textbf{Y}_j)$ es máxima. 
\end{itemize}

\end{itemize}

\noindent De esta manera, estamos buscando una nueva base que consiga reunir las direcciones de máxima variación. 
\end{defi}

\noindent El cálculo de la primera componente principal se lleva a cabo con un proceso de optimización de la función $Var(\textbf{Y}_1)$ sujeto a la restricción de que $\textbf{a}_1^T\textbf{a}_1=1$. 

\noindent Aplicando el método de los multiplicadores de Lagrange,  dada una función $f(\textbf{x})=f(x_1,\ldots, x_p)$ diferenciable con una restricción $g(\textbf{x})=g(x_1, \ldots, x_p)=c$ entonces existe una constante $\lambda$ de manera la ecuación:
\begin{equation}
\dfrac{\partial f}{\partial x_i}-\lambda\dfrac{\partial g}{\partial x_i}=0 \quad i=1,\ldots p 
\end{equation}

\noindent Tiene como solución los puntos estacionarios de $f(\textbf{x})$. Entonces, se puede definir la función $L(\textbf{x})= f(\textbf{x})-\lambda[g(\textbf{x})-c]$  que permite simplificar la expresión anterior a:
\begin{equation}
\dfrac{\partial L}{\partial \textbf{x}}=0
\end{equation}

\noindent Para el caso de las componentes principales, la función objetivo es la varianza de la combinación lineal, es decir, $f(\textbf{x})=\textbf{x}^T \Sigma \textbf{x}$ y la restricción aplicada es $g(\textbf{x})=\textbf{x}^T\textbf{x}=1$. 

\noindent Tomando $\textbf{x}=\textbf{a}_1$ se puede establecer $L(\textbf{a}_1)=\textbf{a}_1^T \Sigma \textbf{a}_1 - \lambda[\textbf{a}_1^T \textbf{a}_1-1]$. Que al derivarla se obtiene:
%\noindent Para el cálculo de la $L(\textbf{a}_1)=\textbf{a}_1^T \Sigma \textbf{a}_1 - \lambda[\textbf{a}_1^T \textbf{a}_1-1]$. Al derivarla obtenemos que :
\begin{align*}
\dfrac{\partial L}{\partial \textbf{a}_1} &= 2\Sigma \textbf{a}_1 - 2\lambda\textbf{a}_1\\
& = 2(\Sigma-\lambda)\textbf{a}_1 
\end{align*}

\noindent Igualando a 0 tenemos la siguiente ecuación: 
\begin{equation}
(\Sigma-\lambda I)\textbf{a}_1=0
\end{equation}

\noindent Para que $\textbf{a}_1$ sea un vector no trivial, tenemos que elegir $\lambda$ de tal manera que $|\Sigma-\lambda I| = 0$, es decir, $\lambda$ es un vector propio de la matriz de covarianzas, $\Sigma$. Al ser ésta una matriz semidefinido positiva y simétrica, los valores propios son reales y positivos. Por tanto, $\textbf{a}_1$ es un vector propio de la matriz de covarianza.

\noindent La función a maximizar es $Var(\textbf{Y}_1)=Var(\textbf{a}_1^T\textbf{X})=\textbf{a}_1^T\Sigma \textbf{a}_1=\textbf{a}_1^T\lambda \textbf{a}_1$, y para maximizarla basta tomar $\lambda=\max{\lbrace\lambda_1\ldots \lambda_p\rbrace}$. reordenando si es necesario, se tiene que $\lambda=\lambda_1$ 

%\noindent Para que la ecuación tenga una solución que no sea la trivial, tenemos que elegir $\lambda$ de manera que $|\Sigma-\lambda I| = 0$. Luego $\lambda$ es uno de los valores propios de la matriz de covarianzas. Generalmente una matriz $(p\times p)$ tiene $p$ valores propios $\lbrace\lambda_1, \ldots ,\lambda_p \rbrace$ y como $Var(\textbf{Y}_1)=Var(\textbf{a}_1^T\textbf{X})= \textbf{a}_1^T \Sigma \textbf{a}_1 =\textbf{a}_1^T \lambda \textbf{a}_1=\lambda$ que es la variable a maximizar, elegimos $\lambda=\max \lbrace\lambda_1, \ldots ,\lambda_p \rbrace$, por tanto, el vector $\textbf{a}_1$ es el vector propio con valor propio $\lambda=\lambda_1$ reordenando si es necesario.



\noindent Una vez calculada la primera componente principal $\textbf{Y}_1$, la segunda componente se calcula de manera análoga, maximizando $Var(\textbf{Y}_2)=Var(\textbf{a}_2^T\textbf{X})$ condicionada por $\textbf{a}_2^T\textbf{a}_2=1$. A esta restricción tenemos que añadir la restricción $Cov(\textbf{Y}_1,\textbf{Y}_2)=0 $

\begin{propo}
La condición $Cov(\textbf{Y}_1,\textbf{Y}_2)=0 $ equivale a la condición $\textbf{a}_2^T\textbf{a}_1 = 0$.
\begin{proof}
Utilizando que $\textbf{Y}_j=\textbf{a}_j^T \textbf{X}\quad \forall j$ , tenemos entonces que:
\begin{align*}
Cov(\textbf{Y}_2,\textbf{Y}_1)&= Cov (\textbf{a}_2^T\textbf{X},\textbf{a}_1^T\textbf{X})\\ 
&= \mathbb{E}(\textbf{a}_2^T(\textbf{X}-\mu)(\textbf{X}-\mu)^T \textbf{a}_1)\\
&= \textbf{a}_2^T \mathbb{E}((\textbf{X}-\mu)(\textbf{X}-\mu)^T) \textbf{a}_1\\
&= \textbf{a}_2^T \Sigma \textbf{a}_1 \\
&= \textbf{a}_2^T \lambda_1 \textbf{a}_1
\end{align*}
\noindent De manera que, si $a_2^T \lambda_1 a_1 = 0 \Rightarrow a_2^T a_1=0 $, luego son vectores ortogonales entre sí.
\end{proof}
\end{propo}


\noindent \emph{Observación: } Esta proposición se puede extender de manera simple al caso de tener que calcular la $i$-ésima componente principal habiendo calculado las anteriores de las cuales se sepan los valores propios asociados. 

\begin{coro}
Las componentes principales son todas ortogonales entre sí. 
\end{coro}

\noindent Tomando la matriz formada por los $p$ vectores propios como columnas tenemos la matriz ortogonal $\textbf{A}=[\textbf{a}_1,\ldots, \textbf{a}_p]$, de manera que el vector aleatorio 
$$\textbf{Y}=[Y_1,\ldots , Y_p]^T=\textbf{A}\textbf{X}$$

\noindent Se deduce utilizando la ortogonalidad de $\textbf{A}\Rightarrow \textbf{A}^T\textbf{A}=\textbf{A}\textbf{A}^T=\textbf{I}$ que:
\begin{align*}
Var(\textbf{Y})&=Var(\textbf{AX})\\
&=\mathbb{E}((\textbf{AX})^T(\textbf{AX}))\\
&=\mathbb{E}(\textbf{X}^T\textbf{A}^T\textbf{A}\textbf{X})\\
&=\mathbb{E}(\textbf{X}^T\textbf{X})\\
&=Var(\textbf{X})
\end{align*} 

\noindent Por tanto, se puede afirmar que las componentes principales contienen toda la variación del vector \textbf{X} aleatorio inicial.

\noindent Sea ahora la matriz \textbf{X} de datos de tamaño $n \times p$ resultante de tomar $n$ observaciones de las $p$ variables del vector aleatorio. Se supone además que esta es centrada y dividida todos sus elementos por $\sqrt{n-1}$. 

\noindent Dado este marco la matriz \textbf{X} tiene una descomposición en valores singulares $\textbf{X}=\textbf{U}\Sigma \textbf{V}^T$. Donde tanto \textbf{U} como $\textbf{V}$ son matrices ortogonales. Esto implica que $\textbf{U}^T \textbf{U}=\textbf{U}\textbf{U}^T=\textbf{I}$. 

\noindent  En particular, \textbf{U} es la matriz de vectores singulares por la izquierda  de tamaño $n \times r$, \textbf{V} la matriz de vectores singulares por la derecha de tamaño $p \times p$ y la matriz $\Sigma$ es la matriz diagonal de valores singulares generalmente ordenados de forma decreciente, es decir $\Sigma^2$ es la matriz diagonal que contiene todos los valores propios de las matrices $\textbf{X}^T\textbf{X}$ y $\textbf{X}\textbf{X}^T$.

\noindent Por las transformaciones previas realizadas la matriz de estimaciones de las covarianzas es $\hat{S}=\textbf{X}^T\textbf{X}$ en consecuencia, el proceso que se describía en primera instancia para un vector aleatorio en el que conocíamos su matriz de covarianzas se puede aplicar de manera análoga a la matriz de datos.

\noindent Esto implica que la obtención de las componentes principales en el caso de ser calculadas desde la matriz de datos también se obtienen calculando los vectores y valores propios de la matriz de covarianzas muestral en este caso. 

\begin{propo}
La matriz \textbf{V} de tamaño $(p\times p)$ es la matriz que contiene los vectores para hacer la combinación lineal que definen las componentes principales.
\begin{proof}
La matriz $\textbf{X}^T \textbf{X}$ es la matriz de covarianzas $(p \times p)$  por la descomposición en valores singulares tenemos que:
\begin{align*}
\textbf{X}^T \textbf{X} &= (\textbf{U}\Sigma \textbf{V}^T)^T (\textbf{U}\Sigma \textbf{V}^T)\\
&= \textbf{V}\Sigma ^T \textbf{U}^T \textbf{U}\Sigma \textbf{V}^T\\
&= \textbf{V}\Sigma ^T \Sigma \textbf{V}^T \\
&=  \textbf{V}\Sigma ^2  \textbf{V}^T 
\end{align*}

\noindent Que es la diagonalización de la matriz $\textbf{X}^T \textbf{X}$ donde los elementos de la diagonal de $\Sigma$ son las raíces cuadradas de los valores propios de $\textbf{X}^T \textbf{X}$. A esto también hay que añadir que \textbf{V} es la matriz que tiene como columnas los vectores propios, es decir, es la matriz con los vectores que forman las componentes principales. 
\end{proof}
\end{propo}

\subsection{Reducción de la dimensionalidad}

\noindent La utilidad de esta técnica es reducir el tamaño de la matriz de datos para poder representar las observaciones como puntos de una subvariedad de dimensión $m < p$. Con este fin en mente, se deben definir los siguientes conceptos.
\begin{defi}
Sea $\textbf{A}\in \mathbb{M}_{n\times m}(\mathbb{R})$ definimos la \textit{norma de Frobenius} de la matriz \textbf{A} como :
\begin{equation}
||\textbf{A}||_F=(tr(\textbf{A}^T\cdot \textbf{A}))^{\frac{1}{2}}=\left(\sum_{i=1}^{n}\sum _{j=1}^{m}a_{ij}^2\right)^{\frac{1}{2}}
\end{equation}
\end{defi}

\begin{propo}
La norma de Frobenius es invariante a transformaciones ortogonales
\begin{proof}
Sea $\mathbf{U}$ una matriz ortogonal, que cumple $\mathbf{U}^T\cdot \mathbf{U}=\mathbf{U}\cdot \mathbf{U}^T=I$, sea una matriz cualquiera $\mathbf{A}$, entonces:
\begin{align*}\tag{2.7}
||\mathbf{U} \cdot \mathbf{A}||_F^2&=tr((\mathbf{U} \mathbf{A})^T\cdot(\mathbf{U} \mathbf{A}))\\
&=tr((\mathbf{A}^T \mathbf{U}^T)\cdot \mathbf{U} \mathbf{A}))\\
&=tr(\mathbf{A}^T \mathbf{A})\\
&=||\mathbf{A}||_F^2
\end{align*}
\qedhere
\end{proof}
\end{propo}

\noindent Aunque hay más normas que son invariantes ante transformaciones ortogonales, nos centraremos en la norma de Frobenius.

\begin{defi}
Dada una matriz $\textbf{A}=\textbf{U}\Sigma \textbf{V}^T$ de tamaño $n\times m $ y de rango $r$, entonces para $l < r$ se define la matriz truncada  $\textbf{A}_l=\textbf{U}_l \Sigma_l \textbf{V}^T_l$ donde: 
\begin{itemize}
\item $\textbf{U}:=[\textbf{U}_l, U_{r-l}]$
\item 
\item
\end{itemize}
\end{defi}

\begin{teorema}[De Eckart-Young]
Sea una matriz $\textbf{A}$ de rango $r$, y sea una matriz $\textbf{B}$ rango $l<r$, entonces:
\begin{equation}
||\textbf{A}-\textbf{B}||_F \leq ||\textbf{A}-\textbf{A}_l||_F
\end{equation}
\end{teorema}
\begin{propo}
Dada una matriz $\textbf{A}=$
\end{propo}




