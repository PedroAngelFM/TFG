\section{Análisis de Clusters}
\noindent Según \cite{Jain 1988} el análisis de clusters tiene como objetivo conseguir una clasificación de las observaciones en subconjuntos que tengan sentido de acuerdo al contexto de la investigación. El clustering es un tipo de clasificación de los datos que tiene las siguientes características:
\begin{itemize}
\item \textit{Exclusividad: }Una observación no puede pertenecer a más de un cluster a la vez. 
\item \textit{Es intrínseco: }No hay ninguna etiqueta que permita clasificar las observaciones, son las características de las propias muestras las que servirán como diferenciadores. Lo que hace que sea un método \textit{no supervisado}.
\end{itemize}
Además de estas características el clustering puede ser jerárquico \textit{(concepto que se desarrollará más adelante)} o particional. \\
\noindent Por otro lado, los algoritmos se pueden clasificar según  sean \textit{aglomerativos}, en los que se empieza teniendo cada una de las observaciones y se van formando los clusters hasta tener un solo cluster con todas las muestras. Por el contrario, los \textit{algoritmos   divisivos}, empiezas con un solo cluster y se van haciendo subconjuntos del mismo.  

\noindent Otra clasificación puede venir dada por el modo en el que se fijan en las variables, si es \textit{Monotético}, en cada paso se centra en una variable, mientras que si es \textit{Politético} se utilizan todas las variables a la vez.\\



\noindent Con el objetivo de formalizar lo anterior, de ahora en adelante, sea el caso en el que se han tomado $n$ observaciones $\omega_1,\ldots, \omega_n$, se denota como $\Omega=\lbrace\omega_1,\ldots, \omega_n\rbrace=\lbrace 1,\ldots, n\rbrace$ para abreviar. 

\noindent Crear una clasificación entre las observaciones de $\Omega$ es establecer una relación de equivalencia $\mathcal{R}$ sobre $\Omega$. De esta manera, podemos establecer que:
\begin{equation}
\Omega=\bigsqcup_{i=1}^m c_i
\end{equation}
\noindent Donde $c_i$ son las clases de equivalencia de la relación $\mathcal{R}$.

\begin{defi}
Se llama \textit{clustering} a la partición que provoca la relación $\mathcal{R}$ y se definen los \textit{clusters} como las clases de equivalencia de dicha relación, $\lbrace c_i\rbrace$. 
\end{defi}
\newpage
\subsection{Estructuración jerárquica de $\Omega$}\label{Estructuración jerárquica}

\noindent En un proceso de clustering jerárquico se crea una sucesión de particiones sobre el mismo conjunto donde los sucesivos clusterings se obtienen agrupando clusters.  

\noindent Para poder formalizar el clustering jerárquico debemos definir el concepto de \textit{jerarquía indexada}

\begin{defi}
Una \textit{jerarquía indexada} $(C,\alpha)$ sobre un conjunto $\Omega$, es una colección de clusters $C\subset \mathcal{P}(\Omega)$ y un índice que cumplen:
\begin{itemize}
\item \textit{Intersección}. Si $c,c'\in C$ entonces $c\cap c'\in\lbrace c, c', \emptyset\rbrace$. Es decir, dos clusters o están contenido el uno en el otro o son disjuntos. 
\item \textit{Reunión} Cada cluster se puede caracterizar como la unión de los clusters que contiene. Es decir, $c\in C \Rightarrow c=\cup \lbrace c'/c'\in C, \quad c'\subset c\rbrace$
\item El conjunto de todos los clusters es el conjunto total.
\end{itemize}

\noindent El índice es una aplicación $\alpha:C \rightarrow \mathbb{R}$ que cumple lo siguiente :
\begin{equation}
\alpha(i)=0, \forall i\in\Omega \quad\quad \alpha(c)\leq \alpha(c') \text{ si } c\subset c'
\end{equation}
De esta manera, el índice puede medir como de heterogéneos son los clusters, es decir, puede otorgar una medida de lo similares o no que son las observaciones dentro de un mismo cluster. 
\end{defi}

\begin{propo}
Para todo $x\geq 0 $ la relación $\mathcal{R}_x$ sobre $\Omega$:
\begin{equation}
i\mathcal{R}_x j \quad \text{ si } \quad i,j\in c \quad \text{ siendo }\quad\alpha(c)\leq x
\end{equation}
Es una relación de equivalencia.
\end{propo}
\noindent Es decir, con esta relación se establece que dos elementos son ``iguales'' cuando hay un cluster de un nivel de heterogeneidad menor que un umbral que los contiene. Es decir, se fija un criterio por el cual dos observaciones son lo suficientemente similares.

\begin{defi}
Un espacio ultramétrico es una pareja $(\Omega,u)$ donde $\Omega$ es un conjunto finito y $u$ una función distancia sobre $\Omega\times \Omega $ que verifica para cada elemento de $i,j,k\in \Omega$
\begin{itemize}
\item $u(i,j)\geq u(i,i)=0$
\item $u(i,j)=u(j,i)$
\item \textit{Propiedad ultramétrica: } $u(i,j)\leq \sup \lbrace u(i,k),u(j,k)\rbrace $
\end{itemize}
\end{defi}
\noindent Se puede ver que toda distancia ultramétrica cumple la desigualdad triangular. Por tanto, todo espacio ultramétrico es métrico. Cómo añadido, juntando los elementos próximos de $\Omega$ se mantiene la propiedad ultramétrica. 

\begin{teorema}
Supongamos un clustering con $m$ clusters de $\Omega =\bigsqcup_{i=1}^m c_i$ sobre el que se tiene una distancia ultramétrica $u$.\\
Tomando $c_i,c_j$ los dos clusters más cercanos y uniéndolos podemos definir una nueva distancia $u'$ sobre los $m-1$ clusters nuevos. 
\begin{proof}
Sea $k\neq i,j$ por la propiedad ultramétrica se tiene que: 
\begin{equation}
\begin{split}
u(i,k)&\leq sup\lbrace u(i,j),u(j,k) \rbrace = u(j,k) \\
u(j,k)&\leq sup\lbrace u(i,j),u(i,k) \rbrace = u(i,k)
\end{split}
\end{equation}
En consecuencia, $u(i,k)=u(j,k)$ y por tanto, $u(c_k,c_i)=u(c_k,c_j)$.\\
Definiendo de la distancia ultramétrica $u'$:
\begin{equation}
\begin{split}
u'(c_k,c_i\cup c_j) &= u(c_k,c_i) = u(c_k,c_j) \quad k\neq i,j\\
u'(c_a,c_b) &= u(c_a,c_b) \quad a,b\neq i,j
\end{split}
\end{equation}
Con esta nueva distancia se cumple la propiedad ultramétrica. Para elementos $c_a,c_b$ con $a,b \neq i,j $ es evidente, pues no hemos cambiado la definición de la distancia para esos clusters. Ahora sea el siguiente caso:
\begin{equation}
\begin{split}
u'(c_a,c_i\cup c_j)=u(c_a,c_i)&\leq \sup\lbrace u(c_a,c_b), u(c_b,c_i)\rbrace =\\
&= \sup\lbrace u'(c_a,c_b), u'(c_b,c_i\cup c_j)\rbrace
\end{split}
\end{equation}
Por tanto, haciendo uniones entre los elementos más cercanos no modifica la estructura de espacio ultramétrico. \\
\qedhere
\end{proof}
\end{teorema}

\noindent Dado estos resultados se puede seguir el siguiente procedimiento para obtener una jerarquía indexada.   

\noindent \textbf{Algoritmo fundamental de clasificación}

\noindent Dado un espacio ultramétrico $(\Omega, u)$ utilizando el teorema anterior se pueden ir juntando los clusters más próximos conservamos la distancia ultramétrica. \\
\begin{enumerate}
\hrule
\item Se empieza con el clustering $\Omega= \lbrace 1\rbrace \sqcup \ldots \sqcup\lbrace n \rbrace$. 
\item Calculamos todas las distancias $u(c_a,c_b)$ y unimos los clusters que más cercanos estén utilizando la misma distancia ultramétrica como en el teorema anterior.
\item Se considera la nueva partición
\begin{equation}
\Omega= \lbrace 1\rbrace \sqcup\ldots \sqcup \lbrace i,j\rbrace \sqcup \ldots \sqcup\lbrace n \rbrace
\end{equation}
Cada vez que se une $c_i$ con $c_j$ se define el índice $\alpha(c_i\cup c_j)=u(c_i,c_j)$.\\
\hrule
\end{enumerate}


\noindent Este procedimiento nos brinda una forma sencilla de dada una distancia ultramétrica construir una jerarquía indexada. 

\noindent En el caso contrario en el que se disponga de una jerarquía indexada, se puede crear un espacio ultramétrico con la siguiente proposición. 
\begin{propo}
Dado un conjunto $\Omega$ finito sobre el que tenemos una jerarquía indexada $(C,\alpha)$ podemos establecer sobre $\Omega$ una estructura de espacio ultramétrico. 
\begin{proof}
Definiendo la distancia
\begin{equation}
u(i,j)=\alpha(c_{ij})
\end{equation}
Donde $c_{ij}$ es el mínimo cluster que contiene a ambos $i,j$. Sean ahora los clusters mínimos que contienen a $\lbrace i,k\rbrace,\lbrace j,k\rbrace, c_{ik},c_{jk}$ respectivamente, entonces:
\begin{equation}
c_{ik}\cap c_{jk}\neq \emptyset
\end{equation}
Por ser una jerarquía indexada tenemos dos posibilidades de inclusión uno en el otro:
\begin{equation}
\begin{cases}
c_{ik}\subset c_{jk}\Rightarrow i,j,k\in c_{jk}\Rightarrow c_{ij}\subset c_ {jk}\Rightarrow u(i,j)=\alpha(c_{ij})\leq u(j,k)=\alpha(c_{jk})\\
c_{jk}\subset c_{ik}\Rightarrow i,j,k\in c_{ik}\Rightarrow c_{ij}\subset c_ {ik}\Rightarrow u(i,j)=\alpha(c_{ij})\leq u(i,k)=\alpha(c_{ik})
\end{cases}
\end{equation}
Por tanto, se obtiene que $u(i,j)\leq \sup\lbrace u(i,k),u(j,k) \rbrace$\\
\qedhere
\end{proof}
\end{propo}

\subsection{Algoritmos basados en jerarquías}
\noindent Sea ahora una matriz de datos $\textbf{X}$ resultante de hacer $n$ observaciones sobre $p$ variables aleatorias. Sea una distancia $\delta$ \textit{(Se detallará más adelante como elegirla en función del tipo de datos que tengamos.)} sobre el espacio de observaciones y $\Delta$ la matriz $\Delta=(\delta(i,j))$ de tamaño $n \times n$, la matriz que contiene todas las distancias entre observaciones. 

\noindent En el caso de que $\delta$ sea ultramétrica se puede utilizar el algoritmo fundamental, en el caso contrario hay que realizar una modificación. \\

\noindent \textbf{Algoritmo de clasificación de observaciones}

\noindent Lo que se busca en este tipo de algoritmos es convertir $\delta$ en una distancia últramétrica, por ende dado el espacio métrico $(\Omega, \delta)$ el algoritmo es el siguiente:\\
\begin{enumerate}
\hrule
\item Se empieza con el clustering $\Omega= \lbrace 1\rbrace \sqcup \ldots \sqcup\lbrace n \rbrace$. 
\item Calculamos todas las distancias $u(c_a,c_b)$ y unimos los clusters que más cercanos estén y se define la distancia de un elemento $k$ al cluster de la siguiente manera:
\begin{equation}
\delta(k,\lbrace i,j \rbrace)=f(\delta(i,k),\delta(j,k))
\end{equation}
Donde la función $f$ hace que $\delta$ cumpla la propiedad ultramétrica

\item Se considera la nueva partición
\begin{equation}
\Omega= \lbrace 1\rbrace \sqcup\ldots \sqcup \lbrace i,j\rbrace \sqcup \ldots \sqcup\lbrace n \rbrace
\end{equation}
Cada vez que se une $c_i$ con $c_j$ se define el índice $\alpha(c_i\cup c_j)=\delta'(c_i,c_j)$.\\
\hrule
\end{enumerate}
De esta manera se obtiene también una jerarquía indexada $(C,\alpha)$

\noindent Dependiendo de la función $f$ que se elija, se tienen distintos algoritmos:
\begin{itemize}
\item Si $f(x,y)=min(x,y)$ se obtiene el algoritmo de \textit{single linkage}.
\item  Si $f(x,y)=min(x,y)$ se obtiene el algoritmo de \textit{complete linkage}.
\item  Si $f(x,y)=\dfrac{x+y}{2}$ se obtiene el algoritmo de \textit{average linkage}.
\end{itemize}

\noindent Es decir, tomando distintas $f$ obtenemos distintos algoritmos. En estos casos, los algoritmos todos son algoritmos aglomerativos. 

\newpage
\subsection{Elección de las diferencias }
\noindent Añadir a esto que el mayor problema que conllevan este tipo de algoritmos es elegir y definir una distancia. Sea un vector aleatorio \textbf{x} de longitud $p$ sobre el que se realizan $n$ observaciones, entonces la forma más habitual de definir la distancia entre dos observaciones es:
\begin{equation}
D(x_i,x_j)=\sum_{k=1}^p \omega_k \cdot d_k(x_{ik},x_{jk}); \quad \sum_{k=1}^p \omega_k=1
\end{equation}

\noindent Donde las $d_k$ son las distancias adaptadas al tipo de variable que sea la variable aleatoria $X_k$. En particular:

\noindent \textbf{\textit{Variables Cuantitativas}: }Basta con que sea una función monótona creciente del valor absoluto de la diferencia entre las dos observaciones. De manera general, se utilizaría la distancia euclídea, pero esta tiene el problema de que no tiene en cuenta las escalas de los datos es por eso que tenemos que estandarizar y centrar los datos para que no haya problemas de escalas. Una medida de distancia que suple esto es la distancia de Mahalanobis.
\begin{defi}
Dadas dos muestras u observaciones $\textbf{x}_i,\textbf{x}_j$ sacadas de la misma población de un total de $n$ de las cuales se observan $p$ variables aleatorias, con matriz de covarianzas muestrales $\textbf{S}$. Se define la distancia de Mahalanobis de la siguiente manera: 
\begin{equation}
\textbf{M}^2(\textbf{x}_i,\textbf{x}_j)=(\textbf{x}_i,\textbf{x}_j)^T \textbf{S}^{-1} (\textbf{x}_i,\textbf{x}_j)
\end{equation}
\end{defi}
\noindent \textbf{\textit{Variables Ordinales: }} Para una variable categórica cuyos $M$ valores tienen un orden se puede hacer la siguiente transformación para tratarlas como una variable cuantitativa:
\begin{equation}
\dfrac{i-\frac{1}{2}}{M} \quad i=1\ldots M
\end{equation}
\noindent \textbf{\textit{Variables Categóricas: }}En este caso, suponiendo que la variable tiene $M$ posibles valores, lo habitual es definir una matriz $\textbf{L}$ de tamaño $M\times M$ simétrica , de manera que el elemento $L_{r,r'}=L_{r',r}$ es la distancia entre la categoría $r$ y la $r'$ que habitualmente se suele tomar como 1 y la diagonal igual a 0. 

\noindent De esta manera, podemos definir de manera general una distancia entre las observaciones. 

\subsection{Algoritmo de K-medias y K-medioides}

\noindent Definida la distancia entre observaciones podemos definir el algoritmo de K-medias. 
\paragraph*{Algoritmo de K-medias}



















