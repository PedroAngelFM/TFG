\newpage
\section{Análisis Discriminante}
\noindent Sea un vector aleatorio \textbf{x} de longitud $p$ del que se extraen observaciones, las cuales pertenecen a 2 poblaciones $\Omega_1,\Omega_2$. Se quiere establecer un criterio para una vez se extrae un nuevo individuo u observación, saber a cual de las poblaciones pertenece.\\


\begin{defi}
Una \textit{regla discriminante} es un criterio de clasificación de observaciones, a menudo se da mediante una función discriminante de los valores de la observación $D(x_1 \ldots x_p)$. 
\end{defi}

\noindent \emph{Observación} Lo más habitual es tener una regla de este tipo:
\begin{equation}
\begin{split}
\text{Si } D(x_1 \ldots x_p)\geq 0 &\Rightarrow \omega\in \Omega_1\\ 
\text{En el caso contrario } &\Rightarrow \omega\in \Omega_2
\end{split}
\end{equation}

\noindent Esta definición divide el espacio $\mathbb{R}^p$ en dos regiones. Antes de continuar se debe definir el siguiente concepto:
\begin{defi}
Llamaremos probabilidad de clasificación errónea a la probabilidad de clasificar una muestra $\omega$ que pertenece en realidad a $\Omega_1$ como $\Omega_2$ y viceversa:
\begin{equation}
pce=\mathbb{P}(R_2|\Omega_1)\mathbb{P}(\Omega_1)+\mathbb{P}(R_1|\Omega_2)\mathbb{P}(\Omega_2)
\end{equation}
\end{defi}

\noindent Estos conceptos son generalizables a $k$ poblaciones $\Omega_1 \ldots \Omega_k$. 
Se detallan a continuación varias formas de crear una regla discriminante: 
\subsection{Reglas Discriminantes Básicas}
\noindent A continuación se detallará el caso para $2$ poblaciones $\Omega_1, \Omega_2$ y se desarrollará a partir de ahí el caso para $k$ poblaciones dependiendo de la facilidad de desarrollo a partir del primero. 
\subsection*{Discriminador lineal}
\noindent Sean dos poblaciones $\Omega_1, \Omega_2 $ con medias poblacionales $\mu_1,\mu_2$ y matriz de covarianzas $\Sigma$ entonces definimos la siguiente distancia:
\begin{defi}
Llamaremos \textit{distancia de Mahalanobis} de una observación\\$\textbf{x}=(x_1\ldots x_p)^T$ a una de las dos poblaciones  a la siguiente aplicación:
\begin{equation}
 M^2(\textbf{x},\mu_i)=(\textbf{x}-\mu_i)^T\Sigma^{-1}(\textbf{x}-\mu_i)\quad i=1,2
\end{equation} 
\end{defi} 

\noindent Una \textit{regla discriminante} teniendo en cuenta las distancias es la siguiente:
\begin{equation}
\begin{split}
\text{Si } M^2(\textbf{x},\mu_1)< M^2(\textbf{x},\mu_2) &\Rightarrow \omega\in \Omega_1\\ 
\text{En el caso contrario } &\Rightarrow \omega\in \Omega_2
\end{split}
\end{equation}

\noindent Sean $k$ poblaciones $\Omega_1 \ldots \Omega_k$.