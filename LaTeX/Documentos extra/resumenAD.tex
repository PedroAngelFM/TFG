\section{Análisis Discriminante}
\subsection{Objetivos}

\noindent Dadas dos poblaciones $\Omega_1,\Omega_2$ de las cuales observamos las variables $X_1, \ldots, X_p$, buscamos encontrar un \textbf{discriminador}, una regla que nos diga como poder diferenciar muestras que pertenezcan a una u otra población. 

\noindent En el caso se detalla a continuación se consideran dos poblaciones y tomando las observaciones $\textbf{x}=(x_1,\ldots,x_p)\in \mathbb{R}^p $. Tras aplicar el análisis discriminante, obtenemos una división del espacio $\mathbb{R}^p$ en 2 regiones, $R_1, R_2$. 

\subsection{Conceptos previos}
\begin{defi}
Llamaremos \textit{distancia de Mahalanobis} de dos observaciones $\textbf{x}_i,\textbf{x}_j$ cuyas variables observadas tienen matriz de covarianzas \textbf{S} a:
\begin{equation}
d_M(\textbf{x}_i,\textbf{x}_j)=\sqrt{(\textbf{x}_i-\textbf{x}_j)^T\textbf{ S}^{-1}(\textbf{x}_i-\textbf{x}_j) }
\end{equation}
\end{defi}

\begin{defi}
Llamaremos \textit{probabilidad de clasificación errónea, (pce)} a la probabilidad de que dado un nuevo individuo o muestra esté en la región $R_1$ siendo en realidad de la población $\Omega_2$ y viceversa. 
\begin{equation}
pce=\mathbb{P}(R_1|\Omega_2) \mathbb{P}(\Omega_1)+\mathbb{P}(R_2|\Omega_1) \mathbb{P}(\Omega_2)
\end{equation}
\end{defi}

\noindent El objetivo es encontrar una función $f$ de tal manera que al tomar una nueva observación $\omega$ :
\begin{align*}
f(\omega)<0 \quad &\text{asignamos } \omega \text{ a } R_1\\
f(\omega)\geq 0 \quad &\text{asignamos } \omega \text{ a } R_2\\
\end{align*}

\noindent La forma más intuitiva de separar ambas regiones, es hacer la diferencia de las distancias entre el nuevo individuo y las medias de cada región. 
\begin{align*}
M^2(\textbf{x}-\mu_2)-M^2(\textbf{x}-\mu_1)  &= \textbf{x}^T\textbf{S}^{-1} \textbf{x}+()
\end{align*}


\noindent