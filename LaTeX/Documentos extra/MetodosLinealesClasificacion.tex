\newpage
\section{Clasificación}

\noindent Hasta ahora se ha considerado el caso en el que la variable respuesta, es una variable continua en la cual se puede definir una métrica usual \cite{Hair 1995}, por ejemplo, en el caso de valores respuesta continuos y reales se puede definir fácilmente métricas como puede ser el error cuadrático. 

\noindent En los casos de la clasificación o discriminación, \emph{(Su diferencia se discutirá más adelante)} se tienen variables respuestas que no tienen un orden en específico. Por tanto, el modo de tratar los datos cambia de manera significativa en un primer momento.

\noindent En este caso, la clasificación y la discriminación se diferencian, según \emph{Johnson, R.A, y Wichern D.W.} \cite{Johnson 2007}, en que la discriminación busca describir, ya sea gráficamente \emph{(Análisis canónico de poblaciones)} o algebraicamente \emph{(Análisis discriminante lineal, logísitico o cuadrático)} las características de las distintas poblaciones. En el otro lado, la clasificación busca asociar nuevas observaciones a las distintas clases o poblaciones. Aún así, las distintas  técnicas descritas se pueden usar para ambas tareas, ya que al describir las características que posee una cierta población se pueden realizar nuevas asignaciones.  

\noindent Las distintas técnicas de estudio de la dependencia entre variables continuas y no continuas asumen hipótesis y buscan objetivos distintos \cite{Villardón 2006}:
\begin{itemize}
\item Cuando no se conozca las distribuciones de probabilidad de las distintas poblaciones, se puede analizar con fines exploratorios el \emph{Análisis canónico de poblaciones} \cite{Villardón 2006}, en el que se estudian las direcciones de mayor distancia entre las dos poblaciones, en particular, también se le llama \emph{Análisis discriminante lineal de Fisher}\cite{Lebart 1984}.

\item En el caso de que se conozcan los parámetros de las distribuciones se pueden crear funciones discriminantes que permitan clasificar nuevas observaciones $\mathbf{x}_0$ en una de las poblaciones. A esto se le llama \emph{clasificación}, uno de los métodos más habituales es la creación de funciones discriminantes por máxima verosimilitud o siguiendo un criterio geométrico\cite{Peña 2002}. Es en este caso en el que se difuminan los límites entre discriminación y clasificación, en particular, se hace un trabajo de discriminación para luego realizar nuevas clasificaciones.
\end{itemize}

\noindent Para simplificar los cálculos y los desarrollos, en el caso de la primera parte en la que se desarrollan las funciones discriminante, se consideran dos poblaciones en las que en un inicio únicamente conocemos la distribución de probabilidad\cite{Johnson 2007} y luego se asumirá que son normales para construir el discriminante lineal habitual en la que aparece la distancia de Mahalanobis \cite{Peña 2002} aunque \emph{Morrison, D.F}\cite{Morrison 1976} construye dicha función discriminante buscando una combinación lineal que maximice el estadístico $T^2$ el cual mide la diferencia de las medias de las dos poblaciones. Añadir que para esto último, se utiliza la matriz de covarianzas muestral, mientras que el desarrollo dado durante la memoria se hará suponiendo conocida la matriz de covarianzas poblacional.  

\noindent En la segunda parte de la sección se detallará el método del análisis canónico de poblaciones o \emph{análisis discriminante de Fisher}\cite{Villardón 2006} que toma la matriz de covarianzas dentro y fuera de las poblaciones permitiendo tanto hacer predicciones en subespacios de dimensión menor. 



\subsection{Funciones discriminantes}
\noindent Supongase que se tienen dos valores de la variable respuesta $y$, entonces, se tienen dos poblaciones,$\pi_1, \pi_2$ de las cuales se conocen sus funciones de densidad de cada una de ellas, $f_1,f_2$ respectivamente, entonces si además se conoce la probabilidad de clasificación errónea $P(1|2),P(2|1)$. Se tiene que:
\begin{equation}
P(i|\textbf{x}_0)=\dfrac{P(\textbf{x}_0|i)P(i)}{P(1)P(\textbf{x}_0|1)+P(2)P(\textbf{x}_0|2)}=\dfrac{f_i(\mathbf{x}_0)P(i)}{P(1)f_1(\mathbf{x}_0)+P(2)f_2(\mathbf{x}_0)}
\end{equation}

\noindent Donde $P(1), P(2)$ son las probabilidades de pertenecer a cada una de las poblaciones. Por tanto, se puede clasificar como una población u otra ya que se conoce la distribución de cada una de las poblaciones. Primero definamos lo que es una función discriminante\cite{Cuadras 2014}:

\begin{defi}
Se llama \textit{función discriminante }, $f_{d}$ aquella que:
\begin{equation}
f_{d}:\mathbf{\Omega}\longrightarrow \mathbb{R}
\end{equation}
Donde $\Omega$ es el espacio de observaciones posibles. Definida de tal manera que si $f_{d}(\textbf{x}_0)>0\Rightarrow \textbf{x}_0\in \pi_i$ y en caso contrario $\textbf{x}_0\notin P_i$. 
\end{defi}
\noindent En el caso anterior de dos poblaciones, $f_1(\textbf{x}_0)P(1)-f_2(\textbf{x}_0)P(2)$ sería la función discriminante para discernir si $\textbf{x}_0$ pertenece a $P_1$ o no. Ya que representa la probabilidad de ser de dicha clase conociendo el valor de las variables predictoras.

\begin{defi}
Se llama \emph{Error de clasificación} al coste de clasificar de manera errónea una observación y se denota como $c(i|j)=$\emph{``Error de clasificar como $\pi_i$ una observación perteneciente a $\pi_j$"}.
\end{defi}

\noindent En general, la función de pérdida de un problema de clasificación se da en forma de matriz de pérdida $\mathbf{L}$ de tamaño $L\times L$ con la diagonal nula, que puede ser simétrica o no y donde $L_{ij}=c(i|j)$ \cite{Hastie 2001}. 

\noindent Se puede ponderar con respecto del coste de cada error de manera que una clasificación errónea con un coste mayor no se pueda dar, es decir, es penalizar a la que mayor coste tenga 
\begin{equation}
\dfrac{f_1(\textbf{x}_0)P(1)}{c(1|2)}-\dfrac{f_2(\textbf{x}_0)P(2)}{c(2|1)}
\end{equation}

\noindent Por consiguiente, a igualdad de los otros términos escogeremos el que menos coste, el que mayor verosimilitud  o el que mayor probabilidad a priori tenga \cite{Peña 2002}.

\noindent \emph{Johnson R.A. y Wichern D.W.}\cite{Johnson 2007} desarrollan otro enfoque, tomando el coste medio esperado \emph{ECM}:
\begin{equation}
ECM=c(2|1)P(2|1)P(1)+c(1|2)P(1|2)P(2)
\end{equation}

\noindent Además afirman que las regiones que minimizan el coste de clasificación son las que vienen dadas por la función discriminante anterior. 

\noindent Supóngase ahora que tanto $f_1,f_2$ son densidades normales con medias distintas $\mu_1,\mu_2$ pero con una matriz de covarianzas común $\mathbf{\Sigma}$, Entonces tenemos la siguiente expresión \cite{Johnson 2007, Cuadras 2014}:
\begin{equation}
f_i(\textbf{x})=\dfrac{1}{(2\pi)^{\frac{p}{2}}|\mathbf{\Sigma}|^{\frac{1}{2}}} exp \left\lbrace\dfrac{-1}{2}(\textbf{x}-\mu_i)^T \mathbf{\Sigma}^{-1}(\textbf{x}-\mu_i) \right\rbrace
\end{equation}

\noindent De esta manera, la función discriminante se puede transformar sustituyendo las densidades por la expresión anterior y tomando logaritmos : 
\begin{equation}
(\textbf{x}-\mu_1)^T \mathbf{\Sigma}^{-1}(\textbf{x}-\mu_1)+log\left(\dfrac{\pi_1}{c(1|2)}\right)-(\textbf{x}-\mu_2)^T \mathbf{\Sigma}^{-1} (\textbf{x}-\mu_2)-log\left(\dfrac{\pi_2}{c(2|1)}\right)
\end{equation}

\noindent Hay que tener en cuenta que el término $D_i^2=(\textbf{x}-\mu_i)^T \mathbf{\Sigma}^{-1} (\textbf{x}-\mu_i)$ es la \emph{distancia de Mahalanobis} de la observación \textbf{x} a la media de $i-$ésima población. En cambio, los  términos logarítmicos son términos que ponderan el coste y la probabilidad a priori, de manera que si la probabilidad es menor que el coste, penalizarán esa clasificación y si no la potencian. En el caso de que tanto las probabilidades como el coste fueran iguales, entonces la función discriminante para la población 1 sería:
\begin{equation}
D_1^2-D_2^2
\end{equation}

\noindent De esta manera, se obtiene un criterio geométrico \cite{Cuadras 2014} en el que una nueva observación se clasifica según a que media esté más cerca, es decir cuanto más se parezca la nueva observación a la media de la región más opciopnes tiene de ser clasificada de esa manera.  En particular, \emph{Peña, D.} \cite{Peña 2002} desarrolla el enfoque geométrico desde la función discriminante antes obtenida dando en el caso de 2 variables independientes una visualización de las dos poblaciones.

\noindent \emph{Wichern, D.W. y Johnson, R.A.}\cite{Johnson 2007} desarrollan  el caso cuando no se conoce la matriz de covarianzas poblacional y cuando ambas matrices de cada una de las poblaciones $\pi_1,\pi_2$ no son iguales, obteniendo la regla discriminante cuadrática cuyo razonamiento es similar, sustituyendo las densidades de probabilidad adecuadas.


\subsection{Análisis canónico de poblaciones}

\noindent Hasta ahora, se ha asumido que conocemos las distribuciones de las poblaciones que estamos estudiando. Utilizando estas probabilidades y los costos asociados, hemos desarrollado funciones discriminantes de manera simple. Sin embargo, la realidad rara vez se ajusta a estos supuestos. Por esta razón, hay métodos que afrontan el problema de manera exploratoria, por tanto, se busca analizar como y cuanto las variables influyen a la hora de esta discriminación. 

\noindent Esto se puede realizar teniendo en cuenta cómo es la variabilidad entre grupos y cómo es dentro de los grupos, en este caso buscaremos cuales son las direcciones en la cual se maximiza la variabilidad entre grupos y se minimiza la varianza dentro de las poblaciones.

\noindent A continuación, se describirá el método para calcular una nueva base de observaciones que cumpla con el objetivo antes descrito. Este método se basa en el trabajo de \emph{Lebart L., Morineau A. y Warwick K.M.}\cite{Lebart 1984}, aunque también se puede seguir el desarrollo de \emph{Johnson R.A. y Wichern, D. W. }\cite{Johnson 2007} en el que utilizan la matriz de varianzas entre grupos en vez de la matriz de covarianzas totales.

\noindent Sea $\mathbf{X}$ la matriz de datos de tamaño $N \times p$
donde las filas $\textbf{x}_i$ son cada una de las observaciones de las $p$ variables. Dichas observaciones están particionadas en general por $L$ grupos, sea $I_l$ el conjunto de observaciones pertenecientes al $l$-ésimo grupo, sea también $N_l$ el número de observaciones que pertenecen al $l$-ésimo grupo.

\begin{defi}
Se define la media muestral de una variable como:
\begin{equation}
\overline{x}_j=\frac{1}{N}\sum_{i=1}^N x_{ij}, j=1\ldots p
\end{equation}
\end{defi}

\begin{defi}
Se define la media muestral de la variable $j$-ésima variable en la $l$-ésima población :
\begin{equation}
\overline{x}_{jl}=\frac{1}{N_l}\sum 
_{i\in I_l} x_{ij}
\end{equation}
\end{defi}

\noindent Por ende, podemos dar la distancia entre dos variables como:
\begin{align}
Cov(X_j,X_{j'})&=\dfrac{1}{N}\sum_{i=1}^N(x_{ij}-\overline{x}_j)(x_{ij'}-\overline{x}_j')
\intertext{Esto se puede particionar por grupos de la siguiente manera: }
Cov(X_j,X_{j'})&=\dfrac{1}{N}\sum_{l=1}^L\sum_{i\in I_l}(x_{ij}-\overline{x}_j)(x_{ij'}-\overline{x}_j')
\intertext{y a su vez cada uno de los $(x_{ij}-\overline{x}_j)$ se pueden dividir en la parte intergrupos e intragrupos: }
(x_{ij}-\overline{x}_j)&=(x_{ij}-\overline{x}_{jl})+(\overline{x}_{jl}-\overline{x}_{j})
\end{align}
Sustituyendo y simplificando lo necesario: 
\begin{equation}
Cov(X_j,X_{j'})=\dfrac{1}{N}\sum_{l=1}^L\sum_{i\in I_l}(x_{ij}-\overline{x}_{jl})(x_{ij'}-\overline{x}_{j'l})+\sum_{l=1}^L\dfrac{N_l}{N}(\overline{x}_{jl}-\overline{x}_{j})(\overline{x}_{j'l}-\overline{x}_{j'})
\end{equation}

\noindent Esto nos permite dar una descomposición de la matriz de covarianzas total de la siguiente forma :
\begin{equation}\label{descomposicion varianza}
\textbf{T}=\textbf{B}+\textbf{W}
\end{equation}

Donde:
\begin{itemize}
\item $\mathbf{T}$ es la matriz de tamaño $p\times p$ que expresa la covarianza total y sus coeficientes  $t_{jj'}=Cov(X_j,X_j')$
\item \textbf{B} es la matriz que expresa la covarianza entre los grupos y sus coeficientes son $b_{jj'}=\sum_{k=1}^q\dfrac{n_k}{n}(\overline{x}_{jk}-\overline{x}_{j})(\overline{x}_{j'k}-\overline{x}_{j'})$
\item \textbf{W} es la matriz que expresa la covarianza dentro de los grupos y sus coeficientes son $w_{jj'}=\dfrac{1}{n}\sum_{k=1}^q\sum_{i\in I_k}(x_{ij}-\overline{x}_{jk})(x_{ij'}-\overline{x}_{j'k})$
\end{itemize}

\noindent Para cualquier combinación lineal que se quiera hacer de las variables de entrada de la forma $\textbf{a}^T \textbf{x}$, donde el vector $\textbf{a}$ es un vector de $p$ constantes,  entonces la varianza se transforma de la siguiente manera:
\begin{equation}
Var(\textbf{a}^T \textbf{x})=\textbf{a}^T \Sigma \textbf{a}
\end{equation}

\noindent Entonces transformando por el vector $\textbf{a}$ tenemos que la Ecuación \eqref{descomposicion varianza} se transforma de la siguiente manera: 
\begin{equation}
\textbf{a}^T \textbf{T}\textbf{a}= \textbf{a}^T \textbf{B}\textbf{a}+\textbf{a}^T \textbf{W}\textbf{a}
\end{equation}

\noindent Recopilando, el objetivo del análisis discriminante lineal es encontrar combinaciones lineales que maximicen la varianza entre grupos y minimicen la varianza dentro de los grupos. Eso es equivalente a encontrar el vector $\textbf{a}$ tal que:
\begin{equation}
f(\textbf{a})=\dfrac{\textbf{a}^T \textbf{B}\textbf{a}}{\textbf{a}^T \textbf{T}\textbf{a}}
\end{equation}
\noindent Es máxima. Si además utilizamos la restricción $\textbf{a}^T \textbf{T}\textbf{a} = 1$. En principio la función objetiva es homogénea, es decir, $f(\mu \textbf{a})=f(\textbf{a})$
Utilizando el método de los multiplicadores de Lagrange derivamos respecto del vector $\textbf{a}$ tendremos que:
\begin{align}
L(\textbf{a})= \textbf{a}^T \textbf{B}\textbf{a}&-\lambda(\textbf{a}^T \textbf{T}\textbf{a}-1) 
\intertext{al derivarla respecto de \textbf{a} se obtiene que: }
\dfrac{\partial L(\textbf{a})}{\partial \textbf{a} } = 2\textbf{B}\textbf{a}&-2\lambda\textbf{T}\textbf{a}
\intertext{En consecuencia: }
\textbf{B}\textbf{a} &= \lambda \textbf{T} \textbf{a}
\intertext{Si además \textbf{T} es no singular}
\textbf{T}^{-1}\textbf{B}\textbf{a}&=\lambda \textbf{a}
\end{align}

\noindent Es decir, el vector $\textbf{a}$ es el vector de valor propio $\lambda$, tomando el valor propio máximo de la matriz $\textbf{T}^{-1}\textbf{B}$.

\begin{defi}
Al valor $\lambda$ se le conoce como \textit{potencia discriminante} de la combinación \textbf{a}.
\end{defi}

\begin{propo}
El desarrollo antes detallado es análogo para $f(\textbf{a})=\dfrac{\textbf{a}^T \textbf{B}\textbf{a}}{\textbf{a}^T \textbf{W}\textbf{a}}$ \cite{Johnson 2007}
\end{propo}

\noindent En el caso de que tomamos el desarrollo para obtener los vectores propios de $\mathbf{W^{-1}B}$ es una matriz con rango $r=\min(p,q-1)$ entonces podemos utilizar la matriz de cambio de base $\mathbf{U_r}$ que tiene como columnas los $r$ vectores propios con valores propios no nulos de la matriz $\mathbf{W^{-1}B}$. Sea $\mathbf{x}$ el vector a transformar, entonces  tenemos que $\mathbf{z=U_r^T x}$.

\noindent Aunque el fin de este método no sea este, si se desea determinar si una nueva observación $\mathbf{x}_0$ basta con calcular la distancia con las medias de los grupos transformadas, es decir, $||\mathbf{z_0-U_r\mu_i}||^2$ donde $\mathbf{z}_0$ es el vector en la nueva base y comprobar cual es la menor. 

\noindent Para determinar cuantas variables canónicas se puede utilizar la varianza entre grupos que explica cada variable canónica. 

\begin{propo}
La varianza explicada por cada variable canónica es el valor propio de la matriz $\mathbf{W^{-1} B}$ de su vector propio asociado, su \emph{potencia discriminante}. 
\begin{proof}
Sea el vector propio $\textbf{a}_i$ con valor propio $\lambda_i$ entonces tendremos que por construcción $\mathbf{a}_i^T \mathbf{W a}_i=1$, entonces los vectores son unitarios respecto a la métrica que induce la matriz de covarianzas intra-grupos, entonces la varianza explicada por cada variable es $VE(\textbf{a}_i)=\mathbf{a}_i^T\mathbf{Ba}_i$, por ser valores propios de $\mathbf{W^{-1}B}$ se cumple que $\mathbf{Ba}_i=\lambda_i \mathbf{W a}_i$, entonces
\begin{equation}
VE(\textbf{a}_i)\mathbf{a}_i^T\mathbf{Ba}_i=\lambda_i \mathbf{a}_i^T\mathbf{Wa}_i=\lambda_i\qedhere
\end{equation}
\end{proof}
\end{propo}

\noindent Para simplificar los datos, podemos seleccionar un número de variables canónicas estableciendo un umbral. Elegiremos las $m$ variables cuya varianza explicada acumulada sea mayor. Esto nos permite reducir la dimensionalidad de los datos de manera similar al Análisis de Componentes Principales. Sin embargo, hay una diferencia clave: mientras el Análisis de Componentes Principales busca representar la variabilidad de los datos de manera simple, las variables canónicas buscan identificar las direcciones en las cuales las poblaciones se distinguen de manera más significativa.

