\newpage
\section{Clasificación}

\noindent Hasta ahora se ha considerado el caso en el que la variable respuesta, es una variable continua en la cual se puede definir una métrica del tipo habitual \cite{Hair 1995}.  

\noindent En los casos de la clasificación o discriminación, se tienen variables respuestas que son cualitativas o categóricas. Por tanto, el modo de tratar los datos cambia de manera significativa en un primer momento.

\noindent En este caso, la clasificación y la discriminación se diferencian, según Johnson, y Wichern, en que la discriminación busca describir, ya sea gráficamente (análisis canónico de poblaciones) o algebraicamente (análisis discriminante lineal, logísitico o cuadrático), las características de las distintas poblaciones \cite{Johnson 2007}. En el otro lado, la clasificación busca asociar nuevas observaciones a las distintas clases o poblaciones. Aún así, las distintas  técnicas descritas se pueden usar para ambas tareas, ya que al describir las características que posee una cierta población se pueden realizar nuevas asignaciones.  

\noindent Las distintas técnicas de estudio de la dependencia entre variables continuas y categóricas o cualitativas asumen hipótesis y buscan objetivos distintos:
\begin{itemize}

\item En el caso de que se conozcan los parámetros de las distribuciones se pueden crear funciones discriminantes que permitan clasificar nuevas observaciones $\mathbf{x}_0$ en una de las poblaciones. A esto se le llama clasificación, uno de los métodos más habituales es la creación de funciones discriminantes por máxima verosimilitud o siguiendo un criterio geométrico \cite{Peña 2002}. Es en este caso, en el que se difuminan los límites entre discriminación y clasificación, en particular, se hace un trabajo de discriminación para luego realizar nuevas clasificaciones.

\item Cuando no se conozca las distribuciones de probabilidad de las distintas poblaciones, se puede utilizar con fines exploratorios el análisis canónico de poblaciones  en el que se estudian las direcciones de mayor distancia entre las distintas poblaciones. También se le llama análisis discriminante lineal de Fisher \cite{Lebart 1984}.
\end{itemize}

\noindent En la primera parte donde se desarrollan las funciones discriminantes, para simplificar los cálculos y los desarrollos, se consideran conocidas las distribuciones de probabilidad de dos poblaciones \cite{Johnson 2007}. Luego se asumirá que las distribuciones son normales para construir el discriminante lineal habitual, en la que aparece la distancia de Mahalanobis \cite{Peña 2002}. Aunque Morrison  construye dicha función discriminante buscando una combinación lineal que maximice el estadístico $T^2$ de Hotelling el cual mide la diferencia de las medias entre varias poblaciones \cite{ Johnson 2007, Mardia 1979, Morrison 1976}. Añadir que para esto último, se utiliza la matriz de covarianzas muestral, mientras que el desarrollo dado durante la memoria se hará suponiendo conocida la matriz de covarianzas poblacional.  

\noindent En la segunda parte de la sección se detalla el método del análisis canónico de poblaciones o análisis discriminante de Fisher que toma la matriz de covarianzas dentro y entre las poblaciones, permitiendo reducir la dimensión de la matriz en un espacio de dimensión menor \cite{Lebart 1984}. 

\subsection{Funciones discriminantes}

\noindent Supóngase que se tienen dos valores de la variable respuesta $Y$, entonces, se tienen dos poblaciones, $\pi_1$, $\pi_2$ de las cuales se conocen sus funciones de probabilidad, $f_1$, $f_2$ respectivamente. Si además, se conoce la probabilidad de clasificación errónea $P(1|2),P(2|1)$, se tiene que:
\begin{equation}
P(i|\textbf{x}_0)=\dfrac{P(\textbf{x}_0|i)P(i)}{P(1)P(\textbf{x}_0|1)+P(2)P(\textbf{x}_0|2)}=\dfrac{f_i(\mathbf{x}_0)P(i)}{P(1)f_1(\mathbf{x}_0)+P(2)f_2(\mathbf{x}_0)} \quad i=1,2
\end{equation}

\noindent Donde $P(1)$, $P(2)$ son las probabilidades de pertenecer a cada una de las poblaciones. Por tanto, se puede clasificar en una población u otra ya que se conoce la distribución de cada una de las poblaciones. Primero definamos lo que es una función discriminante \cite{Cuadras 2014}:

\begin{defi}
Se llama función discriminante, $f_{d}$ aquella que:
\begin{equation}
f_{d}:\mathbf{\Omega}\longrightarrow \mathbb{R},
\end{equation}
donde $\mathbf{\Omega}$ es el espacio de observaciones posibles, definida de tal manera que si $f_{d}(\textbf{x}_0)>0\Rightarrow \textbf{x}_0\in \pi_i$ y en caso contrario $\textbf{x}_0\notin \pi_i$. 
\end{defi}
\noindent En el caso anterior de dos poblaciones:
\begin{equation}
 f_d(\mathbf{x}_0)=P(1|\mathbf{x}_0)-P(2|\mathbf{x}_0)=f_1(\textbf{x}_0)P(1)-f_2(\textbf{x}_0)P(2)
\end{equation} 
ésta sería la función discriminante más simple  para discernir si $\textbf{x}_0$ pertenece a $\pi_1$ o no.

\begin{defi}
Se llama error de clasificación al coste de clasificar en una población erróneamente una observación y se denota como $c(i|j)=$\emph{``Error de clasificar en $\pi_i$ una observación perteneciente a $\pi_j$"}.
\end{defi}

\noindent En general, la función de pérdida de un problema de clasificación se da en forma de matriz de pérdida $\mathbf{L}$ de tamaño $L\times L$ con la diagonal nula, que puede ser simétrica o no y donde $L_{ij}=c(i|j)$ \cite{Hastie 2001}. 

\noindent Podemos penalizar las probabilidades de cada una de las poblaciones utilizando el coste. De esta manera, aunque se cometa error se potencia clasificar en la población que menor coste tenga:

\begin{equation}
f_d(\mathbf{x}_0)=\dfrac{f_1(\textbf{x}_0)P(1)}{c(1|2)}-\dfrac{f_2(\textbf{x}_0)P(2)}{c(2|1)}
\end{equation}

\noindent En el caso de que la probabilidad de pertenecer a cada una de las dos poblaciones sea la misma y el coste de clasificar erróneamente sea distinto, se clasificará como aquella que menor coste conlleve.

\noindent En resumen, cuando los demás términos sean iguales, elegiremos la opción con el menor costo, la mayor verosimilitud o la mayor probabilidad a priori \cite{Peña 2002}.

\noindent Johnson  y Wichern  desarrollan otro enfoque, tomando el coste medio esperado (\emph{ECM} en inglés) \cite{Johnson 2007}:
\begin{equation}
ECM=c(2|1)P(2|1)P(1)+c(1|2)P(1|2)P(2)
\end{equation}

\noindent Además afirman que las regiones que minimizan el coste de clasificación son las que vienen dadas por la función discriminante anterior. 

\noindent Supóngase ahora que tanto $f_1,f_2$ son las funciones de densidad de distribuciones normales con medias distintas $\mu_1,\mu_2$ y matriz de covarianzas común $\mathbf{\Sigma}$. Entonces tenemos la siguiente expresión \cite{ Cuadras 2014, Johnson 2007}:
\begin{equation}
f_i(\textbf{x})=\dfrac{1}{(2\pi)^{\frac{p}{2}}|\mathbf{\Sigma}|^{\frac{1}{2}}} e^{ \left(\dfrac{-1}{2}(\textbf{x}-\mu_i)^T \mathbf{\Sigma}^{-1}(\textbf{x}-\mu_i) \right)}
\end{equation}

\noindent De esta manera, la función discriminante se puede transformar sustituyendo las funciones de densidad $f_1,f_2$ por la expresión anterior. Tomando logaritmos se obtiene: 
\begin{equation}
(\textbf{x}-\mu_1)^T \mathbf{\Sigma}^{-1}(\textbf{x}-\mu_1)+log\left(\dfrac{P(1)}{c(1|2)}\right)-(\textbf{x}-\mu_2)^T \mathbf{\Sigma}^{-1} (\textbf{x}-\mu_2)-log\left(\dfrac{P(2)}{c(2|1)}\right)
\end{equation}

\noindent Hay que tener en cuenta que el término $D_i^2=(\textbf{x}-\mu_i)^T \mathbf{\Sigma}^{-1} (\textbf{x}-\mu_i)$ es la distancia de Mahalanobis (definición \ref{Mahalanobis}) de la observación \textbf{x} a la media de $i-$ésima población. En cambio, los  términos logarítmicos son términos que ponderan el coste y la probabilidad a priori, de manera que si la probabilidad es menor que el coste, penalizarán esa clasificación y si no la potencian. En el caso de que tanto las probabilidades como el coste fueran iguales, entonces la función discriminante para la población 1 sería:
\begin{equation}
D_1^2-D_2^2
\end{equation}

\noindent De esta manera, se obtiene un criterio geométrico  en el que una nueva observación se clasifica según a que media esté más cerca. Es decir, cuanto más se parezca la nueva observación a la media de la región más opciones tiene de ser clasificada en esa población \cite{Cuadras 2014}.  En particular, Peña desarrolla el enfoque geométrico desde la función discriminante antes obtenida dando, en el caso de dos variables independientes, una visualización de las dos poblaciones \cite{Peña 2002}.

\noindent Wichern y Johnson, desarrollan el caso cuando no se conoce la matriz de covarianzas poblacional y cuando ambas matrices de cada una de las poblaciones $\pi_1$, $\pi_2$ no son iguales, obteniendo la regla discriminante cuadrática cuyo razonamiento es similar, sustituyendo las funciones de densidad adecuadas \cite{Johnson 2007}.

\subsection{Análisis canónico de poblaciones}

\noindent Hasta ahora, se ha asumido que conocemos las distribuciones de las poblaciones que estamos estudiando. Utilizando estas probabilidades y los costes asociados, se han desarrollado funciones discriminantes. Sin embargo, hay métodos que afrontan el problema de manera exploratoria, por tanto, se busca analizar cómo  se estructura la diferenciación entre las poblaciones. 

\noindent Este estudio de las diferencias entre poblaciones se puede realizar teniendo en cuenta cómo es la variabilidad entre poblaciones y cómo es dentro de las poblaciones. En este caso, se buscarán  cuales son las direcciones en las cuales se maximiza la variabilidad entre grupos con respecto a la varianza dentro de las poblaciones.

\noindent A continuación, se describe el método para calcular una nueva base del espacio de las variables que cumpla con el objetivo antes descrito \cite{Johnson 2007, Lebart 1984}.

\noindent Sea $\mathbf{X}$ la matriz de datos de tamaño $N \times p$
donde las filas $\textbf{x}_i$ son cada una de las observaciones de las $p$ variables. Dichas observaciones están particionadas  por $L$ poblaciones. Sea $I_l$ el conjunto de observaciones pertenecientes al $l$-ésimo grupo $\forall l=1,\ldots, L$, sea también $N_l$ el número de observaciones que pertenecen al $l$-ésimo grupo $\forall l=1,\ldots, L$.

\begin{defi}
Se define la media muestral de la $j$-ésima variable como:
\begin{equation}
\overline{x}_j=\frac{1}{N}\sum_{i=1}^N x_{ij}, \forall j=1\ldots p
\end{equation}
Para un conjunto de $N$ observaciones realizadas en $p$ variables. 
\end{defi}

\begin{defi}
Se define la media muestral de la  $j$-ésima variable en la $l$-ésima población :
\begin{equation}
\overline{x}_{jl}=\frac{1}{N_l}\sum 
_{i\in I_l} x_{ij}
\end{equation}
\end{defi}

\noindent Por ende, podemos evaluar la relación entre dos variables $X_j, X_{j'}$ como:
\begin{equation}
Cov(X_j,X_{j'})=\dfrac{1}{N}\sum_{i=1}^N(x_{ij}-\overline{x}_j)(x_{ij'}-\overline{x}_{j'})
\end{equation}

\noindent A su vez, cada uno de los $(x_{ij}-\overline{x}_j)$ se pueden dividir en la parte intergrupos e intragrupos.


\begin{defi}
\noindent La covarianza intragrupo de un par de variables $X_j,X_{j'}$ en la $l$-ésima población se define como:
\begin{equation}
W_l(X_j,X_{j'})=\dfrac{1}{N_l}\sum_{i\in I_l} (x_{ij}-\overline{x}_{jl})(x_{ij'}-\overline{x}_{j'l})
\end{equation}   
\end{defi}
\begin{defi}
\noindent La covarianza intergrupos de un par de variables $X_j,X_{j'}$  se define como:
\begin{equation}
B(X_j,X_{j'})=\sum_{l=1}^L\dfrac{N_l}{N}(\overline{x}_{jl}-\overline{x}_{j})(\overline{x}_{j'l}-\overline{x}_{j'})
\end{equation}   
\end{defi}
\noindent Es decir, se puede simplificar la covarianza entre dos variables teniendo en cuenta la siguiente descomposición. 
\begin{equation}
(x_{ij}-\overline{x}_j)=(x_{ij}-\overline{x}_{jl})+(\overline{x}_{jl}-\overline{x}_{j})
\end{equation}
\noindent Sustituyendo y simplificando lo necesario: 
\begin{equation}
Cov(X_j,X_{j'})=\dfrac{1}{N}\sum_{l=1}^L\sum_{i\in I_l}(x_{ij}-\overline{x}_{jl})(x_{ij'}-\overline{x}_{j'l})+\sum_{l=1}^L\dfrac{N_l}{N}(\overline{x}_{jl}-\overline{x}_{j})(\overline{x}_{j'l}-\overline{x}_{j'})
\end{equation}

\noindent Esto nos permite dar una descomposición de la matriz de covarianzas total de la siguiente forma :
\begin{equation}\label{descomposicion varianza}
\textbf{T}=\textbf{W}+\textbf{B}
\end{equation}

Donde:
\begin{itemize}
\item $\mathbf{T}$ es la matriz de tamaño $p\times p$ que expresa la covarianza total y sus coeficientes  $t_{jj'}=Cov(X_j,X_j'),\quad \forall j,j'=1,\ldots, p$
\item \textbf{W} es la matriz que expresa la covarianza dentro de los grupos y sus coeficientes son $w_{jj'}=\dfrac{1}{N}\sum_{l=1}^L\sum_{i\in I_l}(x_{ij}-\overline{x}_{jl})(x_{ij'}-\overline{x}_{j'l}),\forall j,j'=1,\ldots, p$
\item \textbf{B} es la matriz que expresa la covarianza entre los grupos y sus coeficientes son $b_{jj'}=\sum_{l=1}^L\dfrac{N_l}{N}(\overline{x}_{jl}-\overline{x}_{j})(\overline{x}_{j'l}-\overline{x}_{j'})\forall j,j'=1,\ldots, p$

\end{itemize}

\noindent Para cualquier combinación lineal que se quiera hacer de las variables de entrada $X_1,\ldots, X_p$ de la forma $\mathbf{x a}$, donde el vector $\textbf{a}$ es un vector columna de $p$ constantes. La varianza se transforma de la siguiente manera:
\begin{equation}
Var(\mathbf{xa})=\textbf{a}^T \Sigma \textbf{a}
\end{equation}

\noindent Entonces transformando por el vector $\textbf{a}$ tenemos que la ecuación \eqref{descomposicion varianza} se escribe de la siguiente manera: 
\begin{equation}
\textbf{a}^T \textbf{T}\textbf{a}= \textbf{a}^T \textbf{B}\textbf{a}+\textbf{a}^T \textbf{W}\textbf{a}
\end{equation}

\noindent Recopilando, el objetivo del análisis discriminante lineal es encontrar combinaciones lineales que maximicen la varianza entre grupos con respecto a la varianza total. Eso es equivalente a encontrar el vector $\textbf{a}$ tal que \cite{Lebart 1984}:
\begin{equation}
f(\textbf{a})=\dfrac{\textbf{a}^T \textbf{B}\textbf{a}}{\textbf{a}^T \textbf{T}\textbf{a}}
\end{equation}
\noindent sea máxima. utilizando la restricción $\textbf{a}^T \textbf{T}\textbf{a} = 1$.
Utilizando el método de los multiplicadores de Lagrange derivamos respecto del vector $\textbf{a}$ tendremos que:
\begin{align}
L(\textbf{a})= \textbf{a}^T \textbf{B}\textbf{a}&-\lambda(\textbf{a}^T \textbf{T}\textbf{a}-1) 
\intertext{al derivarla respecto de \textbf{a} se obtiene que: }
\dfrac{\partial L(\textbf{a})}{\partial \textbf{a} } = 2\textbf{B}\textbf{a}&-2\lambda\textbf{T}\textbf{a}
\intertext{En consecuencia: }
\textbf{B}\textbf{a} &= \lambda \textbf{T} \textbf{a}
\intertext{Si además \textbf{T} es no singular}
\textbf{T}^{-1}\textbf{B}\textbf{a}&=\lambda \textbf{a}
\end{align}

\noindent Es decir, el vector $\textbf{a}$ es el vector de valor propio $\lambda$ de la matriz $\textbf{T}^{-1}\textbf{B}$, tomando el valor propio máximo de la matriz $\textbf{T}^{-1}\textbf{B}$ se concluye.
 
\noindent Para el resto de variables canónicas se debe establecer además la ortogonalidad entre variables lo que implicará la no correlación entre estas. Esto se detalla más en profundidad en la sección del análisis de componentes principales.  

\begin{defi}
Al valor $\lambda$ se le conoce como potencia discriminante de los pesos de la combinación \textbf{a}.
\end{defi}
\noindent Johnson y Wichern realizan un desarrollo análogo maximizando $\dfrac{\textbf{a}^T \textbf{B}\textbf{a}}{\textbf{a}^T \textbf{W}\textbf{a}}$ \cite{Johnson 2007}. 

\noindent Las variables canónicas aunque no sea su objetivo principal también pueden usarse para construir funciones discriminantes en pos de clasificar nuevas observaciones. Sea $\mathbf{x}_0$ la nueva observación y $\mathbf{x}'_0$ su expresión en la nueva base de las variables canónicas, sea además $\overline{\mathbf{x}}_l$ la media muestral de cada una de las $l=1,\ldots, L$ poblaciones y $\overline{\mathbf{x}}'_l$ su expresión en la nueva base $\forall l=1,\ldots, L$. Se puede clasificar la nueva observación en la $l$-ésima población como aquella que tenga la mínima distancia a la media de la población $l$-ésima. 

\noindent Por otro lado, para determinar cuantas variables canónicas van a estudiarse se puede utilizar la varianza entre grupos que explica cada variable canónica. 

\begin{propo}
La varianza entre grupos explicada por cada variable canónica es el valor propio de la matriz $\mathbf{T}^{-1} \mathbf{B}$ de su vector propio asociado, su potencia discriminante. 
\begin{proof}
Sea el vector propio $\textbf{a}_i$ con valor propio $\lambda_i$ entonces tendremos que por construcción $\mathbf{a}_i^T \mathbf{T a}_i=1$, entonces los vectores son unitarios respecto a la métrica que induce la matriz de covarianzas. Entonces,  la varianza explicada por cada variable es $B(\textbf{a}_i\mathbf{x})=\mathbf{a}_i^T\mathbf{Ba}_i$, por ser valores propios de $\mathbf{T}^{-1}\mathbf{B}$ se cumple que $\mathbf{Ba}_i=\lambda_i \mathbf{T a}_i$, entonces
\begin{equation}
B(\textbf{a}_i\mathbf{x})=\mathbf{a}_i^T\mathbf{Ba}_i=\lambda_i \mathbf{a}_i^T\mathbf{Ta}_i=\lambda_i\qedhere
\end{equation}
\end{proof}
\end{propo}


\noindent Para reducir la dimensionalidad, podemos seleccionar un número de variables canónicas estableciendo un umbral. Elegiremos las $m< min(p,L-1)$ primeras variables canónicas cuya varianza explicada acumulada sea mayor que dicho umbral \cite{Johnson 2007}. Esto nos permite reducir la dimensionalidad de los datos de manera similar al análisis de componentes principales. Sin embargo, hay una diferencia clave: mientras el análisis de componentes principales busca representar la variabilidad de los datos explicando la máxima variabilidad total, las variables canónicas buscan identificar las direcciones en las cuales las poblaciones se distinguen de manera más significativa.

