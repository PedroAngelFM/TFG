\newpage
\section{Clasificación supervisada}

\noindent Supóngase un conjunto de observaciones de un vector aleatorio $\textbf{x}$ de tamaño $p$ las cuales pertenecen a dos o más poblaciones conocidas a priori. El objetivo es dada una nueva observación $\textbf{x}_0$ poder asignarla a una de las poblaciones. Otro objetivo que se puede buscar con estas técnicas es determinar que variables son determinantes y tienen mayor importancia a la hora de discriminar entre varias categorías. 

\noindent Este problema se le conoce como \emph{Clasificación Supervisada}, ya que  se conoce un conjunto de observaciones de las cuales se sabe a que población pertenece. También se le conoce como \emph{Análisis Discriminante}, ya que se pueden establecer funciones que permitan discriminar observaciones unas de otras. 

\noindent Alguna de las aplicaciones que utilizan las técnicas que se describen en esta sección son el reconocimiento de patrones, la concesión de créditos, o asignación de textos a autores. 

\noindent Las primeras aproximaciones que se dieron de estas técnicas fueron dadas por \emph{Fisher R.} en 1936 introduciendo el conjunto de datos  \emph{Iris de Fisher} para la clasificación. 

\noindent Durante esta sección se seguirán \emph{Hastie T. et.al.}\cite{Hastie 2001}, \emph{Cuadras C.M.}\cite{Cuadras 2014} y \emph{Peña D.} \cite{Peña 2002}

\subsection{Formalización}

\noindent Supóngase dadas dos poblaciones $P_1,P_2 $ de las cuales son conocidas sus funciones de densidad $f_1, f_2 $. Si además se conocen las probabilidades a priori de pertenencia a cada una de las poblaciones $\pi_1,\pi_2$ Entonces la probabilidad conociendo el valor de $\textbf{x}_0$ de pertencia a $i=1,2$ es :
\begin{equation}
P(i|\textbf{x}_0)=\dfrac{P(\textbf{x}_0|i)\pi_i}{\pi_1P(\textbf{x}_0|1)+\pi_2P(\textbf{x}_0|2)}=\dfrac{f_i(\textbf{x}_0)\pi_i}{f_1(\textbf{x}_0)\pi_1+f_2(\textbf{x}_0)\pi_2}
\end{equation}

\noindent De esta manera se puede clasificar como una población dependiendo de que $f_i(\textbf{x}_0)\pi_i$ sea mayor \emph{(Esto es generalizable a más poblaciones.)}

\begin{defi}
Se llama \textit{función discriminante } aquella que:
\begin{equation}
f_i:\mathbf{\Omega}\longrightarrow \mathbb{R}
\end{equation}
Donde $\Omega$ es el espacio de observaciones posibles. Definida de tal manera que si $f_i(\textbf{x}_0)>0\Rightarrow \textbf{x}_0\in P_i$ y en caso contrario $\textbf{x}_0\notin P_i$. 
\end{defi}
\noindent En el caso anterior de dos poblaciones, $f_1(\textbf{x}_0)\pi_1-f_2(\textbf{x}_0)\pi_2$ sería la función discriminante para discernir si $\textbf{x}_0$ pertenece a $P_1$. Ya que representa la probabilidad de ser de dicha clase conociendo el valor de las variables predictoras.

\begin{defi}
Se llama \emph{Error de clasificación} al coste de clasificar de manera errónea una observación y se denota como $c(i|j)=$\emph{``Error de clasificar como $P_i$ una observación perteneciente a $P_j$"}
\end{defi}

\noindent De esta manera, si el coste de mala clasificación en una y otra población son distintas podemos trasladarlo a la función discriminante de la siguiente manera:
\begin{equation}
\dfrac{f_1(\textbf{x}_0)\pi_1}{c(1|2)}-\dfrac{f_2(\textbf{x}_0)\pi_2}{c(2|1)}
\end{equation}

\noindent Por consiguiente, a igualdad de los otros términos escogeremos el que menos coste, el que mayor verosimilitud  o el que mayor probabilidad a priori tenga.

\noindent Supóngase ahora que tanto $f_1,f_2$ son densidades normales con medias distintas $\mu_1,\mu_2$ pero con una matriz de covarianzas común $\mathbf{\Sigma}$:
\begin{equation}
f_i(\textbf{x})=\dfrac{1}{(2\pi)^{\frac{p}{2}}|\mathbf{\Sigma}|^{\frac{1}{2}}} exp \left\lbrace\dfrac{-1}{2}(\textbf{x}-\mu_i)^T \mathbf{\Sigma}^{-1}(\textbf{x}-\mu_i) \right\rbrace
\end{equation}

\noindent De esta manera la función discriminante se puede transformar sustituyendo las densidades por la expresión anterior y tomando logaritmos : 
\begin{equation}
(\textbf{x}-\mu_1)^T \mathbf{\Sigma}^{-1}(\textbf{x}-\mu_1)+log\left(\dfrac{\pi_1}{c(1|2)}\right)-(\textbf{x}-\mu_2)^T \mathbf{\Sigma}^{-1} (\textbf{x}-\mu_2)-log\left(\dfrac{\pi_2}{c(2|1)}\right)
\end{equation}

\noindent Hay que tener en cuenta que el término $D_i^2=(\textbf{x}-\mu_i)^T \mathbf{\Sigma}^{-1} (\textbf{x}-\mu_i)$ se puede interpretar como la \emph{distancia de Mahalanobis} de la observación \textbf{x} a la $i-$ésima población. Por otro lado, se tiene un término que relaciona probabilidad de pertenencia con el coste de clasificación errónea. En el caso de que tanto las probabilidades como el coste fueran iguales, entonces la función discriminante para la población 1 sería:
\begin{equation}
D_2^2 > D_1^2
\end{equation}

\begin{defi}
Llamaremos \emph{frontera de decisión} al hiperplano que divide el espacio de observaciones en tantas regiones como poblaciones haya. En el caso de una función discriminante como las desarrolladas la frontera son las observaciones tales que $f(\textbf{x})=0$
\end{defi}
\noindent Para obtener la frontera de decisión se igualan ambos términos y desarrollando ambas distancias podríamos desechar el término $\mathbf{x^T\Sigma^{-1} x}$ ya que es común a ambas, obteniendo la siguiente expresión:
\begin{equation}
.
\end{equation}



\subsection{Análisis mediante Variables Canónicas discriminantes}

\noindent Hasta ahora, se han supuesto conocidas las distribuciones de las poblaciones estudiadas de manera que utilizando las probabilidades de pertenencia y los costes se desarrollan funciones discriminantes de manera sencilla. Pero la realidad pocas veces se ajusta a estos supuestos, es por ello que surgen las técnicas que descomponen la varianza en dos términos la varianza dentro de los grupos y fuera de ellos de tal manera que se busca minimizar la primera y maximizar la segunda. 

\noindent Es por ello, que a continuación se detallará el método por el cual se va a calcular una nueva base del espacio de observaciones que cumpla lo anterior, minimizar la variación entre poblaciones y maximizar la variabilidad entre grupos siguiendo el desarrollo dado por \textit{Lebart L., Morineau, A. y Warwick K.M.} \cite{Lebart 1984} 

\noindent Sea \textbf{X} la matriz de datos de tamaño $n \times p$
donde las filas $\textbf{x}_i$ son cada una de las observaciones de las $p$ variables. Dichas observaciones están particionadas en general por $q$ grupos, sea $I_k$ el conjunto de observaciones pertenecientes al $k$-ésimo grupo, sea también $n_k$ el número de observaciones que pertenecen al $k$-ésimo grupo.

\noindent Se definen las medias muestrales $\overline{x}_j=\frac{1}{n}\sum_{i=1}^n x_{ij}$ de la $j$-ésima variable en la población en total. También se define la media muestral dentro de cada grupo que es $\overline{x}_{jk}=\frac{1}{n_k}\sum 
_{i\in I_k} x_{ij}$ 

\noindent Por ende podemos dar la distancia entre dos variables como:
\begin{align}
Cov(X_j,X_{j'})&=\dfrac{1}{n}\sum_{i=1}^n(x_{ij}-\overline{x}_j)(x_{ij'}-\overline{x}_j')
\intertext{Esto se puede particionar por grupos de la siguiente manera: }
Cov(X_j,X_{j'})&=\dfrac{1}{n}\sum_{k=1}^q\sum_{i\in I_k}(x_{ij}-\overline{x}_j)(x_{ij'}-\overline{x}_j')
\intertext{y a su vez cada uno de los $(x_{ij}-\overline{x}_j)$ se pueden dividir en la parte intergrupos e intragrupos: }
(x_{ij}-\overline{x}_j)&=(x_{ij}-\overline{x}_{jk})+(\overline{x}_{jk}-\overline{x}_{j})
\end{align}
Sustituyendo y simplificando lo necesario: 
\begin{equation}
Cov(X_j,X_{j'})=\dfrac{1}{n}\sum_{k=1}^q\sum_{i\in I_k}(x_{ij}-\overline{x}_{jk})(x_{ij'}-\overline{x}_{j'k})+\sum_{k=1}^q\dfrac{n_k}{n}(\overline{x}_{jk}-\overline{x}_{j})(\overline{x}_{j'k}-\overline{x}_{j'})
\end{equation}

\noindent Esto nos permite dar una descomposición de la matriz de covarianzas total de la siguiente forma :
\begin{equation}\label{descomposicion varianza}
\textbf{T}=\textbf{B}+\textbf{W}
\end{equation}

Donde:
\begin{itemize}
\item \textbf{T} es la matriz que expresa la covarianza total y sus coeficientes  $t_{jj'}=Cov(X_j,X_j')$
\item \textbf{B} es la matriz que expresa la covarianza entre los grupos y sus coeficientes son $b_{jj'}=\sum_{k=1}^q\dfrac{n_k}{n}(\overline{x}_{jk}-\overline{x}_{j})(\overline{x}_{j'k}-\overline{x}_{j'})$
\item \textbf{W} es la matriz que expresa la covarianza dentro de los grupos y sus coeficientes son $w_{jj'}=\dfrac{1}{n}\sum_{k=1}^q\sum_{i\in I_k}(x_{ij}-\overline{x}_{jk})(x_{ij'}-\overline{x}_{j'k})$
\end{itemize}

\noindent Para cualquier combinación lineal que se quiera hacer de las variables de entrada de la forma $\textbf{a}^T \textbf{x}$, donde el vector $\textbf{a}$ es un vector de $p$ constantes,  entonces la varianza se transforma de la siguiente manera:
\begin{equation}
Var(\textbf{a}^T \textbf{x})=\textbf{a}^T \Sigma \textbf{a}
\end{equation}

\noindent Entonces transformando por el vector $\textbf{a}$ tenemos que la Ecuación \eqref{descomposicion varianza} se transforma de la siguiente manera: 
\begin{equation}
\textbf{a}^T \textbf{T}\textbf{a}= \textbf{a}^T \textbf{B}\textbf{a}+\textbf{a}^T \textbf{W}\textbf{a}
\end{equation}

\noindent Recopilando, el objetivo del análisis discriminante lineal es encontrar combinaciones lineales que maximicen la varianza entre grupos y minimicen la varianza dentro de los grupos. Eso es equivalente a encontrar el vector $\textbf{a}$ tal que:
\begin{equation}
f(\textbf{a})=\dfrac{\textbf{a}^T \textbf{B}\textbf{a}}{\textbf{a}^T \textbf{T}\textbf{a}}
\end{equation}
\noindent Es máxima. Si además utilizamos la restricción $\textbf{a}^T \textbf{T}\textbf{a} = 1$. En principio la función objetiva es homogénea, es decir, $f(\mu \textbf{a})=f(\textbf{a})$
Utilizando el método de los multiplicadores de Lagrange derivamos respecto del vector $\textbf{a}$ tendremos que:
\begin{align}
L(\textbf{a})= \textbf{a}^T \textbf{B}\textbf{a}&-\lambda(\textbf{a}^T \textbf{T}\textbf{a}-1) 
\intertext{al derivarla respecto de \textbf{a} se obtiene que: }
\dfrac{\partial L(\textbf{a})}{\partial \textbf{a} } = 2\textbf{B}\textbf{a}&-2\lambda\textbf{T}\textbf{a}
\intertext{En consecuencia: }
\textbf{B}\textbf{a} &= \lambda \textbf{T} \textbf{a}
\intertext{Si además \textbf{T} es no singular}
\textbf{T}^{-1}\textbf{B}\textbf{a}&=\lambda \textbf{a}
\end{align}

\noindent Es decir, el vector $\textbf{a}$ es el vector de valor propio $\lambda$, tomando el valor propio máximo de la matriz $\textbf{T}^{-1}\textbf{B}$.

\begin{defi}
Al valor $\lambda$ se le conoce como \textit{potencia discriminante} de la combinación \textbf{a}.
\end{defi}

\begin{propo}
El desarrollo antes detallado es análogo para $f(\textbf{a})=\dfrac{\textbf{a}^T \textbf{B}\textbf{a}}{\textbf{a}^T \textbf{W}\textbf{a}}$
\end{propo}

\noindent En el caso de que tomamos el desarrollo para obtener los vectores propios de $\mathbf{W^{-1}B}$ es una matriz con rango $r=\min(p,q-1)$ entonces podemos utilizar la matriz de cambio de base $\mathbf{U_r}$ que tiene como columnas los vectores propios de la matriz $\mathbf{W^{-1}B}$ y $\mathbf{x}$ el vector a transformar, de esta manera, tenemos que $\mathbf{z=U_r^T x}$.

\noindent A continuación para determinar si una nueva observación $\mathbf{x}_0$ basta con calcular la distancia con las medias de los grupos transformadas, es decir, $||\mathbf{z_0-U_r\mu_i}||^2$ y comprobar cual es la menor. 

\noindent Para determinar cuantas variables canónicas se puede utilizar la varianza entre grupos que explica cada variable canónica. 

\begin{propo}
La varianza explicada por cada variable canónica es el valor propio de la matriz $\mathbf{W^{-1} B}$ de su vector propio asociado, su \emph{potencia discriminante}. 
\begin{proof}
Sea el vector propio $\textbf{a}_i$ con valor propio $\lambda_i$ entonces tendremos que por construcción $\mathbf{a}_i^T \mathbf{W a}_i=1$, entonces los vectores son unitarios respecto a la métrica que induce la matriz de covarianzas intra-grupos, entonces la varianza explicada por cada variable es $VE(\textbf{a}_i)=\mathbf{a}_i^T\mathbf{Ba}_i$, por ser valores propios de $\mathbf{W^{-1}B}$ se cumple que $\mathbf{Ba}_i=\lambda_i \mathbf{W a}_i$, entonces
\begin{equation}
VE(\textbf{a}_i)\mathbf{a}_i^T\mathbf{Ba}_i=\lambda_i \mathbf{a}_i^T\mathbf{Wa}_i=\lambda_i\qedhere
\end{equation}
\end{proof}
\end{propo}

\noindent Para elegir un número de variables canónicas a usar se puede fijar un umbral por el cual tomemos las $m$ variables cuya varianza explicada acumulada sea superior. De esta manera, se puede llevar a cabo una reducción de la dimensionalidad de los datos de manera parecida a la que se desarrollará en el \emph{Análisis de Componentes Principales.} La principal diferencia con este último es que mientras uno busca representar de una manera simple la variabilidad de los datos, esta busca las direcciones en las cuales los grupos o clases se distinguen de manera más significativa