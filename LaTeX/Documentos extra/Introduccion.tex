\addcontentsline{toc}{chapter}{Introducción}
\chapter*{Introducción}
El análisis multivariante se define como la rama del análisis estadístico que interpreta de manera simultánea la relación entre más de dos variables. Este rama del ha experimentado una gran expansión tanto en investigación como en aplicación debido al avance de la capacidad de computación de los actuales ordenadores haciendo que la posibilidad de  

En la primera parte de la memoria se abordarán los llamados métodos supervisados. Estos son aquellos que, dado un conjunto de variables de entrada observadas $X_1 \ldots X_p$ nos permiten predecir de distintas maneras una variable de salida. En este caso, el conjunto de datos recogidos y con los que se ``entrena'' al modelo contienen la observación de nuestra variable objetivo. 

\noindent Dentro de este tipo de métodos tendríamos métodos tan variados como los más simples métodos de regresión lineal multivariante, hasta las más complejas redes neuronales que podamos construir. 

Por otro lado, los métodos no supervisados buscan relaciones entre las variables, de modo que no tenemos en el conjunto de entrenamiento ninguna información de cómo de correcto o incorrecto es lo que estamos afirmando. En este tipo de métodos entraría el apartado del análisis de componentes principales o 


