\noindent Este tipo de métodos se definen de la siguiente manera \cite{Mahesh 2020}. 
\begin{defi}
Se llaman métodos supervisados  aquellos  que buscan inferir una relación estocástica entre dos grupos de variables, predictoras y respuestas en un conjunto de datos que contiene observaciones simultáneas de ambos conjuntos de datos. 
\end{defi}
\noindent Sea un conjunto de variables aleatorias observables de manera simultánea en una población. Estas variables se pueden dividir en dos tipos \cite{Hastie 2001}:
\begin{itemize}
\item Variables de entrada o predictoras: Son las variables independientes que determinarán de manera aleatoria al segundo conjunto de variables. Al conjunto de variables aleatorias de entrada se la denotará como el vector aleatorio $\textbf{x}=[X_1,\ldots X_p]$.

\item Variables de salida o respuesta: Son las variables dependientes de las anteriores. A este conjunto de variables se les denota con el vector aleatorio $\textbf{y}=[Y_1,\ldots Y_K]$, en el caso de que se tenga una única variable respuesta se denotará como $Y$. 
\end{itemize}

\noindent Estas $K+p$ variables son observables en una población, pero no siempre es posible estudiar los valores de estas variables en el conjunto total de la población. Es por ello, que normalmente se utiliza una muestra. 

\begin{defi}
Se llama muestra a un subconjunto de la población. Para que nuestro estudio sea correcto, esta debe ser representativa.
\end{defi}

\noindent De manera habitual se trabajará con muestras aleatorias.  De esta forma se podrán hacer inferencias sobre las estructuras de medidas de la población. 

\begin{defi}
Se llama muestra aleatoria simple de una variable aleatoria con una cierta distribución de probabilidad $F$, a un conjunto de $N$ variables aleatorias independientes con la misma distribución. 
\end{defi} 

\noindent Sea una muestra representativa con $N$ observaciones de las $K+p$ variables. Entonces, se pueden definir los siguientes conceptos \cite{Chatfield 1989}:

\begin{defi}
Se define la matriz de datos $\textbf{X}$ como una matriz de tamaño $N\times p$ que contiene como filas los vectores de longitud $p$ que representan los datos de cada observación. Estas se denotarán a lo largo de la memoria como $\textbf{x}_i,$ donde $i=1,\ldots, N$. Por ejemplo, en el caso de que se recojan datos sobre distintos modelos de coches las observaciones serían los valores medidos de las distintas variables consideradas en cada uno de los coches. 
\end{defi}

\begin{defi}
La matriz de respuestas $\textbf{Y}$ es una matriz de tamaño $N \times K$, donde cada observación de las variables respuesta es una fila. En este trabajo cada observación del vector respuesta se denota como $\mathbf{y}_i$ donde $i=1,\ldots, N$, en el caso de que $K>1$ y como $y_i$ cuando $K=1$. Siguiendo el ejemplo anterior, se podrían tener variables respuesta como el tipo de etiqueta medioambiental siendo esta una variable categórica o el precio del vehículo que tiene naturaleza continua.
\end{defi}

\noindent Por ende, se recogen observaciones simultáneas de las variables de entrada y de salida formando parejas $(\mathbf{y}_i,\textbf{x}_i), i=1\ldots N$, de manera que se obtiene un vector fila de tamaño $p+K$. Estas observaciones de las variables forman una muestra aleatoria de la población .


\noindent En conclusión, el objetivo de estos métodos supervisados es encontrar una relación estocástica que llamaremos predictor, de tal manera que para una nueva observación de las variables de entrada, que denotaremos con $\textbf{x}_0$, se pueda hacer una predicción $\hat{\textbf{y}}_0$ del valor real $\textbf{y}_0$ de la variable respuesta siempre con un cierto error que será una variable  aleatoria. 
 
