\section{Introducción}
\noindent Sea un caso en el que tenemos unas variables aleatoria \textit{inputs} que influyen o determinan otras variables \textit{outputs}. De estas variables, tomamos observaciones conjuntas de manera que a la hora de predecir las variables respuesta a partir de las variables de entrada. Una vez recogidas estas muestras se planteará un modelo \textit{(Hay varias opciones como se desarrollará a lo largo de este capítulo)}.
\noindent Sea \textbf{x} el vector aleatorio de longitud $p$ formado por las variables de entrada, sea a su vez las variables respuesta $Y_k$. De esta manera el objetivo de los métodos supervisados es conseguir una función $f$ que consiga que \begin{equation}
Y=f(\textbf{x})+\varepsilon
\end{equation}
En otras palabras, conocidas las variables de entrada de una nueva observación poder predecir sus variables de respuesta con la mayor precisión posible. 
\subsection{Mínimos cuadrados y K-vecinos más cercanos}
\noindent 

























































































































\newpage