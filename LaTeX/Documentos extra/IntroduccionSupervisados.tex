
\noindent Sea un conjunto de datos obtenidos de observar ciertas variables aleatorias de manera simultánea. De ese conjunto de variables se dividirán en:
\begin{itemize}
\item \textbf{\textit{Variables de entrada o predictoras}}: Son las variables independientes que determinarán de manera aleatoria al segundo conjunto de variables. Al conjunto de variables aleatorias de entrada se la denotará como el vector aleatorio $\textbf{x}=[X_1,\ldots X_p]$. \\
\begin{defi}
Tomando $N$ observaciones de estas variables  de manera simultánea se obtiene la \textit{matriz de datos} $\textbf{X}$. Esta matriz es de tamaño $N\times p$ y contiene como filas los vectores de longitud $p$ que representan los datos de cada observación, se denotarán a lo largo de la memoria como $\textbf{x}_i, i=1\ldots N$. Por ejemplo, en el caso de que se recojan datos sobre distintos modelos de coches las observaciones serían los valores medidas de las distintas variables consideradas en cada uno de los coches. 
\end{defi}
\item \textbf{\textit{Variables de salida }}: Son las variables dependientes de las anteriores. A este conjunto de variables se les denota con el vector aleatorio $\textbf{y}=[Y_1,\ldots Y_K]$, en el caso de que se tenga una única variable respuesta se denotará como $Y$. \\
Como en el caso de las variables de entrada, se tomarán $N$ observaciones de las $K$ variables, dando como resultado la matriz de respuestas $\textbf{Y}$ de tamaño $N \times K$, donde cada observación es una fila y se denota como $\mathbf{y}_i, i=1\ldots K$, en el caso de que $K>1$ y como $y_i$ cuando $K=1$. Siguiendo el ejemplo anterior, se podrían tener variables respuesta como el tipo de etiqueta medioambiental siendo esta una variable discreta o el precio del vehículo que tiene naturaleza continua. 
\end{itemize}
\noindent Por ende, se recogen observaciones simultáneas de las variables de entrada y de salida formando parejas $(\mathbf{y}_i,\textbf{x}_i), i=1\ldots N$, estas observaciones de las variables forman una muestra aleatoria de la población .
\begin{defi}
Se llama \emph{muestra aleatoria} de una variable aleatoria con una cierta distribución de probabilidad $F$, a un conjunto de $N$ variables aleatorias con la misma distribución. 
\end{defi} 

\noindent Durante la memoria, se establecerán los fundamentos a nivel poblacional, solo teniendo en cuenta las propiedades de las variables aleatorias y luego en caso de que sea necesario se tomarán $N$ observaciones o realizaciones del experimento para formar las matrices de datos correspondientes. 

\noindent El objetivo de estos métodos supervisados es encontrar una relación estocástica que llamaremos predictor de tal manera que para una nueva observación de las variables de entrada, que denotaremos con $\textbf{x}_0$ se pueda hacer una predicción $\hat{\textbf{y}}_0$ del valor real $\textbf{y}_0$ de la variable respuesta siempre con un cierto error que será una variable  aleatoria. 
 
