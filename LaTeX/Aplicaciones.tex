\chapter{Aplicación sobre datos}

\noindent A lo largo de este capítulo se  aplicarán algunas de las técnicas desarrolladas anteriormente a ejemplos prácticos con bases de datos reales obtenidas de distintos repositorios. 

\noindent La principal herramienta usada ha sido el lenguaje de programación Python, utilizando librerías como Pandas para el manejo de las bases de datos, Scikit-Learn para la implementación de los modelos  y otras librerias de visualización como Matplotlib para la inclusión de gráficos sencillos. 

\section{Tipo de alubias}
\subsection*{Descripción del DataSet}
\noindent Este conjunto de datos,  \cite{Alubias} está formado por 12 variables predictoras que detallan \emph{Koklu, M. y Ozkan, I.A.}\cite{Koklu 2020}
\subsection*{Objetivos}
\noindent Los objetivos son estudiar la estructura de la matriz de covarianzas, es decir se quiere estudiar cuales son las direcciones de mayor variabilidad y si las variables observadas guardan un factor latente. Es decir, se está utilizando una óptica exploratoria, ya que no queremos confirmar ninguna suposición previa. 

\noindent Además, se busca definir un modelo predictor para poder crear una clasificador de las propias alubias. 

\subsection*{Métodos a utilizar}
\noindent Para estudiar la estructura de los datos, se utilizará el análisis de componentes principales. Por otro lado, para comprobar la existencia de factores latentes se aplicará el análisis de factores principales. Tras aplicar ambas técnicas,
\subsection*{Desarrollo y resultados}
\subsection*{Conclusiones}

