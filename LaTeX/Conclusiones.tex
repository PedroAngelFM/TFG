\chapter{Conclusiones}

\noindent Como resumen, este trabajo ha intentado dar una introducción básica desde la óptica más matemática posible a los principales métodos multivariantes, dentro de las limitaciones dadas de tiempo y espacio. Añadir que  se ha observado que esta rama de estudio tiene un gran potencial sobre todo a la hora de las aplicaciones, tocando casi todas las ramas de conocimiento.

\noindent En particular, se han desarrollado los principales métodos supervisados como los modelos lineales para regresión, así como las principales características inferenciales de los distintos parámetros hallados. En la parte de clasificación se ha descrito el concepto de función discriminante. También  se ha hablado sobre el análisis canónico de poblaciones que permite conocer las direcciones en las que las distintas poblaciones se diferencian de manera más significativa. Además se han mencionado los conceptos básicos de las redes neuronales como el concepto de neurona, capa de neuronas y muchos otros, necesarios para entender las ventajas y desventajas de dichos modelos. Por último,  se ha hablado de los distintos árboles de decisión, tanto los de clasificación como regresión.

\noindent En el apartado de los métodos no  supervisados, se han hablado de las componentes principales, que permiten analizar las direcciones en las que mayor variabilidad se da en los datos. Se ha detallado las capacidades del análisis factorial discernir estructuras latentes en los datos, es decir, si existe una parte común en cada una de las variables. Por último, se ha desarrollado el análisis de \emph{cluster} que busca crear agrupaciones homogéneas de las observaciones o variables, aunque este último aspecto no se ha desarrollado. 

\noindent Para finalizar, se ha podido desarrollar una aplicación utilizando ciertas técnicas multivariantes. Esta aplicación ha estado dedicada a estudiar las propiedades mecánicas de distintas mezclas de cementos. 

 
\noindent En la memoria se ha desarrollado una parte de los métodos multivariantes más importantes. La ampliación de este marco teórico del siendo una línea de investigación a seguir en el futuro. Por ejemplo, en el ámbito de la clasificación se podría hablar sobre la regresión logística, o desarrollar de manera más extensa el concepto de la \emph{backpropagation} o las neuronas \emph{LSTM}, ya que éstas permiten el estudio de las series temporales.  

%\noindent Durante más de un siglo las técnicas multivariantes han sufrido un desarrollo meteórico, desde los artículos de Galton en 1889 o Pearson en 1901 hasta la actualidad que sustentan distintos modelos de \emph{Machine Learning}. Dichos modelos permiten a investigadores de distintas áreas de conocimiento como podría ser la biología, la medicina, la física, y muchos otros llevar a cabo multitud de estudios de manera sencilla y llegar a conclusiones que hace años no hubieran podido. 
%
%\noindent Además, cada vez hay más facilidades para poder aplicar dichas técnicas, el lenguaje Python con las librerías adecuadas permite al usuario final aplicarlas con poco código. Esta facilidad de aplicación hace que cada vez más el interés general hacia los métodos multivariantes sea mayor, todo esto junto a la capacidad de recogida de datos que se tiene actualmente. 




