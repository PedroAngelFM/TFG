\chapter{Métodos no supervisados}


\noindent Los métodos no supervisados son aquellos que tienen como principal objetivo estudiar la estructura de los datos, y extraer la información más importante. De esta manera, es posible reducir la dimensionalidad o llevar a cabo un análisis de manera más intuitiva.

\noindent En esta categoría de técnicas no se puede establecer una función de pérdida, es decir no se da una medida  como podría ser la distancia entre predicciones y valor real de las variables.

\noindent Sin embargo esa necesidad de estudiar las estructuras de los datos más allá de la dependencia funcional de unos grupos de variables con otros como pueden ser la mayoría de métodos de regresión que se han expuesto anteriormente es lo que dan a este tipo de métodos multivariantes la importancia que tienen. 

\noindent En el caso del Análisis de Componentes Principales y el Análisis Factorial, se busca extraer la máxima información de la matriz de covarianzas o de correlaciones para poder dar una simplificación de la estructura de los datos.Sin embargo, en el caso del Análisis de Clusters el objetivo no es simplificar la estructura de los datos si no ver como estos están agrupados y en algunos casos como de homogéneos son. Otros métodos multivariantes no supervisados serían las \emph{Reglas de Asociación} que detallan \emph{Hastie et.al.}\cite{Hastie 2001} en el capítulo 14 que busca asociaciones entre variables. 

\noindent También hay variaciones de las técnicas que se verán en esta memoria, como Curvas o Superficies Principales o las técnicas de escalado multidimensional, que por falta de tiempo no han sido consideradas.  

\noindent  
\section{Análisis de Componentes Principales}
%
\textit{Aquí va la intro de la seccion de las componentes principales.}
\noindent Dado un conjunto de datos, recogidos en una matriz \textbf{X} que llamaremos matriz de datos. Estos datos provienen de tomar observaciones de $p$ variables de manera simultánea. 

\noindent Cada vector fila es una observación $\textbf{x}_i$ que se puede pensar como un elemento del espacio $\mathbb{R}^p$. Cuando el numero de variables a estudiar excede 



%\noindent Sea una población de la que se han tomado $n$ observaciones de las cuales hemos medido $p$ variables. Del conjunto de datos que resulta, podemos dar la matriz de datos $\mathbf{X}$ de tamaño $n\times p$. 
%
%\noindent Una vez tenemos la matriz de datos, tenemos que cada 
%
%El análisis de componentes principales, aunque normalmente se utiliza con el objetivo de representar los datos de manera sencilla, también podría llegar a ser útil en la reducción de la dimensionalidad para evitar el overfitting en el aprendizaje automático. 
%
%Este método es una técnica matemática que dado un vector aleatorio $\textbf{X}^T=[X_1,\ldots X_p]$, con vector de media $\mu$ y matriz de covarianzas $\Sigma$, utiliza transformaciones ortogonales para conseguir las componentes principales. Aunque también es realizable con la media poblacional $\overline{x}$ y la matriz de varianzas \textbf{S} de manera análoga.  
%
%Estas componentes principales son combinaciones lineales de las variables que forman el vector, de manera que estando correladas las iniciales, las componentes no lo están y se busca calcularlas con la máxima varianza posible. 
%
%Esta transformación se busca ya que si varias variables están altamente correladas, entonces están aportando la misma información, siempre que no sean dependientes. Podemos encontrar por esta razón variables que no estén correladas, lo que implica que no dan la misma información y que en consecuencia pueden aportar mayor información acerca de la variación de los datos, sin tener que ser compartida o repetida por varias variables. 
%
%Al finalizar este Análisis de Componentes Principales, obtendremos un conjunto de variables nuevas no correladas entre sí, que son combinación lineal de las iniciales que maximizan la varianza en cada paso.
%
%Añadir que si las variables no están correladas o están cerca de no estarlo, el Análisis de Componentes Principales no tiene sentido ya que el conjunto de componentes principales será parecido a las variables iniciales, con la única diferencia que estarán ordenadas por orden creciente de varianza.  



\subsection{Definición y cálculo de las Componentes}

\noindent Sea un vector aleatorio $\textbf{X}^T=[X_1,\ldots X_p]$ con vector de medias $\mu$ y matriz de covarianzas $\Sigma$.
\begin{defi}
Las componentes principales son combinaciones lineales de las variables $X_1 \ldots X_p$
\begin{equation}
\textbf{Y}_j=a_{1j}X_1+\ldots a_{pj}X_p=\textbf{a}_j^T\textbf{X}\quad 
\end{equation}

\noindent Donde $\textbf{a}_j$ es un vector de constantes y la variable $\textbf{Y}_j$ cumple lo siguiente:
\begin{itemize}
\item Si $j=1$ $Var(\textbf{Y}_1)$ es máxima restringido a $\textbf{a}_1^T \textbf{a}_1=1$
\item Si $j>1$ debe cumplir:
\begin{itemize}
\item $Cov(\textbf{Y}_j,\textbf{Y}_i)=0\quad \forall i<j $
\item $\textbf{a}_j^T \textbf{a}_j=1$
\item $Var(\textbf{Y}_j)$ es máxima. 
\end{itemize}

\end{itemize}

\noindent De esta manera, estamos buscando una nueva base que consiga reunir las direcciones de máxima variación. 
\end{defi}

\noindent El cálculo de la primera componente principal se lleva a cabo con un proceso de optimización de la función $Var(\textbf{Y}_1)$ sujeto a la restricción de que $\textbf{a}_1^T\textbf{a}_1=1$. 

\noindent Aplicando el método de los multiplicadores de Lagrange,  dada una función $f(\textbf{x})=f(x_1,\ldots, x_p)$ diferenciable con una restricción $g(\textbf{x})=g(x_1, \ldots, x_p)=c$ entonces existe una constante $\lambda$ de manera la ecuación:
\begin{equation}
\dfrac{\partial f}{\partial x_i}-\lambda\dfrac{\partial g}{\partial x_i}=0 \quad i=1,\ldots p 
\end{equation}

\noindent Tiene como solución los puntos estacionarios de $f(\textbf{x})$. Entonces, se puede definir la función $L(\textbf{x})= f(\textbf{x})-\lambda[g(\textbf{x})-c]$  que permite simplificar la expresión anterior a:
\begin{equation}
\dfrac{\partial L}{\partial \textbf{x}}=0
\end{equation}

\noindent Para el caso de las componentes principales, la función objetivo es la varianza de la combinación lineal, es decir, $f(\textbf{x})=\textbf{x}^T \Sigma \textbf{x}$ y la restricción aplicada es $g(\textbf{x})=\textbf{x}^T\textbf{x}=1$. 

\noindent Tomando $\textbf{x}=\textbf{a}_1$ se puede establecer $L(\textbf{a}_1)=\textbf{a}_1^T \Sigma \textbf{a}_1 - \lambda[\textbf{a}_1^T \textbf{a}_1-1]$. Que al derivarla se obtiene:
%\noindent Para el cálculo de la $L(\textbf{a}_1)=\textbf{a}_1^T \Sigma \textbf{a}_1 - \lambda[\textbf{a}_1^T \textbf{a}_1-1]$. Al derivarla obtenemos que :
\begin{align*}
\dfrac{\partial L}{\partial \textbf{a}_1} &= 2\Sigma \textbf{a}_1 - 2\lambda\textbf{a}_1\\
& = 2(\Sigma-\lambda)\textbf{a}_1 
\end{align*}

\noindent Igualando a 0 tenemos la siguiente ecuación: 
\begin{equation}
(\Sigma-\lambda I)\textbf{a}_1=0
\end{equation}

\noindent Para que $\textbf{a}_1$ sea un vector no trivial, tenemos que elegir $\lambda$ de tal manera que $|\Sigma-\lambda I| = 0$, es decir, $\lambda$ es un vector propio de la matriz de covarianzas, $\Sigma$. Al ser ésta una matriz semidefinido positiva y simétrica, los valores propios son reales y positivos. Por tanto, $\textbf{a}_1$ es un vector propio de la matriz de covarianza.

\noindent La función a maximizar es $Var(\textbf{Y}_1)=Var(\textbf{a}_1^T\textbf{X})=\textbf{a}_1^T\Sigma \textbf{a}_1=\textbf{a}_1^T\lambda \textbf{a}_1$, y para maximizarla basta tomar $\lambda=\max{\lbrace\lambda_1\ldots \lambda_p\rbrace}$. reordenando si es necesario, se tiene que $\lambda=\lambda_1$ 

%\noindent Para que la ecuación tenga una solución que no sea la trivial, tenemos que elegir $\lambda$ de manera que $|\Sigma-\lambda I| = 0$. Luego $\lambda$ es uno de los valores propios de la matriz de covarianzas. Generalmente una matriz $(p\times p)$ tiene $p$ valores propios $\lbrace\lambda_1, \ldots ,\lambda_p \rbrace$ y como $Var(\textbf{Y}_1)=Var(\textbf{a}_1^T\textbf{X})= \textbf{a}_1^T \Sigma \textbf{a}_1 =\textbf{a}_1^T \lambda \textbf{a}_1=\lambda$ que es la variable a maximizar, elegimos $\lambda=\max \lbrace\lambda_1, \ldots ,\lambda_p \rbrace$, por tanto, el vector $\textbf{a}_1$ es el vector propio con valor propio $\lambda=\lambda_1$ reordenando si es necesario.



\noindent Una vez calculada la primera componente principal $\textbf{Y}_1$, la segunda componente se calcula de manera análoga, maximizando $Var(\textbf{Y}_2)=Var(\textbf{a}_2^T\textbf{X})$ condicionada por $\textbf{a}_2^T\textbf{a}_2=1$. A esta restricción tenemos que añadir la restricción $Cov(\textbf{Y}_1,\textbf{Y}_2)=0 $

\begin{propo}
La condición $Cov(\textbf{Y}_1,\textbf{Y}_2)=0 $ equivale a la condición $\textbf{a}_2^T\textbf{a}_1 = 0$.
\begin{proof}
Utilizando que $\textbf{Y}_j=\textbf{a}_j^T \textbf{X}\quad \forall j$ , tenemos entonces que:
\begin{align*}
Cov(\textbf{Y}_2,\textbf{Y}_1)&= Cov (\textbf{a}_2^T\textbf{X},\textbf{a}_1^T\textbf{X})\\ 
&= \mathbb{E}(\textbf{a}_2^T(\textbf{X}-\mu)(\textbf{X}-\mu)^T \textbf{a}_1)\\
&= \textbf{a}_2^T \mathbb{E}((\textbf{X}-\mu)(\textbf{X}-\mu)^T) \textbf{a}_1\\
&= \textbf{a}_2^T \Sigma \textbf{a}_1 \\
&= \textbf{a}_2^T \lambda_1 \textbf{a}_1
\end{align*}
\noindent De manera que, si $a_2^T \lambda_1 a_1 = 0 \Rightarrow a_2^T a_1=0 $, luego son vectores ortogonales entre sí.
\end{proof}
\end{propo}


\noindent \emph{Observación: } Esta proposición se puede extender de manera simple al caso de tener que calcular la $i$-ésima componente principal habiendo calculado las anteriores de las cuales se sepan los valores propios asociados. 

\begin{coro}
Las componentes principales son todas ortogonales entre sí. 
\end{coro}
%%%ESto está maaaal
\noindent Tomando la matriz formada por los $p$ vectores propios como columnas tenemos la matriz ortogonal $\textbf{A}=[\textbf{a}_1,\ldots, \textbf{a}_p]$, de manera que el vector aleatorio 
$$\textbf{Y}=[Y_1,\ldots , Y_p]^T=\textbf{A}\textbf{X}$$

\noindent Se deduce utilizando la ortogonalidad de $\textbf{A}\Rightarrow \textbf{A}^T\textbf{A}=\textbf{A}\textbf{A}^T=\textbf{I}$ que:
\begin{align*}
Var(\textbf{Y})&=Var(\textbf{AX})\\
&=\mathbb{E}((\textbf{AX})^T(\textbf{AX}))\\
&=\mathbb{E}(\textbf{X}^T\textbf{A}^T\textbf{A}\textbf{X})\\
&=\mathbb{E}(\textbf{X}^T\textbf{X})\\
&=Var(\textbf{X})
\end{align*} 

\noindent Por tanto, se puede afirmar que las componentes principales contienen toda la variación del vector \textbf{X} aleatorio inicial.


\begin{propo}
Para cualquier $1\leq q \leq p $ entero sea la matriz ortonormal $\textbf{B}$, entonces la transformación:
\begin{equation}

\end{equation}
\end{propo}

\noindent Sea ahora la matriz \textbf{X} de datos de tamaño $n \times p$ resultante de tomar $n$ observaciones de las $p$ variables del vector aleatorio. Se supone además que esta es centrada y dividida todos sus elementos por $\sqrt{n-1}$. 

\noindent Dado este marco la matriz \textbf{X} tiene una descomposición en valores singulares $\textbf{X}=\textbf{U}\Sigma \textbf{V}^T$. Donde tanto \textbf{U} como $\textbf{V}$ son matrices ortogonales. Esto implica que $\textbf{U}^T \textbf{U}=\textbf{U}\textbf{U}^T=\textbf{I}$. 

\noindent  En particular, \textbf{U} es la matriz de vectores singulares por la izquierda  de tamaño $n \times r$, \textbf{V} la matriz de vectores singulares por la derecha de tamaño $p \times p$ y la matriz $\Sigma$ es la matriz diagonal de valores singulares generalmente ordenados de forma decreciente, es decir $\Sigma^2$ es la matriz diagonal que contiene todos los valores propios de las matrices $\textbf{X}^T\textbf{X}$ y $\textbf{X}\textbf{X}^T$.

\noindent Por las transformaciones previas realizadas la matriz de estimaciones de las covarianzas es $\hat{S}=\textbf{X}^T\textbf{X}$ en consecuencia, el proceso que se describía en primera instancia para un vector aleatorio en el que conocíamos su matriz de covarianzas se puede aplicar de manera análoga a la matriz de datos.

\noindent Esto implica que la obtención de las componentes principales en el caso de ser calculadas desde la matriz de datos también se obtienen calculando los vectores y valores propios de la matriz de covarianzas muestral en este caso. 

\begin{propo}
La matriz \textbf{V} de tamaño $(p\times p)$ es la matriz que contiene los vectores para hacer la combinación lineal que definen las componentes principales.
\begin{proof}
La matriz $\textbf{X}^T \textbf{X}$ es la matriz de covarianzas $(p \times p)$  por la descomposición en valores singulares tenemos que:
\begin{align*}
\textbf{X}^T \textbf{X} &= (\textbf{U}\Sigma \textbf{V}^T)^T (\textbf{U}\Sigma \textbf{V}^T)\\
&= \textbf{V}\Sigma ^T \textbf{U}^T \textbf{U}\Sigma \textbf{V}^T\\
&= \textbf{V}\Sigma ^T \Sigma \textbf{V}^T \\
&=  \textbf{V}\Sigma ^2  \textbf{V}^T 
\end{align*}

\noindent Que es la diagonalización de la matriz $\textbf{X}^T \textbf{X}$, ya que $\textbf{V}^T=\textbf{V}^{-1}$, donde los elementos de la diagonal de $\Sigma$ son las raíces cuadradas de los valores propios de $\textbf{X}^T \textbf{X}$. A esto también hay que añadir que \textbf{V} es la matriz que tiene como columnas los vectores propios, es decir, es la matriz con los vectores que forman las componentes principales. 
\end{proof}
\end{propo}

\noindent Teniendo en cuenta lo anterior podemos obtener la transformación de la matriz de datos a las componentes principales,\textbf{Y}, multiplicando por la matriz $\textbf{V}$, lo que nos resulta en :
\begin{equation}
\textbf{XV}=\textbf{U}\Sigma \textbf{V}^T \textbf{V}=\textbf{U}\Sigma=\textbf{Y}
\end{equation}
\subsection{Reducción de la dimensionalidad}

\noindent La utilidad de esta técnica es reducir el tamaño de la matriz de datos para poder representar las observaciones como puntos de una subvariedad de dimensión $m < p$. Con este fin en mente, se deben definir los siguientes conceptos.
\begin{defi}
Sea $\textbf{A}\in \mathbb{M}_{n\times m}(\mathbb{R})$ definimos la \textit{norma de Frobenius} de la matriz \textbf{A} como :
\begin{equation}
||\textbf{A}||_F=(tr(\textbf{A}^T\cdot \textbf{A}))^{\frac{1}{2}}=\left(\sum_{i=1}^{n}\sum _{j=1}^{m}a_{ij}^2\right)^{\frac{1}{2}}
\end{equation}
\end{defi}

\begin{propo}
La norma de Frobenius es invariante a transformaciones ortogonales
\begin{proof}
Sea $\mathbf{U}$ una matriz ortogonal, que cumple $\mathbf{U}^T\cdot \mathbf{U}=\mathbf{U}\cdot \mathbf{U}^T=I$, sea una matriz cualquiera $\mathbf{A}$, entonces:
\begin{align*}\tag{2.7}
||\mathbf{U} \cdot \mathbf{A}||_F^2&=tr((\mathbf{U} \mathbf{A})^T\cdot(\mathbf{U} \mathbf{A}))\\
&=tr((\mathbf{A}^T \mathbf{U}^T)\cdot \mathbf{U} \mathbf{A}))\\
&=tr(\mathbf{A}^T \mathbf{A})\\
&=||\mathbf{A}||_F^2
\end{align*}
\end{proof}
\end{propo}

\noindent Aunque hay más normas que son invariantes ante transformaciones ortogonales, nos centraremos en la norma de Frobenius.

\begin{defi}
Dada una matriz $\textbf{A}=\textbf{U}\Sigma \textbf{V}^T$ de tamaño $n\times m $ y de rango $r$, entonces para $l < r$ se define la matriz truncada  $\textbf{A}_l=\textbf{U}_l \Sigma_l \textbf{V}^T_l$ donde $\textbf{U}_l,\textbf{V}_l,\Sigma_l$, son las matrices resultantes de seleccionar las $l$ primeras columnas . 
\end{defi}


\begin{teorema}[De Eckart-Young]
Sea una matriz $\textbf{A}$ de rango $r$, y sea una matriz $\textbf{B}$ rango $l<r$, entonces:
\begin{equation}
||\textbf{A}-\textbf{B}||_F \leq ||\textbf{A}-\textbf{A}_l||_F
\end{equation}
\end{teorema}






\newpage
\newpage
\section{Análisis Factorial}
\noindent El análisis factorial es un conjunto de técnicas que en primera instancia fue desarrollado en el campo de la psicología ya que estos buscaban encontrar las causas que provocaban la variabilidad de los datos que recogían \cite{Vincent 1953}. 

\noindent En 1888 \emph{Galton, F} escuchó la exposición de un articulo que relacionaba las sociedades que habían abandonado un modelo matriarcal con una estructura más compleja sociológicamente hablando. Ante esto, \emph{Galton} no estuvo de acuerdo, ya que detectó un fallo en el razonamiento de su compañero y es por ello, que escribió el artículo \cite{Galton 1889} dando pie a los trabajos sobre el estudio de las correlaciones.   

\noindent Es en 1904 cuando \emph{Spearman, C.} \cite{Spearman 1904} viendo el trabajo de \emph{Galton, F.} aplicó esas ideas al estudio de la inteligencia, buscando una medida objetiva de la misma. En este desarrollo, dio las primeras nociones de qué era un factor y calculó una aproximación para un modelo factorial con 6 factores comunes. A pesar de que no especifica el modelo usado, se ha comprobado a posteriori  que eran bastante precisos. A partir de este momento, fue cuando el análisis factorial empezó a ser importante y varios psicólogos comenzaron a usarlo de forma habitual. 
Desde hace años el Análisis factorial, al igual que otras técnicas multivariantes, ha experimentado un nuevo impulso en el desarrollo teórico y práctico debido al auge de la computación. 

\noindent El Análisis factorial nació con un caracter confirmatorio, en el que se tenían unas ciertas ideas preconcebidas y se buscaba aceptar o rechazar dichas ideas. Actualmente, la mayoría de los investigadores lo utilizan de manera exploratoria para extraer información de los datos.\cite{Hair 1995}

\noindent El objetivo específico del Análisis factorial es, dado un conjunto de variables aleatorias, construir un modelo en el cual con un conjunto de nuevas variables llamadas \emph{factores comunes}, que expliquen la mayor variabilidad posible de las variables iniciales. De esta manera, se descompondrá la varianza de cada una de las variables en una parte común llamada \emph{comunalidad}  y otra específica denominada \emph{especificidad}. 

\noindent La mayor diferencia con el análisis de componentes principales en el que se busca nuevas direcciones que maximicen la varianza explicada, es que en este caso, son de interés tanto las comunalidades como las especifidades ya que nos permiten analizar si una variable da información de manera común al resto o tiene algo diferencial. 

\noindent Para empezar, empezaremos formalizando el modelo factorial, en particular el modelo multifactorial ortogonal \cite{Johnson 2007} sin pérdida de generalidad. Tras la modelización, se desarrolla el método de estimación de las matrices de carga mediante el método del factor principal \cite{Peña 2002} y se mencionarán los fundamentos de otros métodos como el de  máxima verosimilitud. Seguidamente, se darán algunos detalles sobre el problema de la no unicidad del modelo y soluciones ante ello. Y por último, se dará una introducción a la interpretación de los datos que arroja el modelo factorial. 

\newpage
\subsection{Formalización}
\noindent Supóngase un vector aleatorio $\mathbf{x}$ de longitud $p$ con una distribución $N_p(0,\Sigma)$, centrada sin pérdida de generalidad, donde la matriz de covarianzas $\mathbf{\Sigma}$ es simétrica y definido positiva. El modelo factorial afirma lo siguiente \cite{Chatfield 1989}:  
\begin{equation}\label{eq Fact}
X_j= \lambda_{1j}F_1+\ldots+\lambda_{mj}F_m+\psi_j U_j\quad j=1\ldots p 
\end{equation}
\noindent Donde:
\begin{itemize}
\item $\lambda_{jk}$ son las \emph{cargas del factor común} $k$-ésimo de la $j$-ésima variable.
\item $F_k$ es el $k$-ésimo factor común
\item $\psi_j$ es la \emph{carga específica} de la variable $X_j$
\item $U_j$ es el \emph{factor específico} para la variable $X_j$
\end{itemize}

\noindent En este caso haremos las siguientes suposiciones\cite{Cuadras 2014}:
\begin{itemize}
\item Los factores comunes $F_k$ son variables aleatorias que siguen una distribución marginal $N(0,1)$. Además se supondrá que $Cov(F_k,F_{k'})=0, k\neq k'$, de manera que el vector aleatorio $\mathbf{f}$ de longitud $m$ con distribución $N_m(0,\mathbf{I})$. Y son completamente independientes de los factores específicos. 

\item Los factores específicos $U_j$ son variables aleatorias con una distribución normal $N(0,1)$ no correladas, que conjuntamente forman el vector aleatorio $\mathbf{u}\sim N_p(0,\mathbf{I})$ de longitud $p$. Son completamente independientes de los factores comunes. 
\end{itemize}

\noindent En consecuencia, se pueden definir las siguientes matrices: 

\begin{defi}
Llamaremos \emph{matriz de cargas de los factores o matriz de cargas}, $\mathbf{\Lambda}$ de tamaño $p \times m$ a la matriz :
\begin{equation}
\mathbf{\Lambda}=\begin{pmatrix}
\lambda_{11} & \cdots & \lambda_{1 m}\\
\vdots & \ddots & \vdots\\
\lambda_{p1} & \cdots & \lambda_{pm}
\end{pmatrix}
\end{equation}
\end{defi}

\begin{defi}
Llamaremos matriz específica a la matriz diagonal $\mathbf{\Psi}$ de tamaño $p\times p$ a aquella que tiene los términos $\psi_j$ donde $j=1,\ldots , p$:
\begin{equation}
\mathbf{\Psi}=\begin{pmatrix}
    \psi_1 & 0 & \dots & 0 \\
    0 & \psi_2 & \dots & 0 \\
    \vdots & \vdots & \ddots & \vdots \\
    0 & 0 & \dots & \psi_p
\end{pmatrix}
\end{equation}
\end{defi}

\noindent \emph{Observación:} En distintas fuentes no aparece la matriz $\mathbf{\Psi}$ ya que los factores específicos $U_j, j=1\ldots p$ en esos casos no se consideran variables estandarizadas no correladas, como sería el caso descrito. Por ejemplo,  \emph{Mardia, K.V.} \cite{Mardia 1979} denotan como $\mathbf{\Psi}$ a la matriz de covarianzas del vector $\mathbf{u}$.

\noindent De esta manera, se puede dar una versión matricial de la expresión \ref{eq Fact}:
\begin{equation}
\mathbf{x}^T=\Lambda \mathbf{f}^T+ \mathbf{\Psi}\mathbf{u}^T
\end{equation}
Ya que se consideran los vectores aleatorios $\mathbf{x}, \mathbf{f}, \mathbf{u}$ vectores ordenados como filas. 

\begin{propo}
La varianza de la variable aleatoria $X_j$, $\sigma_j^2$ se puede descomponer de la siguiente manera:
\begin{equation}
\sigma_j^2 = \sum_{k=1}^{m}\lambda_{kj}^2+\psi_j^2
\end{equation}
\begin{proof}
\begin{align*}
\mathbb{E}(X_j^2)&=\sigma_j^2 = \sum_{k=1}^{m}\lambda_{kj}^2\mathbb{E}(F_k
^2)+\sum_{k=1}^{m}\sum_{n=1,n\neq k}^{m} \lambda_{kj}\lambda_{nj}\mathbb{E}(F_k\cdot F_n)\\
&+\sum_{k=1}^{m}\lambda_{kj}\psi_j \mathbb{E}(F_k\cdot U_j)+\psi_j^2\mathbb{E}(U_j^2)
\intertext{Utilizando las hipótesis del modelo:}
&\ \sigma_j^2=\sum_{k=1}^{m}\lambda_{kj}^2+\psi_j^2
\end{align*}
\end{proof}
\end{propo}
\noindent Teniendo en cuenta esta descomposición se puede definir el siguiente concepto.
\begin{defi}
Se llama \emph{comunalidad} de la variable $X_j$, $h_j^2=\sum_{k=1}^m \lambda_{kj}^2 $ \cite{Peña 2002}
\end{defi} 
\noindent Esto permite interpretar la varianza de la variable como la varianza explicada por los factores comunes por un lado y la variabilidad específica de la propia variable. 
\begin{propo}
La covarianza de dos variables $Cov(X_j, X_j')$, $\sigma_{jj'}$ cumple que \cite{Morrison 1976, Chatfield 1989}: 
\begin{equation}
\sigma_{jj'}=\sigma_{j'j}=\sum_{k=1}^{m}\lambda_{kj}\lambda_{kj'}
\end{equation}
\begin{proof}
\begin{align}
\sigma_{jj'} = \sum_{k=1}^{m}\lambda_{kj}\lambda_{kj'}\mathbb{E}(F_k
^2)&+\sum_{k=1}^{m}\sum_{n=1,n\neq k}^{m} \lambda_{kj}\lambda_{nj'}\mathbb{E}(F_k\cdot F_n)\\
+ \sum_{k=1}^{m}\lambda_{kj}\psi_{j'} \mathbb{E}(F_k\cdot U_{j'})&+\sum_{k=1}^{m}\lambda_{kj'}\psi_j\mathbb{E}(F_k\cdot U_j)+\psi_j\psi_{j'}\mathbb{E}(U_j\cdot U_{j'})
\intertext{De nuevo, teniendo en cuenta las hipótesis}
\sigma_{jj'}=\sigma_{j'j}&=\sum_{k=1}^{m}\lambda_{kj}\lambda_{kj'}
\end{align}
\end{proof}
\end{propo}

\noindent Y tomando estas dos proposiciones se puede tener que la matriz $\mathbf{\Sigma}$, cumple lo siguiente:
\begin{coro}\label{Descomposición Varianza}
La matriz $\mathbf{\Sigma}=\mathbf{\Lambda}\mathbf{\Lambda}^T+\mathbf{\Psi}$
\end{coro}

\noindent Bajo esta idea en la que la varianza se puede descomponer en una parte específica y una común a todas las variables. 

\subsection{Estimación de la matriz de cargas}
\noindent Supóngase el vector aleatorio $x$ como antes del cual se recogen $N$ observaciones para obtener la matriz de datos $\mathbf{X}$ de tamaño $N\times p $que se supondrá centrada sin pérdida de generalidad. Esto provoca que la matriz $\mathbf{S}=\mathbf{\hat{\Lambda}}\mathbf{\hat{\Lambda}}^T+\mathbf{\hat{\Psi}}$ \cite{Peña 2002}. 

\noindent Entonces el objetivo es estimar la matriz de cargas $\mathbf{\Lambda}$, para ello se puede tomar la matriz $\mathbf{S}-\mathbf{\hat{\Psi}}$, si de esta matriz se obtiene la descomposición espectral:
\begin{equation}
\mathbf{S}-\mathbf{\hat{\Psi}}=\mathbf{U}\mathbf{D}\mathbf{U}^T
\end{equation}

\noindent Donde la matriz $\mathbf{D}$ es una matriz diagonal de tamaño $r\times r$ con los $r=rg(\mathbf{X})$ valores propios no nulos, y la matriz $\mathbf{U}$ de tamaño $p\times r$ de manera las columnas son los vectores propios asociados a los valores propios no nulos. 

\noindent De esta manera, \emph{Cuadras C.M;} \cite{Cuadras 2014} toma como estimador de $\mathbf{\hat{\Lambda}}=\mathbf{UD}^{\frac{1}{2}}$, ya que $\mathbf{\hat{\Lambda}}\mathbf{\hat{\Lambda}}^T=\mathbf{U}\mathbf{D}\mathbf{U}^T= \mathbf{S-\hat{\Psi}}$. El problema que surge aquí es que la estimación $\mathbf{\hat{\Psi}}$ no es conocido, pero se desarrollará en lo siguiente. 

\noindent De esta manera, supóngase que es conocido el número de factores, $m$. Por tanto, como el objetivo es que los factores comunes es maximizar su varianza común explicada, entonces siguiendo el mismo razonamiento que en las componentes principales, se puede tomar como estimación de las matrices factoriales la matriz siguiente:

\begin{equation}
\mathbf{\hat{\Lambda}}=(\sqrt{\mu_1}e_1,\ldots,\sqrt{\mu_m}e_m)=\mathbf{U}_m\mathbf{D}_m^{\frac{1}{2}}
\end{equation}

\noindent Donde la pareja $(\mu_i, e_i)$ es el vector propio $e_i$ en formato columna y $\mu_i$ su valor propio asociado. Además la matriz $\mathbf{U}_m$ es la matriz que tiene como columnas los $m$ vectores propios de la matriz $\mathbf{S}-\mathbf{\Psi}$ que tienen asociados los $m$ valores propios más grandes. 

\noindent El problema de que no se conoce $\mathbf{\hat{\Psi}}$, se puede resolver planteando el siguiente método iterativo, para ello se introduce la notación:
\begin{itemize}
\item $\mathbf{S}_i^*=\mathbf{S}-\hat{\mathbf{\Psi}}_i$ es la matriz reducida en la $i$-ésima iteración del método
\item $\mathbf{\hat{\Lambda}}_i$ es la matriz de cargas obtenida en la $i$-ésima iteración del método. 
\end{itemize}
\noindent El método se inicializa con $\mathbf{\hat{\Psi}}_0=0$ de tal manera que $\mathbf{S}_0^*=\mathbf{S}$ y consiste en cada iteración hacer las siguientes operaciones \cite{Johnson 2007, Peña 2002, Cuadras 2014}: 
\begin{enumerate}
\item Se calcula $\mathbf{S}_i^*=\mathbf{S-\hat{\Psi}}_i$
\item Se calcula la descomposición $\mathbf{\hat{\Lambda}}_i$ como se ha detallado antes obteniendo $\mathbf{U}_i$ y $\mathbf{D}_i$ de tamaños $p\times m $ y $m\times m$. 
\item Se obtiene la nueva estimación de la $\mathbf{\hat{\Psi}}_{i+1}=\mathbf{S}-\mathbf{\hat{\Lambda}}_i\mathbf{\hat{\Lambda}}_i^T$
\end{enumerate}

\noindent El proceso se para cuando no haya cambios en las matrices $\mathbf{\hat{\Psi}}_i,\mathbf{\hat{\Psi}}_{i+1}$ o cuando $\mathbf{\hat{\Lambda}}\mathbf{\hat{\Lambda}}^T+\mathbf{\Psi}$ sea lo más parecido posible a la matriz de covarianzas $\mathbf{S}$.

\noindent \emph{Peña D} \cite{Peña 2002} lo generaliza utilizando la función $F=tr(\mathbf{S}-\mathbf{\Lambda}\mathbf{\Lambda}^T-\Psi)^2$. Que en esencia es minimizar la distancia de Frobenius, implicando un razonamiento desde la optimización, ya que se busca minimizar  una función de pérdida, la distancia entre ambas matrices en este caso. 

\noindent Una vez estimadas las cargas de los factores, se pueden estimar las \emph{puntuaciones de los factores}, que son los valores que toman los factores pero no se detallarán en esta memoria.  \emph{(Para más detalles de esta parte véase  la sección 9.5 de \cite{Johnson 2007} o la 12.7 de \cite{Peña 2002})}

\subsection{Unicidad del modelo}

\noindent Un detalle en el que no se ha profundizado, es que el modelo factorial, no es único, es decir, teniendo en cuenta las suposiciones hechas anteriormente, todas las rotaciones de los factores $\mathbf{Tf}^T$, donde $\mathbf{T}$ es una matriz ortogonal de tamaño $m\times m$, entonces al transformar la matriz $\mathbf{\Lambda}$ a la matriz $\mathbf{\Lambda T}^T$ se cumple que \cite{Mardia 1979}:
\begin{equation}
\mathbf{x}^T=(\mathbf{\Lambda T}^T)(\mathbf{Tf}^T)+\Psi\mathbf{u}^T
\end{equation}
\noindent Es equivalente al modelo \ref{eq Fact}, ya que se cumplen las mismas suposiciones. 

\noindent Para evitar estas indeterminaciones se suelen tomar dos restricciones \cite{Mardia 1979}:
\begin{equation}
\mathbf{\Lambda}^T \mathbf{\Psi}^{-1}\mathbf{\Lambda} \text{ ó } \mathbf{\Lambda}^T \mathbf{\Lambda} \text{ son diagonales }
\end{equation}
\noindent En particular, la segunda restricción es la que se toma en el método del factor principal, ya que $\mathbf{\hat{\Lambda}}=\mathbf{UD}^{\frac{1}{2}}$, donde $\mathbf{U}$, es una matriz ortogonal, por tanto, $\mathbf{\hat{\Lambda}^T \hat{\Lambda}}=\mathbf{D}^{\frac{1}{2}}\mathbf{U}^T\mathbf{U}\mathbf{D}^{\frac{1}{2}}=\mathbf{D}$. 

\begin{propo}
La segunda restricción hace que el modelo sea único \cite{Peña 2002}. 
\begin{proof}
Sea la descomposición de la varianza \emph{(Corolario \ref{Descomposición Varianza})}: \begin{align*}
\mathbf{\Sigma}&=\mathbf{\Lambda\Lambda}^T+\mathbf{\Psi}\\
\mathbf{\Sigma}-\mathbf{\Psi}&=\mathbf{\Lambda\Lambda}^T\\
\intertext{Si postmultiplicamos por $\mathbf{\Lambda}$ a ambos lados:}
(\mathbf{\Sigma}-\mathbf{\Psi})\mathbf{\Lambda}&=\mathbf{\Lambda\Lambda}^T \mathbf{\Lambda}\\
\intertext{ Como se supone que $\mathbf{\Lambda}^T \mathbf{\Lambda}=\mathbf{D}$ se tiene que:}
(\mathbf{\Sigma}-\mathbf{\Psi})\mathbf{\Lambda}&=\mathbf{\Lambda D}
\end{align*}
Por tanto, la matriz $\mathbf{\Lambda}$ tiene por columnas los $m$ vectores propios de la matriz $\mathbf{\Sigma-\Psi}$, como se ha señalado anteriormente, esta propiedad ha sido usada anteriormente en el método del factor principal, es decir, la estimación por este método es única. 
\end{proof}
\end{propo}
\subsection{Interpretación del modelo factorial}

\noindent Para interpretar el modelo factorial hay que tener en cuenta los siguientes aspectos:
\begin{itemize}
\item \emph{Proporción de varianza común explicada} Es decir, analizar que parte de la varianza se puede explicar por los factores comunes. A mayor sea, más relación entre las variables explican los factores, 
\item \emph{Especificidad de cada una de las variables}: De esta manera, se puede analizar si una variable está alejada de las demás y no tiene tanto que ver con el resto. 
\item \emph{Cargas de cada uno de los factores en cada variable} Esto también nos puede dar datos, por ejemplo si un grupo de variables tiene cargas mayores en uno de los factores $F_i$ y en otro grupo es más prevalente otro factor, eso también es significativo. 
\end{itemize}

\noindent \emph{Johnson y Wichern } analizan en el capitulo 9 de \cite{Johnson 2007} un ejemplo con un conjunto de datos de varias medidas de aves. 

\noindent Como medidas observables tienen las medidas de 6 huesos y aplican el modelo factorial con 3 factores. Si nos centramos en los datos obtenidos aplicando el método del factor principal \cite{Peña 2002, Johnson 2007}, lo primero a destacar es que 3 factores consiguen acumular un 95\% de la varianza, por lo tanto, es posible afirmar que hay 3 variables latentes, a las que luego daremos interpretación. Por otro lado, la varianza específica es poca o nula en todas las variables, por lo tanto, nos reafirma en nuestra sospecha de que hay 3 variables latentes. 

\noindent Por último, podemos ver que las cargas que tiene en todas el primer factor son cercanas a 1 por tanto tiene gran importancia en todas. Sin embargo el 3er factor en las ultimas 4 variables no tiene esa misma importancia. Por tanto el 3er factor casi no influye en las últimas medidas. 

\noindent \emph{Johnson, R.A. y Wichern, D.W.}\cite{Johnson 2007} además utilizan el método de máxima verosimilitud y utilizan rotaciones para mejorar la interpretación. 


\newpage 
\chapter{Análisis de Conglomerados o Clústers}
\section{Introducción}

Sea un conjunto de observaciones $x_i$ con $i=1,\ldots N$. El objetivo es encontrar agrupaciones de los datos. Las observaciones dentro de estos conglomerados deben tener mayor similitud entre ellas que con cualquier observación que no pertenezca a la agrupación. 

El análisis de clusters tiene 


\newpage
