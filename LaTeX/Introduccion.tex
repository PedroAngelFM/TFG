\chapter*{Introducción}
\pagenumbering{arabic}
\noindent El Análisis Multivariante es, según \textit{Martínez Arias, R.}\cite{Rosario 1999}, el conjunto de técnicas y métodos que busca describir y extraer información de las relaciones entre variables medidas en una o varias muestras u observaciones. Dentro de esta definición, entran todos los procedimientos que analizan de manera simultánea más de una variable. 
Los métodos multivariantes se pueden clasificar según el objetivo a perseguir:
\begin{itemize}
\item \textbf{\textit{Reducción de datos}}: En estos procedimientos se busca analizar la estructura de los datos para buscar una simplificación de la misma. Dentro de este tipo de técnicas están el \textit{Análisis Factorial}, \textit{Análisis de Componentes Principales,} \textit{Análisis Discriminante} etc. 
\item \textit{\textbf{Clasificación y Agrupación:}} Es decir, buscar modos de clasificar las observaciones en grupos homogéneos ya sea de acuerdo a una variable categórica conocida o buscando esa homogeneidad dentro de los datos. Podemos incluir aquí el \emph{Análisis de Conglomerados}
\item \textit{\textbf{Análisis de Relaciones y Dependencias. }}
\item \textbf{\textit{Construcción de Modelos y pruebas de hipótesis}}
\end{itemize}

\noindent También se pueden clasificar  según el conocimiento de los datos o de la variable a analizar como métodos supervisados o no supervisados. 

\noindent En esta memoria se revisarán métodos de todas las tipologías, tanto supervisados como no supervisados. Primero con un acercamiento teórico, otorgando una idea de la base sobre los que se construyen y luego se aplicarán sobre distintos ejemplos desarrollados en Python con distintas librerías especializadas en Análisis Multivariante como TensorFlow o Scikit-Learn. 

\noindent Estas técnicas han visto un auge en los últimos años, debido a su utilidad en el análisis de grandes bases de datos y la creciente capacidad de recogida de datos que se tiene en la actualidad. Además, el llevar a buen término el estudio anteriormente era complejo debido a la gran cantidad de cálculos que se necesitan realizar. Actualmente esos cálculos son automatizados, en algunos casos, con una única línea de código es posible llevar a cabo ciertas técnicas de las descritas.

\noindent Además, ciertos autores como \textit{Hastie T., Tibshirani R. y Friedman J.} \cite{Hastie 2001} enfocan también estos problemas desde el punto del aprendizaje automático, ya sea también supervisado o no.