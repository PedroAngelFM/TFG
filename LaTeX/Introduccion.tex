\chapter*{Introducción}
\pagenumbering{arabic}
\noindent El Análisis Multivariante es, según \textit{Martínez Arias, R.}\cite{Rosario 1999}, el conjunto de técnicas y métodos que busca describir y extraer información de las relaciones entre variables medidas en una o varias muestras u observaciones. Dentro de esta definición, entran todos los procedimientos que analizan de manera simultánea más de una variable. 

\noindent La clasificación que se dará en esta memoria de los métodos multivariantes será la siguiente 
\begin{itemize}
\item \textbf{\textit{Métodos supervisados:}} Son aquellos que explora la relación estocástica entre varias variables, divididas en variables respuesta y variables predictoras, estas técnicas a su vez se pueden dividir según la naturaleza de las variables respuesta \cite{James 2013}:
\begin{itemize}
\item \textit{Regresión: } Cuando las variables de respuesta son continuas, dentro de esta tipología, entran métodos como las  \emph{Redes Neuronales\cite{Mamidi 2021}, los Árboles  de Regresión \cite{Nerini 2007} }etc...
\item \textit{Clasificación: }Cuando las variables de respuesta son discretas, como por ejemplo, el \emph{Análisis discriminante \cite{Diez 2002}, Regresión logística \cite{Ensum 2005}, }
\end{itemize}
\item \textbf{\textit{Métodos no supervisados: }} Los métodos no supervisados son aquellos que buscan analizar la estructura y las relaciones entre las distintas variables \cite{Hastie 2001}. Las distintas técnicas se distinguen según la medida en la que se centren, por ejemplo la variabilidad común \cite{Pages 2005}, la homogeneidad de grupos \cite{Okazaki 2006} etc...
\end{itemize}

\noindent Estas técnicas han visto un auge en los últimos años, debido a su utilidad en el análisis de grandes bases de datos y la creciente capacidad de recogida de datos que se tiene en la actualidad. Basta con ver que la mayoría de artículos recopilados en los que se aplican estas técnicas se han publicado en las dos primeras décadas de siglo. Además, el llevar a buen término el estudio anteriormente era complejo debido a la gran cantidad de cálculos que se necesitaban realizar. Actualmente esos cálculos son automatizados, en algunos casos, con una única línea de código es posible llevar a cabo ciertas técnicas de las descritas.

\noindent Muchas veces, el uso de estas técnicas se combinan con el aprendizaje automático como proponen \textit{Hastie T., Tibshirani R. y Friedman J.} \cite{Hastie 2001} o  \emph{James G, Witten D, Hastie T, y  Tibshirani R.} \cite{James 2013}. De esta manera, los parámetros se estiman extrayendo la información de los propios datos, comparando distintos modelos etc... Utilizando esta aproximación se pueden generar  Este aspecto no se desarrollará en esta memoria.  

\noindent La estructura de esta memoria se divide en tres partes, una dedicada a los métodos supervisados en la que se detallan los más importantes, como la regresión lineal, el análisis discriminante, los árboles de decisión  y las redes neuronales. En la segunda parte se desarrollan los métodos no supervisados con técnicas como el análisis de componentes principales, el análisis factorial y el análisis de conglomerados. Por último, se realizarán aplicaciones en las que se desarrollarán las interpretaciones de las tecnicas descritas durante la memoria. 

\noindent En resumen, el principal objetivo de este trabajo es describir las principales técnicas de análisis de multivariante y fundamentar su aplicación e interpretación sobre cualquier tipo de datos. Además se busca dar ejemplos de aplicación sobre datos, sacando conclusiones que sean provechosas y que con técnicas univariantes



