\chapter{Métodos supervisados}

%Supongamos que tenemos un conjunto de variables, $\textbf{X}$ que influyen sobre una o más variables conocidas $\textbf{Y}$. A partir de ahora, las llamaremos variables de entrada y variables objetivo respectivamente. El principal propósito de los métodos supervisados, es dado una muestra de individuos con observaciones de ambos tipos de variables predecir la variable objetivo para nuevos individuos de los que solo conozcamos las variables de entrada  
%
%Para empezar tenemos que fundamentar de manera teórica como calcular esa función para predecir. 
%\section{Teoría de decisión estadística}
Sea $X\in \mathbb{R}^p$ un vector aleatorio real e $Y \in \mathbb{R}$ una variable aleatoria real. En este contexto, $X$ e $Y$ serán las variables de entrada y la variable de salida respectivamente. Asimismo, sea $\mathbb{P}(X,Y)$ la distribución de probabilidad conjunta.   

Se busca una función $f(X)$ para predecir $Y$. Dicho predictor tiene asociada una pérdida, es decir, una forma de penalizar el error de predicción. En esta memoria, a no ser que se explicite utilizaremos el error cuadrático para las regresiones, $L(Y,f(X))=(Y-f(X))^2$ . 

\begin{defi}
Llamaremos error de predicción esperado de $f$ o $EPE(f)$ a la siguiente expresión:
\begin{equation}
EPE(f)=E(Y-f(X)^2)=\int (y-f(x))^2 \mathbb{P}(dx,dy)
\end{equation}

\noindent A priori se conocen los valores de $X$, entonces si condicionamos a dichos valores, obtenemos que $\mathbb{P}(Y,X)=\mathbb{P}(Y|X)\cdot\mathbb{P}(X)$ aplicándolo en la expresión anterior resulta que 
\begin{equation}
\begin{split}
EPE(f)&=\int (y-f(x))^2 \mathbb{P}(dx,dy)=\int\int (y-f(x))^2 \mathbb{P}(dy|dx)\mathbb{P}(dx)\\
&= \mathbb{E}_X(\mathbb{E}_{Y|X}((Y-f(X))^2|X))
\end{split}
\end{equation}

\end{defi}

\noindent Este parámetro nos ofrece un criterio para encontrar $f$, es decir, $f$ será la que minimice el $EPE(f)$ , en concreto, $f(x)=\mathbb{E}(Y|X=x)$. Añadir que este sería el caso de la regresión. 

\noindent No obstante, para los casos de \textbf{clasificación} cambia el hecho de que la función cuadrática de pérdida no es adecuada. Sea $G$ una variable categórica con $K$ distintos valores posibles, entonces su función de pérdida se puede expresar como una matriz \textbf{L}. Podemos definir el termino $l_{i,j}=$``Pérdida de clasificar como $G_i$ lo que en realidad es  $G_j$". Lo más frecuente es tomar la pérdida $0-1$, esta asigna 0 a las muestras correctamente clasificadas y 1 a las no categorizadas de manera satisfactoria. 

\noindent Con esta función de pérdida el error de predicción esperado pasa a ser: 

%\input{Documentos Extra/Regresion Lineal Multiple.tex}
%\input{Documentos Extra/Clasificacion Lineal Multiple.tex}
\section{Introducción}
\noindent Sea un caso en el que tenemos unas variables aleatoria \textit{inputs} que influyen o determinan otras variables \textit{outputs}. De estas variables, tomamos observaciones conjuntas de manera que a la hora de predecir las variables respuesta a partir de las variables de entrada. Una vez recogidas estas muestras se planteará un modelo \textit{(Hay varias opciones como se desarrollará a lo largo de este capítulo)}.
\noindent Sea \textbf{x} el vector aleatorio de longitud $p$ formado por las variables de entrada, sea a su vez las variables respuesta $Y_k$. De esta manera el objetivo de los métodos supervisados es conseguir una función $f$ que consiga que \begin{equation}
Y=f(\textbf{x})+\varepsilon
\end{equation}
En otras palabras, conocidas las variables de entrada de una nueva observación poder predecir sus variables de respuesta con la mayor precisión posible. 
\subsection{Mínimos cuadrados y K-vecinos más cercanos}
\noindent Supóngase que tenemos una relación lineal entre las variables de entrada \textbf{x} y la variable de salida de manera que el predictor $\hat{Y}$ tiene la siguiente expresión:
\begin{equation}
\hat{Y}=\hat{\beta}_0+\sum_{j=1}^p \hat{\beta}_j X_j
\end{equation}
\noindent Esta expresión se puede simplificar añadiendo una variable $X_0$ constante 1. Por tanto, tenemos que:
\begin{equation}
\hat{Y}=\textbf{x}^T\hat{\beta}\quad \text{ donde } \hat{\beta}^T=[\hat{\beta}_0\ldots \hat{\beta}_p ]
\end{equation}

\noindent El modelo mostrado se puede generalizar al caso de que se tengan multiples variables de salida. Pero hay que tener en cuenta que las parejas de elementos $(\textbf{x},\hat{Y})$ forman en este caso lineal un hiperplano afín

\noindent Para este y muchos otros casos el modelo ha de ajustarse a los datos recogidos para las $p$ variables entrada y las variables respuesta se tengan en ese momento . Para ello se puede definir la suma residual de cuadrados. 
\begin{equation}
RSS(\beta)=\sum_{i=1}^N(y_i-\textbf{x}_i^T\beta)^2 
\end{equation} 

\noindent En este caso, se considera que la distancia entre los puntos es euclideana, es decir, tomamos como medida del error la suma de las distancias habituales del espacio $\mathbb{R}$ entre las predicciones y los valores reales. Aún así, se puede tomar otras medidas como por ejemplo, el valor absoluto en $\mathbb{R}$.\\
El problema con esta última, es que no es diferenciable y lo que estamos buscando es minimizar $RSS(\beta)$ en función de los paramétros $\beta$.

\noindent La expresión de $RSS(\beta)$ en forma matricial teniendo en cuenta que $\textbf{X}$ sea la matriz de observaciones de tamaño $(n\times p+1)$ que tiene la primera columna constante 1 y $\textbf{y}$ es el vector columna o matriz que contiene las observaciones de la o las variables respuestas  es la siguiente: 
\begin{equation}
RSS(\beta)=(\textbf{y}-\textbf{X}^T\beta)^T(\textbf{y}-\textbf{X}^T\beta)
\end{equation}
Que al derivarla respecto de $\beta$ e igualarlo a $0$ se obtiene que:
\begin{equation}
\textbf{X}^T(\textbf{y}-\textbf{X}\beta)=0
\end{equation}
Que tiene solución única si la matriz $\textbf{X}^T\textbf{X}$ es invertible resultando en que $\hat{\beta}=(\textbf{X}^T\textbf{X})^{-1}\textbf{X}^T\textbf{y}$.

\noindent Otra forma de crear predictores es mediante los k-vecinos más cercanos que toma como predicción de una nueva observación el valor medio de las variables respuesta de las k observaciones más cercanas a la nueva muestra. 
\begin{equation}
\hat{Y}(x)=\dfrac{1}{k}\sum_{\textbf{x}_i\in N_k(x)}y_i
\end{equation}
\noindent En estos casos se necesita definir una distancia sobre el conjunto de las observaciones para poder establecer los $N_k(x)$ que son los entornos. 

\noindent Estos dos acercamientos tienen problemas a la hora de las suposiciones que se hacen. El primero asume que la estructura de los datos es lineal de manera global y el segundo asume que es constante localmente. 

\noindent Además el método de los k vecinos sufre de lo que se llama la \textit{maldición de la dimensionalidad}. Este suceso consiste en que a medida que aumentamos la dimensión, en este caso la cantidad de variables aleatorias a observar, la densidad de los datos es cada vez menor, en particular es proporcional a $n^{\frac{1}{p}}$, donde $p$ es el número de variables y $n$ el número de observaciones. Por ejemplo, si una muestra densa en $1$ dimensión tiene $n=100$, para mantener dicha densidad en $p=10$ se necesitarían $n = 100^{10}$. 

\subsection{Decisión Estadística}
\noindent Sea un vector aleatorio real de entrada $\textbf{x}\in\mathbb{R}^p$, una variable de salida $Y\in\mathbb{R}$ y sea la probabilidad conjunta $\mathbb{P}(Y,\textbf{x})$ de ambas, entonces se busca una función $f(\textbf{x})$ para predecir $Y$. La búsqueda de esta función requiere una forma de saber como de correcta son esas predicciones. Para ello, se definen las \textit{funciones de pérdida}, la más utilizada es el error cuadrático $L(Y,f(\textbf{x}))=(Y-f(\textbf{x}))^2$
\begin{defi}
Teniendo en cuenta lo anterior, se define el \textit{error de predicción esperado} como la esperanza de la nueva variable que define la función de pérdida.
\begin{equation}
\begin{split}
EPE(f)& = E(Y-f(\textbf{x}))^2\\
&= \int [y-f(\textbf{x})]^2\mathbb{P}(dx,dy)
\end{split}
\end{equation}
\end{defi}

\noindent Para ajustarlo a la situación de predecir el valor de $Y$ para nuevos valores observados de $\textbf{x}$, se puede condicionar respecto del valor de $\textbf{x}$ obteniendo la siguiente expresión:
\begin{equation}
 EPE(f) = E_{\textbf{x}} E_{Y|\textbf{x}}([Y-f(\textbf{x})]^2|\textbf{x})
\end{equation} 
Esta expresión otorga según Hastie \textit{et.al} \cite{Hastie 2001} un criterio para elegir la $f$. Basta con tomar $f$ de manera que: 
\begin{equation}
 f(x)=\text{argmin}_c E_{Y|\textbf{x}}([Y-c]^2|\textbf{x}=x)
\end{equation}

\noindent Por tanto, el que minimiza en el caso de la pérdida establecida anteriormente es $f(x)=E(Y|\textbf{x}=x)$. 
\begin{defi}
\noindent Se llama \textit{función de regresión} a la función $f(x)=E(Y|\textbf{x}=x)$
\end{defi}

\emph{Aquí podría hacer ejemplos el k- vecinos y el lineal}

\noindent Sea el caso de una variable de salida categórica $G$ que toma $k$ valores de manera que el predictor $\hat{G}$ tambien toma esos valores, entonces la única modificación que debemos hacer es la de la función de pérdida que pasa a ser una matriz $\textbf{L}$ de tamaño $k\times k $ donde $L_{i,j}$ es la penalización por categorizar como $\mathcal{G}_j$ algo que en realidad es $\mathcal{G}_i$. Entonces tenemos que \textit{el error de predicción esperado} es: 
\begin{equation}
EPE = E[L(G,\hat{G})]= E_{\textbf{x}}\sum_{i=1}^k L[\mathcal{G}_k, \hat{G}(\textbf{x})]\mathbb{P}(\mathcal{G}_k|\textbf{x})
\end{equation} 
Esta definición otorga un criterio análogo para establecer el predictor $\hat{G}(x)$, que si se utiliza la pérdida 0-1, es decir, aquella que otorga una penalización de 1, se obtiene el \textit{clasificador bayesiano}:
\begin{defi}
Se llama \textit{clasificador bayesiano} al predictor 
\begin{equation}
\hat{G}(x)=\mathcal{G}_k \text{ si } \mathbb{P}(\mathcal{G}_k|\textbf{x}=x)=\max_{g\in \mathcal{G}}\mathbb{P}(g|\textbf{x}=x)
\end{equation}

\noindent Es decir, se otorga la categoría que más probabilidad de ser tiene conociendo el valor de $\textbf{x}$
\end{defi}

\noindent Una característica que se puede intuir es que el problema de encontrar el predictor se puede traducir como un problema de aproximación de funciones en $\mathbb{R}^p$ conocidos $n$ puntos. 

\subsection{Selección del modelo, Sesgo y Varianza}

Una vez que hemos ajustado el modelo a los puntos hay que comprobar su rendimiento, esto se puede hacer teniendo un conjunto de entrenamiento y otro de testing. El conjunto de entrenamiento nos servirá para optimizar los parámetros y el de testing del cual conocemos las x y las y nos permite evaluar su capacidad de generalización. 

\noindent Sea el modelo $Y=f(\textbf{x})+\varepsilon$ donde $E(\varepsilon)=0$ y $Var(\varepsilon)=\sigma^2$. Entonces podemos hacer la siguiente descomposición del error de generalización  esperado en $x_0$, no perteneciente al conjunto de entrenamiento:
\begin{equation}
EPE(x_0)=E(Y-\hat{f}(x_0)|\textbf{x}=x_0)^2=\sigma^2+sesgo^2(\hat{f}(x_0))+Var_{\mathcal{T}}(\hat{f}(x_0))
\end{equation}

\noindent De esta manera, se puede reducir el sesgo y la varianza, pero hay un error que es irreducible que es $\sigma^2$ y los otros dos términos son el resultado de calcular el Error Cuadrático Medio. En términos generales cuanto más complejo es el modelo se tiene 













































































































\newpage
\section{Métodos Lineales para regresión}
\noindent La regresión es la búsqueda de una relación funcional,  entre una variable respuesta cuantitativa a partir de un conjunto de variables de entrada. 
Teniendo en cuenta el modelo aditivo que se detallaba anteriormente, tendremos en cuenta que la variable de respuesta $Y$ viene dada por una relación de la forma:
\begin{equation}
Y=f(\textbf{x})+\varepsilon
\end{equation}

\noindent En el caso de los modelos lineales se supone que la función de regresión, $f$ es una función lineal de las variables de entrada del vector aleatorio $\textbf{x}$, $X_1\ldots X_p$. De esta manera se tiene. 
\begin{equation}
f(\textbf{x})=\beta_0+\sum_{j=1}^p X_j\beta_j
\end{equation}

\noindent Por tanto, la predicción de la variable de salida $\hat{y
}_i$ para una observación $\textbf{x}_i$ del vector aleatorio  se puede expresar de la siguiente manera:
\begin{equation}
\hat{y}_i=f(\textbf{x}_i)=\beta_0+\sum_{j=1}^p x_{ij}\beta_j
\end{equation}

\noindent De esta manera, se tiene un vector de $p$ parámetros $\beta_1 \ldots \beta_p$ además de otro $\beta_0$. Esto se puede simplificar añadiendo al vector aleatorio una variable aleatoria que sea constantemente 1. De esta manera, sea $\beta^T=[\beta_0,\beta_1\ldots \beta_p]$ y una observación $\textbf{x}_i^T=[1,x_{i1},\ldots x_{ip}]$ la expresión anterior se simplifica de la siguiente manera: 
\begin{equation}
\hat{y}_i= f(\textbf{x}_i)=\textbf{x}_i^T\beta
\end{equation}

\noindent Y en general, para una matriz de datos $\textbf{X}$ de tamaño $n\times p $ al que se le añade una primera columna de 1's por tanto, acaba siendo una matriz de tamaño $n\times (p+1)$ y sea $\hat{\textbf{y}}=[\hat{y}_1,\ldots \hat{y}_p]$ el vector de predicciones. Entonces, se obtiene la siguiente expresión  simplificada:
\begin{equation}
\hat{\textbf{y}}=\textbf{X}\beta
\end{equation} 

En las siguientes secciones se detallarán el calculo de los parámetros $\beta$. 

\subsection{Ajuste de los parámetros por mínimos cuadrados}
\noindent Sea una matriz de datos $\textbf{X}$ de tamaño $n\times p+1$ resultado de hacer $n$ observaciones de $p$ variables aleatorias y añadir en primera instancia de cada observación un 1. Sea también un vector de respuestas $\textbf{y}^T=[y_1,\ldots y_p]$ de tamaño $n$. 

\noindent Tomando la suma de los errores cuadrados y minimizando se puede obtener una estimación de máxima verosimilitud del vector de parámetros $\beta$. Utilizando las expresiones anteriores:
\begin{align}
RSS(\beta)=\sum_{i=1}^n(y_i-\hat{y}_i) &= \sum_{i=1}^n(y_i-\textbf{x}_i^T\beta )=\textbf{y}-\textbf{X}\beta\\
\intertext{Derivando respecto al vector de parámetros $\beta$:}
\dfrac{\partial RSS (\beta)}{\partial \beta}&=-2\textbf{X}^T(\textbf{y}-\textbf{X}\beta)\\
\intertext{Y la segunda derivada respecto de $\beta^T$: }
\dfrac{\partial^2 RSS (\beta)}{\partial \beta \partial \beta^T} &=  -2 \textbf{X}^T\textbf{X}
\end{align}

\noindent Asumiendo que $\textbf{X}$ es una matriz de rango máximo, entonces, la matriz $\textbf{X}^T\textbf{X}$ es definida positiva, por tanto la solución de la siguiente ecuación:
\begin{equation}
\textbf{X}^T(\textbf{y}-\textbf{X}\beta)=0
\end{equation}
Es un mínimo local de $RSS(\beta)$ , si además, $\textbf{X}^T\textbf{X}$ es una matriz invertible, entonces tiene solución única y es :
\begin{equation}
\hat{\beta}=(\textbf{X}^T\textbf{X})^{-1}\textbf{X}^T\textbf{y}
\end{equation}

\begin{propo}
Esta estimación de los parámetros $\beta$ es equivalente a la estimación de estos mediante el método de máxima verosimilitud
\begin{proof}

\end{proof}
\end{propo}

\noindent Por tanto, los valores predichos $\hat{\textbf{y}}$ se calculan de la siguiente manera:
\begin{equation}
\hat{\textbf{y}}=\textbf{X}\hat{\beta}=\textbf{X}^T(\textbf{X}^T\textbf{X})^{-1}\textbf{X}^T\textbf{y}
\end{equation}

\subsection{Regresión Múltiple mediante Ortogonalización sucesiva }
\subsection{Regresión de múltiples variables respuesta}
\subsection{Métodos de Encogimiento}















\newpage
\section{Métodos de Clasificación y Discriminación}

\noindent Los métodos de clasificación buscan separar los elementos de un espacio de observaciones en grupos conocidos previamente. 

\begin{defi}
Llamaremos \textit{discriminación o análisis discriminante} a aquel que busca describir las características principales de cada uno de los grupos o clases mediante \textit{funciones discriminantes} 
\end{defi}

\begin{defi}
Llamaremos \textit{métodos de clasificación} a aquellos que dada una nueva observación, \textbf{x}, buscan predecir con la máxima precisión a que clase pertenecen mediante \textit{reglas de clasificación}
\end{defi}

\noindent Hay que señalar que no siempre hay una diferencia clara entre ambas disciplinas y que puede haber veces que ambas se solapen. 


\subsection{Análisis Discriminante}

\noindent Según \textit{Lebart L., Morineau, A. y Warwick K.M.} \cite{Lebart 1984} el análisis discriminante es un conjunto de técnicas que permiten describir y clasificar un gran número de observaciones de las cuales se han medido una gran cantidad de variables. 

\noindent El método que se describe aquí es un método supervisado ya que se conocen las clases a las que pertenecen cada una de las observaciones. 



\subsection{Formalización del Análisis Discriminante}
 
\noindent Sea \textbf{X} la matriz de datos de tamaño $n \times p$
donde las filas $\textbf{x}_i$ son cada una de las observaciones de las $p$ variables. Dichas observaciones están particionadas en general por $q$ grupos, sea $I_k$ el conjunto de observaciones pertenecientes al $k$-ésimo grupo, sea también $n_k$ el número de observaciones que pertenecen al $k$-ésimo grupo.

\noindent Se definen las medias muestrales $\overline{x}_j=\frac{1}{n}\sum_{i=1}^n x_{ij}$ de la $j$-ésima variable en la población en total. También se define la media muestral dentro de cada grupo que es $\overline{x}_{jk}=\frac{1}{n_k}\sum 
_{i\in I_k} x_{ij}$ 

\noindent Por ende podemos dar la distancia entre dos variables como:
\begin{align}
Cov(X_j,X_{j'})&=\dfrac{1}{n}\sum_{i=1}^n(x_{ij}-\overline{x}_j)(x_{ij'}-\overline{x}_j')
\intertext{Esto se puede particionar por grupos de la siguiente manera: }
Cov(X_j,X_{j'})&=\dfrac{1}{n}\sum_{k=1}^q\sum_{i\in I_k}(x_{ij}-\overline{x}_j)(x_{ij'}-\overline{x}_j')
\intertext{y a su vez cada uno de los $(x_{ij}-\overline{x}_j)$ se pueden dividir en la parte intergrupos e intragrupos: }
(x_{ij}-\overline{x}_j)&=(x_{ij}-\overline{x}_{jk})+(\overline{x}_{jk}-\overline{x}_{j})
\end{align}
Sustituyendo y simplificando lo necesario: 
\begin{equation}
Cov(X_j,X_{j'})=\dfrac{1}{n}\sum_{k=1}^q\sum_{i\in I_k}(x_{ij}-\overline{x}_{jk})(x_{ij'}-\overline{x}_{j'k})+\sum_{k=1}^q\dfrac{n_k}{n}(\overline{x}_{jk}-\overline{x}_{j})(\overline{x}_{j'k}-\overline{x}_{j'})
\end{equation}

\noindent Esto nos permite dar una descomposición de la matriz de covarianzas total de la siguiente forma :
\begin{equation}\label{descomposicion varianza}
\textbf{T}=\textbf{B}+\textbf{W}
\end{equation}

Donde:
\begin{itemize}
\item \textbf{T} es la matriz que expresa la covarianza total y sus coeficientes  $t_{jj'}=Cov(X_j,X_j')$
\item \textbf{B} es la matriz que expresa la covarianza entre los grupos y sus coeficientes son $b_{jj'}=\sum_{k=1}^q\dfrac{n_k}{n}(\overline{x}_{jk}-\overline{x}_{j})(\overline{x}_{j'k}-\overline{x}_{j'})$
\item \textbf{W} es la matriz que expresa la covarianza dentro de los grupos y sus coeficientes son $w_{jj'}=\dfrac{1}{n}\sum_{k=1}^q\sum_{i\in I_k}(x_{ij}-\overline{x}_{jk})(x_{ij'}-\overline{x}_{j'k})$
\end{itemize}

\noindent Para cualquier combinación lineal que se quiera hacer de las variables de entrada de la forma $\textbf{a}^T \textbf{x}$, donde el vector $\textbf{a}$ es un vector de $p$ constantes,  entonces la varianza se transforma de la siguiente manera:
\begin{equation}
Var(\textbf{a}^T \textbf{x})=\textbf{a}^T \Sigma \textbf{a}
\end{equation}

\noindent Entonces transformando por el vector $\textbf{a}$ tenemos que la Ecuación \eqref{descomposicion varianza} se transforma de la siguiente manera: 
\begin{equation}
\textbf{a}^T \textbf{T}\textbf{a}= \textbf{a}^T \textbf{B}\textbf{a}+\textbf{a}^T \textbf{W}\textbf{a}
\end{equation}

\noindent Recopilando, el objetivo del análisis discriminante lineal es encontrar combinaciones lineales que maximicen la varianza entre grupos y minimicen la varianza dentro de los grupos. Eso es equivalente a encontrar el vector $\textbf{a}$ tal que:
\begin{equation}
f(\textbf{a})=\dfrac{\textbf{a}^T \textbf{B}\textbf{a}}{\textbf{a}^T \textbf{T}\textbf{a}}
\end{equation}
\noindent Es máxima. Si además utilizamos la restricción $\textbf{a}^T \textbf{T}\textbf{a} = 1$. En principio la función objetiva es homogénea, es decir, $f(\mu \textbf{a})=f(\textbf{a})$
Utilizando el método de los multiplicadores de Lagrange derivamos respecto del vector $\textbf{a}$ tendremos que:
\begin{align}
L(\textbf{a})&= \textbf{a}^T \textbf{B}\textbf{a}-\lambda(\textbf{a}^T \textbf{T}\textbf{a}-1) 
\intertext{al derivarla respecto de \textbf{a} se obtiene que: }
\dfrac{\partial L(\textbf{a})}{\partial \textbf{a} } &= 2\textbf{B}\textbf{a}-2\lambda\textbf{T}\textbf{a}
\intertext{En consecuencia: }
\textbf{B}\textbf{a} &= \lambda \textbf{T} \textbf{a}
\intertext{Si además \textbf{T} es no singular}
\textbf{T}^{-1}\textbf{B}\textbf{a}&=\lambda \textbf{a}
\end{align}

Es decir, el vector $\textbf{a}$ es el vector de valor propio $\lambda$, tomando el valor propio máximo de la matriz $\textbf{T}^{-1}\textbf{B}$.

\begin{defi}
Al valor $\lambda$ se le conoce como \textit{potencia discriminante} de la combinación \textbf{a}.
\end{defi}

\noindent \textit{Observación} Esta técnica se diferencia del Análisis de Componentes Principales en que en el Análisis Discriminante se maximiza la distancia o variación entre grupos conocidos, mientras que el Análisis de Componentes Principales únicamente busca las direcciones en las que los datos varían más. 

\noindent Pero, en caso de ser aplicada, el análisis discriminante desarrollado aquí se puede utilizar como método de reducción de la dimensionalidad para el caso de la clasificación. De esta manera, utilizando mecanismos similares a los que se describirán en la parte del Análisis de Componentes Principales la matriz de datos puede ser reducida. 

\subsection{Métodos de Clasificación}

\noindent En la introducción de este capítulo se detalló que el clasificador Bayesiano utilizaba las probabilidades conocido el valor de la observación $\textbf{x}_0$. El caso de la regresión logística intenta modelizar el cociente de las probabilidades como una función lineal. 

\noindent \textit{Hastie et.al.}\cite{Hastie 2001} detallan el caso en profundidad cuando la variable $Y$ tiene dos posibles valores.
\begin{equation}
log \dfrac{P(Y=1|\textbf{x}=\textbf{x}_0)}{P(Y=2|\textbf{x}=\textbf{x}_0)}=\beta^T \textbf{x}_0
\end{equation}

\noindent Donde $\beta$ es un vector de longitud $p+1$ y a $\textbf{x}_0$ es una observación del vector aleatorio y se le ha añadido en la primera componente el valor constante 1.

\noindent Esto sería el caso para $2$ valores posibles pero en el caso de que se tuviesen $K$ valores posibles, se definen las siguientes funciones lineales:
\begin{equation}
log \dfrac{P(Y=j|\textbf{x}=\textbf{x}_0)}{P(Y=K|\textbf{x}=\textbf{x}_0)}=\beta_j^T \textbf{x}_0 \quad j=1\ldots K-1
\end{equation}
Por tanto, el modelo queda definido por $K-1$ funciones lineales. Es decir estamos asumiendo que la probabilidad $P(Y=j|\textbf{x}=\textbf{x}_0)$ viene dada por una función logística de 

\newpage
\section{Redes Neuronales}

\noindent Las redes neuronales artificiales, son modelos predictivos basados en el funcionamiento de las propias neuronas del cerebro. Éstas reciben señales de entrada de las neuronas con las cuales están conectadas, las procesan y envían el resultado a las  neuronas siguientes. 

\noindent La ventaja de este tipo de algoritmos es que en esencia, son un conjunto de parámetros y funciones de activación que pueden ser ajustados para cualquier tarea y cualquier tipo de función a aproximar. Solo hace falta la complejidad del modelo adecuada según Hornik, Stinchcombe y  White \cite{Hornik 1989}.  

\noindent Una neurona artificial es mucho más simple que una neurona. López,  Balsa-Canto y  Oñate, definen una neurona en términos matemáticos \cite{Roberto 2008}. Antes hay que definir los siguientes conceptos, para los cuales se han utilizado como base \cite{Grossi 2007, Neural Designer}.

\noindent Sea un vector aleatorio $\mathbf{x}$ de longitud $p$, llamamos datos de entrada al vector $\mathbf{x}_0$ que representa cada una de las observaciones de las $p$  variables medidas. Por otro lado, se llamará datos de salida al vector $\mathbf{y}_0$ obtenido tras haber introducido en la red neuronal el vector $\mathbf{x}_0$.

\noindent Ahora se detallan todos los elementos de una neurona. 

\begin{defi}
Llamaremos pesos sinápticos $\omega$ de una neurona al vector de $p$ constantes que regulan la importancia de cada entrada en la neurona.  A este vector de pesos sinápticos se le puede añadir un término independiente que únicamente se sumará. Se llama sesgo y se denota como $b$.
\end{defi}

\begin{defi}
Se llama función de activación, $f$ de una neurona artificial a la función que transforma la suma ponderada de las entradas para obtener la salida. 

\noindent Las funciones de activación más habituales son: la función identidad (en este caso, es como si se hiciera una simple suma ponderada de los datos de entrada) , la función sigmoide, la tangente hiperbólica o la función lineal rectificada para casos de regresión, es decir, en casos en los que la variable respuesta sea continua. En caso contrario, se pueden utilizar la función softmax o la función logística, ya que devuelven valores en el intervalo $[0,1]$ y se puede asociar con la probabilidad de pertenecer a una clase u otra. 
\end{defi}

\begin{defi}
Una neurona artificial procesa una entrada $\textbf{x}$ de acuerdo con unos pesos sinápticos $(b,\omega)$ que luego es transformada por una función de activación $f(\mathbf{x})$.

\noindent Una  vez definidos los elementos que forman una neurona artificial se puede definir la siguiente función:
\begin{equation}
\begin{split}
g:\mathbb{R}^p &\longrightarrow \mathbb{R}\\
g(\textbf{x})&\longrightarrow g(\textbf{x};b,\omega)
\end{split}
\end{equation}
\begin{equation}
g(\textbf{x})=f\left(b+\sum_{i=1}^p \omega_i x_i\right)
\end{equation}
El siguiente diagrama \ref{fig:neurona-biológica} proporciona una forma sencilla de entender el funcionamiento de dicho modelo, incluyendo la analogía de las neuronas biológicas. 
\end{defi}
\begin{figure}
\begin{center}
%%Hay que pedirle a Carlos que lo edite por que yo la verdad que no se 
\includegraphics[scale=0.6]{Documentos Extra/Imagenes/neurona.png}
\caption{Representación de una neurona en la que se comparan los elementos biológicos y artificiales. Obtenida de \cite{Requena}}
\label{fig:neurona-biológica}
\end{center}
\end{figure}


\noindent La principal ventaja de estos métodos es que las neuronas se pueden conectar entre ellas, es decir, estas se pueden organizar de manera que los datos de salida de un conjunto de neuronas sirvan como  entrada del siguiente.

\begin{defi}
Se llama capa de neuronas al conjunto de neuronas artificiales que tienen el mismo conjunto de datos de entrada y cuyos datos de salida son la entrada del siguiente.
\end{defi}
\noindent Se pueden establecer varios tipos de capas de neuronas \cite{Neural Designer}:
\begin{defi}
Se llama capa de entrada a la primera capa de neuronas que recibe los valores de las observaciones y las estandariza (Se debe entrenar al modelo para ello).
\end{defi}
\begin{defi}
Se llama capa oculta a cada una de las capas intermedias que se utilizan en las redes neuronales. 
\end{defi}

\begin{defi}
Se llama capa de salida a la última capa que tiene tantas neuronas como variables respuesta y sus datos de salida son las predicciones que hace la red neuronal de las variables respuesta. 
\end{defi}

\noindent Para el proceso de ajuste se utiliza de manera habitual el método del gradiente con un conjunto de datos con $N$ observaciones. En el capitulo  11 Hastie et.al.  detallan en profundidad el ajuste \cite{Hastie 2001}. A este proceso se le llama \emph{back-propagation}, ya que una vez aplicado el método del gradiente se van actualizando los parámetros anteriores. 

\noindent Si queremos expresar el modelo de manera concisa, la complejidad de interpretación aumenta significativamente al expandir la red neuronal. Incluso en casos simples como en la figura  \ref{fig:estructura red neuronal} la interpretación se vuelve complicada. Por lo tanto, las redes neuronales se utilizan principalmente con propósitos predictivos, en lugar de brindar una explicación clara de los parámetros y relaciones involucrados en el modelo \cite{Hastie 2001, James 2013}.

\noindent Las principales ventajas de las redes neuronales es que pueden ajustarse a cualquier estructura sin conocerla a priori. Por otro lado, debido a la gran cantidad de parámetros a ajustar pueden tender al sobreajuste.

\noindent La siguiente imagen \ref{fig:estructura red neuronal} es una representación de una red neuronal como un grafo, en el que cada nodo es una neurona, en particular, los capas azules son capas ocultas, formando cinco capas ocultas, mientras que las amarillas son capas de entrada, y las rojas de salida. 

\begin{figure}[ht]
\centering
\includegraphics[scale=0.35]{Documentos Extra/Imagenes/red-neuronal-grande.png}
\caption{Imagen extraída directamente de www.neuraldesigner.com}
\label{fig:estructura red neuronal}
\end{figure}


\noindent En esta memoria se han detallado los tipos más básicos de neuronas. Hay tareas específicas que este tipo de neuronas no pueden afrontar, por ejemplo, en el caso de datos que proceden de series temporales en las que estados previos influyen en los estados futuros como puede ser predicciones meteorológicas, bursátiles y otros. Es por ello que  se han desarrollado un tipo más complejo de neuronas llamadas LSTM \emph{(Long-Short Term Memory)} de las que se puede ver su desarrollo y definición además de las propiedades que poseen en \cite{Hochreiter 1997,Neural Designer}.


\newpage
\section{Árboles de Regresión y Clasificación}
\noindent Sea un vector aleatorio $\textbf{x}$ con $p$ los datos de entrada, e $Y$ la variable respuesta. Supóngase también que se toman $n$ observaciones obteniéndose parejas $(\textbf{x}_i,y_i)$. De esta manera, tenemos que las $\textbf{x}_i\in \mathbb{R}^p$.\\
Los métodos de árboles intentan dividir el espacio $\mathbb{R}^p$ y luego en cada región del espacio se ajusta un modelo más simple, incluso una constante.\\
La ventaja de este tipo de métodos es que son fácilmente interpretables, ya que aunque no sea fácil representar el espacio $\mathbb{R}^p$, permiten ser representados como un diagrama de árbol.\\ 
Las  siguientes imágenes procedentes de \textit{Hastie et. al.}\cite{Hastie 2001} muestran el diagrama resultante tras dividir el espacio de observaciones mediante un árbol, 

\begin{figure}[h]
 \centering
  \subfloat[División de $\mathbb{R}^p$]{
   \label{f:división}
    \includegraphics[width=0.4\textwidth]{Documentos Extra/Imagenes/Regiones árboles.png}}
  \subfloat[Diagrama resultante]{
   \label{f:diagrama arbol}
    \includegraphics[width=0.4\textwidth]{Documentos Extra/Imagenes/Diagrama de arbol.png}}
 \caption{Representación de la división de $\mathbb{R}^p$ y el diagrama de árbol resultante}
 \label{f:MARC1}
\end{figure}


\subsection{Árboles de Regresión}

\subsection{Árboles de Clasificación }